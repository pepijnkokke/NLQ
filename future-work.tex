\section{Future work}

\paragraph*{Integrate Focusing and Display Logic}

\paragraph*{Deep Inference Sequent Calculus}
In \autoref{sec:why-use-display-calculus}, we discussed that in order
to use backward-chaining proof search, we have to ensure that our
structural rules have the sub-structure property, or at very least
only cause predictable loops. \citet{gore2014} demonstrate a
methodology for constructing a deep inference sequent calculus from a
display calculus. Deep inference calculi have the advantage that they
naturally have the sub-structure property, which means that they are
suitable for naive backward-chaining search. It would be interesting
to employ this methodology, and construct a deep inference sequent
calculus for the system constructed in this thesis.

\paragraph*{Forward-Chaining Proof Search}
In \autoref{sec:what-is-type-logical-grammar} it was mentioned that
most research focuses on implementing what I call the `semantic
function' (i.e.\ interpreting). This is a good approach for research:
we can limit ourselves to sequent calculus, which has pleasant
properties, refer to the huge body of work on generative grammar to
inform our choice for sentence structure, and simply focus on making
these known structures derivable. However, in order to be feasible in
a practical system, one must also implement what I call the `syntactic
function' (i.e.\ parsing).

One way to include parsing is by switching to forward-chaining proof
search. In principle, we can do this by constructing all possible
sentences based on the given words, and filtering on those which are
both pronounceable and maintain the correct word-order. Ideally,
though, we would also have an efficient implementation, for instance
one based on the technique of magic sets as developed by
\citet{bancilhon1985}.

%%% Local Variables:
%%% mode: latex
%%% TeX-master: t
%%% End:
