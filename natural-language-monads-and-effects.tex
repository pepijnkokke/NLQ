\section{Natural Language, Monads, and Effects}
\label{sec:natural-language-monads-and-effects}

In sections \autoref{sec:introduction} and
\autoref{sec:display-calculus} we presented a simple type-logical
grammar, and its encoding as a display calculus. In this section, we
will study extensions of the semantic calculus, which will allow us to
analyse a wider range of linguistic phenomena. In the next section, we
will discuss extensions of the syntactic calculus.

Since its original formulation by \citet{lambek1961}, many variants of
the Lambek calculus have been proposed
\citep{steedman1988,moortgat2012,morrill2011,kubota2012,barker2015}.
However, none have yet become truly canonical.
Recently, \citet{moot2015} has begun comparing these extended calculi,
as fragments of first-order linear logic. In time, this may yield a
canonically agreed-upon extension to NL.

\lamET, too, has been extended and revisited many times. Many of its
extensions were created to deal with complex semantic phenomena, such
as intensionality~\citep{winter2009},
expressives~\citep{potts2003,mccready2010,gutzmann2011}, dynamic
semantics~\citep{groenendijk1995}, and quantifier
raising~\citep{barker2015}. \todo{Other important work?}
In \citeyear{shan2002}, \citeauthor{shan2002} proposed an interesting
paradigm to unify these extensions: by implementing them using
techniques for effectful functional programming in \lamET.

In this section, we will discuss \citeauthor{shan2002}'s proposed
extensions to \lamET and its limitations. In the next section, we will
then turn our attention to the extensions made to NL, and see how they
can remedy the limitations of the extended \lamET. We will conclude
the thesis by giving our own extensions to NL, and showing that they
preserve the properties we expect of a grammar logic
(i.e. admissibility of cut, and a decidable and complete proof search
procedure).

\subsection{Monads and Extensible Effects}
In \citeyear{shan2002}, \citeauthor{shan2002} proposed to analyse a
wide range of linguistic phenomena using monads. He defines several
monads which deal with interrogatives, focus, intensionality, binding,
and quantification. \citet{bumford2013}, \citet{charlow2014} and
\citet{barker2015} continued this line of research, defining monads to
deal with a large range of linguistic phenomena.

Formally, a monad is
\begin{enumerate*}[label=(\arabic*)]
\item a type-level constructor, $\mathbb{M}$, mapping each type A
  to a corresponding type $\mathbb{M}A$; and
\item a pair of functions, η and $\star$ (pronounced ``unit'' and
  ``bind''), with the following types\footnote{
    In addition, these functions have to obey the monad laws: left
    identity ($M\star\eta N\equiv M\;N$); right identity
    ($\eta\star M\equiv M$); and associativity ($M\star(\lambda
    x.N\;x\star O) \equiv (M\star\lambda x.N\;x)\star O$).
  }:
\end{enumerate*}
\[
  η:A\ra\mathbb{M}A
  \qquad
  \star:(A\ra\mathbb{M}B)\ra\mathbb{M}A\ra\mathbb{M}B
\]
There are many ways to implement monadic semantics. The most
conventional of these is to apply the monadic translation, as
described by \citet{moggi1991}:
\[
  \begin{aligned}
    &(A\ra B)^\text{M} &&= A^M\ra\mathbb{M}B^M\\
    &A^\text{M}        &&= A
  \end{aligned}
  \qquad
  \begin{aligned}
    &\text{lift}\;x           &&= \eta\;x\\
    &\text{lift}(\lambda x.M) &&= \eta(\lambda x.\text{lift}\;M)\\
    &\text{lift}(M\;N)        &&= (\lambda f.f\star(\text{lift}\;N))\star\text{lift}\;M
  \end{aligned}
\]
Another possibility is to modify our translation to semantic calculus
to insert the monadic operators. If we choose to do this, we can use
the information present in out syntactic calculus to guide our
translation. For instance, we could simply choose to modify our
translation on types to insert an `$\mathbb{M}$' over every atomic
type:
\[
  \tr[\S]=\mathbb{M}\t
  \quad
  \tr[\N]=\mathbb{M}(\e\ra\t)
  \quad
  \tr[\NP]=\mathbb{M}\e
\]
Whichever choice we make, the important point is that the insertion of
the monad constructor $\mathbb{M}$ in our types gives us the
possibility to implement any sort of ``plumbing'' we need in our
lexical entries, as long as it forms a monad.

\vspace*{1\baselineskip}

As an example, we can use monads to analyse expressive content. This is
content that is present in the sentence meaning, but does not directly
affect the truth-conditional meaning. It is information present on a
sort-of side channel. For instance, in ``I walked the damn dog,'' the
word `damn' does not seem to contribute to the truth-conditional
meaning, as the utterance would still be considered truthful if the
dog is well-liked.

We can implement this using a variant of the writer monad: we
represent values of the type $\mathbb{M}A$ as a tuple of
truth-conditional (or ``at-issue'') content, and expressive content:
\begin{align*}
  \mathbb{M}A    &= A\times\t                                         \\
  \eta\;M        &= (M,\text{true})                                   \\
  M\star N       &= \case{M}{x}{y}{(\case{N\;x}{z}{w}{(z,y\land w)})} \\
  \text{tell}(M) &= ((),M)                                            \\
  \intertext{%
  Using this monad, we can define a small lexicon. We lift our regular
  entries into monadic entries:
  }
  \text{john}  &= \eta\;\JOHN\\
  \text{walks} &= \lambda y\;x.(\lambda x'.(\lambda y'.\eta(\WALK(x,y)))\star y)\star x\\
  \text{the}   &= \lambda f.(\lambda f'.\iota(f'))\star f\footnotemark\\
  \text{dog}   &= \eta\;\DOG\\
  \intertext{%
    We treat `damn' as an identity function in its at-issue
    content---it binds $x'$, then returns it. However, we also define
    `damn' as expressing some sort of displeasure, represented as the
    proposition $\DAMN$ in its expressive content:
  }
  \text{damn} &= \lambda f.(\lambda f'.(\lambda().\eta\;f')\star\text{tell}(\DAMN(f')))\star f
\end{align*}
\footnotetext{%
  The semantics for `the' are commonly given in terms of a function
  called `$\iota$', known as the ``definite description operator.''
  Given a set, this operator returns its unique inhabitant. Depending
  on the exact semantics you want for this operator, it can either be
  implemented monadically, as a side-effect, or using
  quantification. For this example, it is best to think of `$\iota$'
  as a function of the type $(\e\t)\e$.
}%
The entire utterance ``I walked the damn dog'' then reduces as follows:
\[
  (\text{walks}\;(\text{the}\;(\text{damn}\;\text{dog}))\;\text{john})
  \mapsto
  (\WALK(\JOHN,\iota(\DOG)), \DAMN(\DOG))
\]
The above analysis is rather coarse, as it does not capture any
displeasure towards the \emph{specific} dog. We \emph{can} get a more
precise meaning, but doing so complicates the example too much.

\vspace*{\baselineskip}

Monads have one big problem: modularity. There is no general procedure
which can compose two arbitrary monads $\mathbb{M}_1$ and
$\mathbb{M}_2$ into a new monad $\mathbb{M}_3 = \mathbb{M}_1 \circ
\mathbb{M}_2$. This means that it is not trivial to separate different
types of effects---\emph{all} side-effects have to be implemented in
one single, monolithic monad.

\citet{shan2002} mentions monad morphisms or transformers as a
solution to mitigate---but not solve---the problem.
Monad transformers were introduced by \citet{liang1995}. In short,
they are functions $\mathbb{T}$ from monads to monads. Transformers
can be chained together, to create combined monads consisting of
``layers'' of elementary monads. Because different monads combine in
different ways, the programmer has to manually define these
transformers, and has to to specify how effectful operations `lift'
through each monad transformer.
One problem with monad transformers is that the order of the
``layers'' is determined statically, and cannot easily be changed in
various parts of the program. In addition, every effectful operation
has to be lifted into this layercake of side-effects. This means that
that every effectful function, or lexical entry, has access to
\emph{all} side-effects, and every effectful function has to be
altered if a new layer is added. It is clear that monad transformers
offer a sub-par solution to the problem.\footnote{%
  The Haskell community is split over whether or not monad
  transformers are useful in practice, but many people---including the
  GHC developers---prefer ``rolling'' their own monolithic monad,
  which includes all required effects, over using monad transformers
  (see \url{http://stackoverflow.com/a/2760709}).
}

\citet{cartwright1994}, \citet{kiselyov2013} and \citet{kiselyov2015}
offer a solution to the problem of modularity, in the form of
\emph{extensible effects}. An in-depth discussion of extensible
effects is beyond the scope of this thesis, so we will simply give an
outline of the interface presented by the \citeyear{kiselyov2015}
implementation of extensible effects.

In short, this implementation a type constructor $\mathbb{E}_X$ which
is indexed by a ``set'' of effects. This constructor forms a monad for
arbitrary sets $X$, so we can keep using the style of lexical
definitions we saw above.
What sets extensible effects apart from monad transformers is that
with extensible effects, instead of defining a transformer and a
lifting operator, the programmer defines an `effect' in isolation from
every other effect. An effect is defined by three things:
\begin{enumerate*}[label=\arabic*)]
\item a type constructor, which links the type of effectful values to
  the type of the effect;
\item primitive effectful functions, such as `tell';
\item a handler, which removes the effect from the set of effects,
  optionally consuming or producing additional input or output.
\end{enumerate*}
In the case of expressive content, the effect is
defined as follows:\footnote{%
  The usage of $\top$ in the definition of Exp means that Exp is
  \emph{only} defined for the $\top$ type---this, in turn, forces the
  output type of `tell' to be $\top$.
}
\[
  \text{Exp}\;\top=\t
\]
Once we have this definition, we can recover the `tell' function using
one of the primitives offered by extensible effects---the `send'
function. In fact, `tell' is exactly the `send' function, with `$F$'
instantiated to the expressive effect:
\[
  \text{send} : FA\ra\mathbb{E}_{\{F\}\cup X}A
  \quad
  \textit{specialises to}
  \quad
  \text{tell} : \t\ra\mathbb{E}_{\{Exp\}\cup X}\top
\]
What we have gained at this point is the parameter $X$---the `tell'
function is now polymorphic in the set of effects. This means that a
word with only expressive content---such as `damn'---only has access
to the effects associated with expressive content, and not---as was
the case with monad transformers---to the entire stack of effects.
Last, we have to define a handler for the effect. This means defining
a function which takes a value which includes the effect, and returns
a value without it. For expressives, our handler will have the
following type:
\[
  \text{run}_{\text{Exp}} : \mathbb{E}_{\{Exp\}\cup X}A\ra\mathbb{E}_{X}(A\times\t)
\]
This handler will remove the expressive effect, and tuples the
expressive content with the at-issue content. In general, handlers
allow us to remove effects from the set of effects step-by-step until
we once again end up with an effect-free value.


\subsection{Quantifier Raising and Structural Ambiguity}

The framework of monadic semantics and extensible effects, as
described in the previous section is extremely powerful. As mentioned,
it provides a flexible and powerful framework for a large spectrum of
complex natural language phenomena such as context, intensionality,
expressives and quantification. In this section, we will discuss some
of the limitations of monadic semantics---specifically, those relating
to quantification.

\vspace*{1\baselineskip}

\citet{barker2002,barker2004} advocates the use of continuations in
natural language semantics. He does this with several case studies,
one of which is quantification. The gist of this is as follows:

For the utterance ``John saw everyone,'' the function-argument
structure is `(sees everyone) john', but we associate it with
the meaning $\forall x.\PERSON(x)\supset\SEE(\JOHN,x)$. This means
that the expression `everyone,' which is deeply nested in the parse
tree, somehow takes scope over the entire expression.
\citeauthor{barker2004} proposes to solve this problem by applying a
(nonstandard) continuation-passing style (CPS) translation to the
lexicon, lifting expressions of type $A$ into the type $(A\ra R)\ra R$
for some answer type $R$:
\begin{alignat*}{3}
  &\text{john}     &&= \lambda k. k\;\JOHN\\
  &\text{sees}     &&= \lambda k. k\;\SEE
  \intertext{%
    He then uses this freedom to define `everyone' as a lifted expression
    of type `\e', with $R$ instantiated to `\t':
  }
  &\text{everyone} &&= \lambda k. \forall x.\PERSON(x)\supset k\;x
\end{alignat*}
Lastly, he defines a translation on terms, replacing function
application with a method of combining CPS-translated
terms:\footnote{%
  It should be noted that \citeauthor{barker2004}'s initial solution
  uses some directional information to ensure that scope-takers are
  always processed in a left-to-right order, instead of always
  processing the argument first, as it is presented here.
  This distinction, however, becomes irrelevant once he introduces the
  ambiguous translation, and therefore I have chosen not to include it.
}
\[
  \overline{M\;N}= \lambda k. \overline{N}\;(\lambda
  n.\overline{M}\;(\lambda m.k\;(m\;n)))
\]
As CPS-translations and monadic translations are closely related
\citep{filinski1994}, it is rather easy to implement this approach as
a monad or an effect---in fact, \citet{shan2002} already provides such
an implementation.

However, the analysis of quantification as a side-effect has several
major issues. The first of these is already mentioned by
\citet{barker2004}: scope ambiguity. Sentences containing two or
more quantifiers, such as ``Everyone likes someone,'' are ambiguous:
\begin{alignat*}{3}
  &\forall x.\PERSON(x) \,&\supset \,&\exists y.\PERSON(y) \,&\wedge  \,&\LIKE(x,y)\\
  &\exists y.\PERSON(y) \,&\wedge  \,&\forall x.\PERSON(x) \,&\supset \,&\LIKE(x,y)
\end{alignat*}
The solution provided above will only derive the first of these
meanings. Barker solves this problem by adding another possible
translation for function application, thereby making the
CPS-translation ambiguous:
\begin{alignat*}{3}
  &\overline{M\;N} &&= \lambda k. \overline{N}\;(\lambda
  n.\overline{M}\;(\lambda m.k\;(m\;n)))\\
  &\overline{M\;N} &&= \lambda k. \overline{M}\;(\lambda
  m.\overline{N}\;(\lambda n.k\;(m\;n)))
\end{alignat*}
There are several problems with this solution. First of all, this
results in a huge amount of spurious ambiguity. A sentence with $n$
words has $n-1$ function applications, and will therefore have
$2^{(n-1)}$ ``different'' interpretations. However, the ambiguity is
only relevant for scope-takers. In the motivating example, ``likes''
is not a scope taker. Nevertheless, there are two possible
translations for ``likes someone'', resulting in \emph{four}
interpretations for the sentence where we only want two.

Secondly, using an ambiguous translation clearly makes our
CPS-translation incompatible with our monadic translation, unless we
are willing to ambiguously translate all our effects. But doing so is
risky, as a right-to-left interpretation may not be desirable for
effects other than quantification. For instance, a right-to-left
interpretation of the state monad for anaphora resolution will allow
sentences such as ``He$_i$ gave a book to John$_i$.''

\vspace*{1\baselineskip}

In more recent work, \citet{moortgat2012} integrate CPS-semantics into
the syntactic calculus. The result is the Lambek-Grishin (LG)
calculus, a classical variant of NL, with semantics inspired on
\citeauthor{parigot1992}'s $\lambda\mu$-calculus
(\citeyear{parigot1992}). LG is a beautifully symmetric calculus, and
display NL, as used in this thesis is, is directly based from it. The
integrated CPS-semantics give some interesting opportunities, for
instance the ability to give quantifiers such as `everyone' the type
$(NP\oslash N)\otimes N$---a pair of a determiner and a noun---and
give them the desired scope-taking semantics.

\vspace*{1\baselineskip}

\citeauthor{barker2004}'s CPS-translation and
\citeauthor{moortgat2012}'s CPS-semantics both suffer from two related
problems:
\begin{enumerate}
\item Both require a \emph{static} choice of answer type. This
  means that the answer type cannot change throughout the
  program. However, there are examples which demonstrate the need for
  allowing dynamic answer types.
\item Both cannot encode \emph{delimiters} past which a nested
  expression cannot take scope. However, in natural language we find
  evidence for scope islands\footnote{%
    We will discuss scope islands in
    \autoref{sec:movement-and-quantifier-raising}.
  }, from which quantifiers cannot escape.
\end{enumerate}
These are the hallmark of \emph{delimited} or \emph{composable}
continuations \citep{danvy1990}. However, while delimited
continuations seem extremely promising, they are still not entirely
without problems. They still suffer from the problem of ambiguity, as
described above for CPS- and monadic translations. Additionally, they
do not form a monad. Instead, they form something known as an indexed
monad, which has two additional type-level parameters, meaning $\eta$
and $\star$ get the following types:
\[
  \eta : A\ra\mathbb{M}\;i\;i\;A
  \qquad
  \star: (A\ra\mathbb{M}\;j\;k\;B)\ra\mathbb{M}\;i\;j\;A\ra\mathbb{M}\;i\;k\;B
\]
This makes sense: since we now allow the answer type of the
continuation to change, we need to add indices to keep track of the
input and output answer type. However, the downside of this is that
since we need these additional parameters, we cannot simply
CPS-translate to delimited continuations---if we use a delimited
continuation indexed monad in our semantics, this will have to be
reflected in our syntactic calculus.

While such a solution using delimited continuations may still be
interesting to examine, we will choose to keep our solution entirely
syntactic. In the next section we will present quantifier movement and
scope islands in the syntactic calculus, based on work by
\citet{moortgat1996} and \citet{barker2015}.
