\section{Introduction}
\label{sec:introduction}

\subsection{What is type-logical grammar?}
\label{sec:what-is-type-logical-grammar}

Before we address the question of what type-logical grammar is, let us
look at the abstract pipeline for natural language understanding (NLU)
in~\autoref{fig:abstract-nlu-pipeline}.

\begin{figure}[hb]
  \begin{mdframed}
    \centering
    \begin{minipage}{0.35\linewidth}%
      \centering
      \framebox{Morphological}\\
      $\downarrow$\\
      \framebox{Lexical}\\
      $\downarrow$\\
      \framebox{Syntactic}\\
      $\downarrow$\\
      \framebox{Semantic}\\
      $\downarrow$\\
      \framebox{Pragmatic}\\
    \end{minipage}
    \begin{minipage}{0.55\linewidth}%
      \centering
      ``Mary saw foxes.''\\
      $\downarrow$\\
      Mary see.PAST fox.PL\\
      $\downarrow$\\
      Mary:NP see:TV.PAST fox:NP.PL\\
      $\downarrow$\\
      Mary:NP [see:TV.PAST fox:NP.PL]\\
      $\downarrow$\\
      $\exists X. X \subseteq\mathbf{fox}\land\mathbf{past}(\mathbf{see}(\text{Mary},X))$\\
      $\downarrow$\\
      \ldots\footnotemark\\
    \end{minipage}
  \end{mdframed}
  \caption{An abstract pipeline for natural language
    understanding.}%
  \label{fig:abstract-nlu-pipeline}
\end{figure}%
\footnotetext{%
  As the pragmatic function is all about integrating context (be it
  textual or environmental) into the meaning, it would not make sense
  to list something here.
}


To the left of the figure, you see the various phases or functions
commonly associated with an NLU-pipeline. To the right, you see the
inputs and outputs of these functions.
For instance, the morphological function will take an unanalysed
sentence, and return a sentence which is lemmatised. This entails that
all morphemes are made explicit---for instance, in the case of the
example in \autoref{fig:abstract-nlu-pipeline}, the previously
``implicit'' morphemes for past tense and plurality are added.

There is some disagreement on the exact role of type-logical grammars in
this pipeline. Ideally, type-logical grammars would play the role of
both the syntactic and the semantic function. However, the current
state of affairs in research is that often only the semantic function
is truly considered\footnote{%
  This statement is not true for \emph{associative} type-logical
  grammars, which fundamentally treat the syntactic and semantic
  functions as the same function by rejecting the tree structure of
  sentences.
}. This makes sense from a research perspective: we can refer to the
huge body of work on generative grammar to inform our choice for
sentence structure, and focus on assigning these structures the right
meaning. This is also the approach we will also take in this
thesis---that is, we consider type-logical grammars to be the function:
\begin{center}
  Mary:NP [see:TV.PAST fox:NP.PL]\\
  $\downarrow$\\
  \framebox{Type-Logical Grammar}\\
  $\downarrow$\\
  $\exists X.X\subseteq\mathbf{fox}\land\mathbf{past}(\mathbf{see}(\text{mary},X))$
\end{center}
The problem that the semantic function is trying to address is the
following:
\begin{center}\itshape
  Given a tree of typed terms, return the meaning associated with that tree.
\end{center}
Given the presence of the phrase ``tree of typed terms'', we may
already suspect that type theory offers a fitting solution to this
problem. And indeed, under the guise of type-logical grammar, it
does. A type-logical grammar generally consists of three things:
\begin{enumerate}[label=(\arabic*)]
\item a syntactic calculus, set up in such a fashion that only
  grammatical sentences are well-typed, and for which an efficient
  method of proof-search exists;
\item a semantic calculus, used to represent the meanings of words and
  sentences; and
\item a translation from the syntactic to the semantic calculus.
\end{enumerate}
We interpret the part-of-speech tags in our input (NP, TV, etc.) as
types in the syntactic calculus, and combine these with the desired
type for the tree---usually \S\ for `sentence'---to form an input
sequent.
We then search for a proof of that sequent in the syntactic calculus,
and translate it to a term in the semantic calculus.
Once there, we interpret the morphemes (e.g.\ lemmas, \texttt{PAST}
and \texttt{PL}) as possibly complex terms in the semantic calculus.

In \autoref{sec:simple-type-logical-grammar}, we will have a look at
an example type-logical grammar, and give some examples of this
process of computing sentence meaning.



\subsection{A simple type-logical grammar}
\label{sec:simple-type-logical-grammar}

The simplest type-logical grammar that comes to mind---drawing
heavily from Montague grammar and categorial grammar---is composed of
the simply-typed lambda calculus with atomic types \e\ and \t\ (\lamET) as
a semantic calculus, and the non-associative Lambek calculus
\citep[NL;][]{lambek1961} as a syntactic calculus.

The usual natural deduction formulation of \lamET\ can to be seen
in~\autoref{fig:implicit-semantic-calculus}. It is a simple lambda
calculus, with atomic types \e\ (`entity') and \t\ (`truth-value'). In
addition, all logical operators and word-meanings we need are assumed
to be defined as constants, with their expected type. For instance,
$\forall$ is a constant of type $(\e\ra\t)\ra\t$, and \john\ is a
constant of type \e. Note that we will often write logical operators
in their usual notation, e.g.\ $M\land N$ or $\forall x.M$, but this
should be taken as syntactic sugar for constants, abstractions and
applications.

\begin{figure}[hb]
  \begin{mdframed}
    \centering
    \begin{alignat*}{4}
      \text{Atom}       \;&α  &&\coloneqq \e\vsep\t\\
      \text{Type}       \;&A,B&&\coloneqq α\vsep A\ra B\\
      \text{Term}       \;&M,N&&\coloneqq x\vsep C\vsep\lambda x.M\vsep(M\;N)\\
      \text{Constant}   \;&C  &&\coloneqq
      {\forall}\vsep{\exists}\vsep{\neg}\vsep{\supset}\vsep{\land}\vsep{\lor}\vsep\ldots\\
      \text{Environment}\;&Γ  &&\text{set of typing assumptions of the form `$M : A$'}
    \end{alignat*}

    \begin{pfbox}
      \AXC{$(x : A)\in Γ$} \RightLabel{{Ax}} \UIC{$Γ\fCenter x : A$}
    \end{pfbox}
    \\[1\baselineskip]
    \begin{pfbox}
      \AXC{$Γ,x : A\fCenter M : B$} \RightLabel{$\ra${I}}
      \UIC{$Γ\fCenter \lambda x.M : A\ra B$}
    \end{pfbox}
    \begin{pfbox}
      \AXC{$Γ\fCenter M : A\ra B$} \AXC{$Γ\fCenter N : A$}
      \RightLabel{$\ra${E}} \BIC{$Γ\fCenter (M\;N) : B$}
    \end{pfbox}

    \vspace*{\baselineskip}
  \end{mdframed}
  \caption{\lamET, a simple semantic calculus.}%
  \label{fig:implicit-semantic-calculus}
\end{figure}
%
%%% Local Variables:
%%% mode: latex
%%% TeX-master: t
%%% End:
%

Using this calculus as a semantics function directly would
over-generate as, e.g.\ for the sequent $\{\john:\e,
\likes:\e\ra\e\ra\t, \mary:\e\}\fCenter\t$ we can derive
both $((\likes\;\john)\;\mary)$ and $((\likes\;\mary)\;\john)$.
The reason for this is, of course, that the set structure used in this
formulation is much too expressive for natural language grammar.

If we want more control over the structure of our terms, a good first
step is to move to a purely syntactic formulation, where all the
structural properties are made explicit in the calculus itself; this
has been done in \autoref{fig:explicit-semantic-calculus}. We have
replaced the set by a (possibly empty) binary tree, spanned by the
structural product `$\prod$'. We have also included a number of new
structural rules, which implement the structure of a set: $\emptyset$E
allows us to have an empty antecedent; contraction and weakening tell
us that we can use formulas multiple times, or not at all; and with
commutativity and associativity we can change the order of the
formulas any way we like.

\begin{figure}[hb]
  \begin{mdframed}
    \centering
    \begin{minipage}{0.6\linewidth}
      \begin{alignat*}{4}
        \text{Atom}     \;&α    &&\coloneqq \e\vsep\t\\
        \text{Type}     \;&A,B  &&\coloneqq α\vsep A\ra B\\
        \text{Structure}\;&Γ,Δ,Π&&\coloneqq A\vsep\emptyset\vsep Γ\prod Δ\\
        \text{Context}  \;&Σ    &&\coloneqq \Box\vsep Σ\prodl Δ\vsep Γ\prodr Σ
      \end{alignat*}
    \end{minipage}%
    \begin{minipage}{0.4\linewidth}
      \begin{align*}
        \Box [Γ]&\mapsto Γ\\
        (Σ\prodl Δ)[Γ]&\mapsto (Σ[Γ]\prod Δ)\\
        (Δ\prodr Σ)[Γ]&\mapsto (Δ\prod Σ[Γ])
      \end{align*}
    \end{minipage}

    \vspace*{\baselineskip}
    \begin{pfbox}[0.9]
      \AXC{}\RightLabel{{Ax}}\UIC{$A\fCenter A$}
    \end{pfbox}

    \vspace*{\baselineskip}
    \begin{pfbox}[0.9]
      \AXC{$Γ\prod A\fCenter B$}
      \RightLabel{$\ra${I}}
      \UIC{$Γ\fCenter A\ra B$}
    \end{pfbox}
    \begin{pfbox}[0.9]
      \AXC{$Γ\fCenter A\ra B$}
      \AXC{$Δ\fCenter A$}
      \RightLabel{$\ra${E}}
      \BIC{$Γ\prod Δ\fCenter B$}
    \end{pfbox}

    \vspace*{\baselineskip}
    \begin{pfbox}[0.9]
      \AXC{$Σ[Γ\prod \emptyset]\fCenter B$}
      \RightLabel{$\emptyset${E}}
      \UIC{$Σ[Γ]\fCenter B$}
    \end{pfbox}

    \vspace*{\baselineskip}
    \begin{pfbox}[0.9]
      \AXC{$Σ[A\prod A]\fCenter B$}
      \RightLabel{Cont.}
      \UIC{$Σ[A]\fCenter B$}
    \end{pfbox}
    \begin{pfbox}[0.9]
      \AXC{$Σ[Γ]\fCenter B$}
      \RightLabel{Weak.}
      \UIC{$Σ[Γ\prod Δ]\fCenter B$}
    \end{pfbox}

    \vspace*{\baselineskip}
    \begin{pfbox}[0.9]
      \AXC{$Σ[Δ\prod Γ]\fCenter B$}
      \RightLabel{Comm.}
      \UIC{$Σ[Γ\prod Δ]\fCenter B$}
    \end{pfbox}
    \begin{pfbox}[0.9]
      \AXC{$Σ[(Γ\prod Δ)\prod Π]\fCenter B$}
      \doubleLine\RightLabel{Ass.}
      \UIC{$Σ[Γ\prod (Δ\prod Π)]\fCenter B$}
    \end{pfbox}
    \vspace*{\baselineskip}
  \end{mdframed}
  \caption{Version of \lamET\ with explicit structural rules.}%
  \label{fig:explicit-semantic-calculus}
\end{figure}
%
%%% Local Variables:
%%% mode: latex
%%% TeX-master: t
%%% End:


Note that, in order to define these structural rules, we had to define
the notion of a `context'---a structure with \emph{exactly one} hole
in it---and a plugging function `\plug'---a function which inserts a
structure into that hole. The reason for this is that we have to be
able to apply commutativity and associativity \emph{anywhere} in the
structure to be able to freely change the order (and
bracketing).\footnote{%
  The contexts are not strictly necessary for $\emptyset$E,
  contraction and weakening, since we can already move any formula
  anywhere we want, but they make the proof system much more usable
  and greatly decrease the length of proofs that need to use any of
  these structural rules.
}

It is not hard to convince yourself that the implicit and explicit
versions of \lamET\ are equivalent---though we will refrain from
giving the full proof here.
Because of this equivalence, we can use the term language from
\autoref{fig:implicit-semantic-calculus} for the explicit version of
\lamET.
The term labelling of the logical rules is exactly the same. The
structural rules only manipulate structures, and therefore do not
change the terms. The only exception to this is contraction, for which
the term labelling is as follows:
\begin{prooftree}
  \AXC{$Σ[y : A\prod z : A]\fCenter M : B$}
  \RightLabel{Cont.}
  \UIC{$Σ[x : A]\fCenter M[x/y][x/z] : B$}
\end{prooftree}
Contraction takes a term with two variables of the same type, and
contracts them using substitution, which is defined as usual:
\begin{alignat*}{3}
  &x             &&[N/y] \mapsto
  \begin{cases}
    N, &\text{if}\;x=y\\
    x, &\text{otherwise}
  \end{cases}
  \\
  &C             &&[N/y] \mapsto C\\
  &(\lambda x.M) &&[N/y] \mapsto
  \begin{cases}
    \lambda x.M[N/y], &\text{if}\;x=y\\
    \lambda x.M,      &\text{otherwise}
  \end{cases}
  \\
  &(M\;M')       &&[N/y] \mapsto (M[N/y]\;M'[N/y])
\end{alignat*}

Using our explicit semantic calculus, we can construct our syntactic
calculus in three simple steps:
\begin{enumerate}
\item%
  we drop \emph{all} structural rules;
\item%
  since the implication `$\ra$' can now only take arguments directly
  from the left, we add a second implication `$\la$' which can only
  take arguments from the right---by convention, implications in this
  system are written as `$\impr$' and `$\impl$' (pronounced ``under''
  and ``over'') with the arguments \emph{under} the slash;
\item%
  we replace the atomic semantic types \e\ and \t\ by atomic syntactic
  types, reminiscent of part-of-speech tags---in this case, these
  are S (`sentence'), NP (`noun phrase'), N (`noun'), PP
  (`prepositional phrase') and INF (`infinitive').
\end{enumerate}
The resulting system can be seen in \autoref{fig:syntactic-calculus},
defined along with some definitions for common part-of-speech tags \A\
(`adjective'), \IV\ (`intransitive verb') and \TV\ (`transitive verb').

\begin{figure}
  \begin{mdframed}
    \centering
    \begin{minipage}{0.66\linewidth}
      \begin{alignat*}{4}
        \text{Atom}     \;&α  &&\coloneqq \S\vsep\N\vsep\NP\vsep\PP\vsep\INF\\
        \text{Type}     \;&A,B&&\coloneqq α\vsep A\impr B\vsep B\impl A\\
        \text{Structure}\;&Γ,Δ&&\coloneqq A\vsep Γ\prod Δ
      \end{alignat*}
    \end{minipage}%
    \begin{minipage}{0.33\linewidth}
      \begin{alignat*}{3}
        &\A \;&&\coloneqq \N\impl\N\\
        &\IV\;&&\coloneqq \NP\impr\S\\
        &\TV\;&&\coloneqq \IV\impl\NP
      \end{alignat*}
    \end{minipage}
    \\[1\baselineskip]
    \begin{pfbox}
      \AXC{} \RightLabel{{Ax}} \UIC{$A\fCenter A$}
    \end{pfbox}
    \\[1\baselineskip]
    \begin{pfbox}
      \AXC{$A\prodΓ\fCenter B$} \RightLabel{$\impr${I}}
      \UIC{$Γ\fCenter A\impr B$}
    \end{pfbox}
    \begin{pfbox}
      \AXC{$Γ\fCenter A$} \AXC{$Δ\fCenter A\impr B$} \RightLabel{$\impr${E}}
      \BIC{$Γ\prodΔ\fCenter B$}
    \end{pfbox}
    \\[1\baselineskip]
    \begin{pfbox}
      \AXC{$Γ\prod A\fCenter B$} \RightLabel{$\impl${I}}
      \UIC{$Γ\fCenter B\impl A$}
    \end{pfbox}
    \begin{pfbox}
      \AXC{$Γ\fCenter B\impl A$} \AXC{$Δ\fCenter A$} \RightLabel{$\impl${E}}
      \BIC{$Γ\prod Δ\fCenter B$}
    \end{pfbox}
    \vspace*{1\baselineskip}
  \end{mdframed}
  \caption{NL, a simple syntactic calculus.}%
  \label{fig:syntactic-calculus}
\end{figure}
%
\begin{figure}
  \begin{mdframed}
    \begin{minipage}{0.333\linewidth}
      \begin{alignat*}{4}
        &\tr[\S]         &&\mapsto\t              \\
        &\tr[\N]         &&\mapsto\e\ra\t         \\
        &\tr[\NP]        &&\mapsto\e              \\
        &\tr[\INF]       &&\mapsto\e\ra\t         \\
        &\tr[\PP]        &&\mapsto\e              \\
        &\tr[(A\impr B)] &&\mapsto\tr[A]\ra\tr[B] \\
        &\tr[(B\impl A)] &&\mapsto\tr[A]\ra\tr[B]
      \end{alignat*}
    \end{minipage}%
    \begin{minipage}{0.666\linewidth}
      \vspace*{1\baselineskip}
      \begin{prooftree}
        \AXC{}\RightLabel{Ax}\UIC{$x:A\fCenter x:A$}
      \end{prooftree}
      \begin{prooftree}
        \AXC{$x:A\prod Γ\fCenter M:B$}
        \RightLabel{$\impr$I}
        \UIC{$Γ\fCenter \lambda x.M:A\impr B$}
      \end{prooftree}
      \begin{prooftree}
        \AXC{$Γ\fCenter N:A$}
        \AXC{$Δ\fCenter M:A\impr B$}
        \RightLabel{$\impr$E}
        \BIC{$Γ\prod Δ\fCenter (M\;N):B$}
      \end{prooftree}
      \begin{prooftree}
        \AXC{$Γ\prod x:A\fCenter M:B$}
        \RightLabel{$\impl$I}
        \UIC{$Γ\fCenter \lambda x.M:B\impl A$}
      \end{prooftree}
      \begin{prooftree}
        \AXC{$Γ\fCenter M:B\impl A$}
        \AXC{$Δ\fCenter N:A$}
        \RightLabel{$\impl${E}}
        \BIC{$Γ\prod Δ\fCenter (M\;N):B$}
      \end{prooftree}
    \end{minipage}
    \vspace*{1\baselineskip}
  \end{mdframed}
  \caption{A translation from NL (\autoref{fig:syntactic-calculus}) to \lamET\ (\autoref{fig:explicit-semantic-calculus}).}%
  \label{fig:syntactic-calculus-to-explicit-lamET}
\end{figure}
%
%%% Local Variables:
%%% mode: latex
%%% TeX-master: t
%%% End:


Dropping \emph{all} structural rules may seem unnecessary, but there
is a good motivation for each rule.  For example, in the presence of
commutativity, there is no way to distinguish between ``Mary walks.''\ %
and ``walks Mary.''; under weakening, we can add any word anywhere in a
grammatical sentence, and the sentence will remain grammatical---
e.g.\ ``Mary banana walks.''; and with contraction, we can remove
consecutive words with the same type---this means that ``John read
a fantastic blue book.'' could be taken to mean the same thing as
``John read a blue book.''

The motivations to drop associativity and the empty structure are a
little bit harder to understand, but the interested reader can refer
to \citet[p. 33, 105-106]{moot2012} or \citet{lambek1961}, who show
some ungrammatical sentences that would be accepted in the presence of
these rules.

\vspace*{1\baselineskip}

The last component we need for our simple type-logical grammar is a
translation from our syntactic calculus to our semantic calculus. This
translation is described in
\autoref{fig:syntactic-calculus-to-explicit-lamET} as:
\begin{enumerate}[label=(\arabic*)]
\item
  a function $\tr$, translating the types in NL to types in \lamET; and
\item
  a set of rewrite rules, that rewrite proofs in NL to proofs in \lamET.
\end{enumerate}
With these two translation functions, we can set up a new presentation
for our syntactic proofs, namely one where the syntactic proofs are
labelled with the semantic terms resulting from the translation.

As an example, let's give an analysis of the sentence ``Mary likes
Bill.'' We assume the morphological, lexical and syntactic phases
have been taken care of. This turns our input sequent into:
\[
  \mary:\NP\prod(\likes:\TV\prod\bill:\NP)\;\fCenter\;?:\S
\]
Fortunately, proof search is decidable for this system, so we can
simply search the space of all possible proofs of this sequent. It
turns out that the only proof is:
\begin{center}
  \vspace*{-1\baselineskip}
  \hspace*{-2em}%
  \begin{pfbox}
    \AXC{}\RightLabel{Ax}\UIC{$\mary:\NP\fCenter\mary:\NP$}
    \AXC{}\RightLabel{Ax}\UIC{$\likes:\TV\fCenter\likes:(\NP\impr\S)\impl\NP$}
    \AXC{}\RightLabel{Ax}\UIC{$\bill:\NP\fCenter\bill:\NP$}
    \RightLabel{$\impl$E}
    \BIC{$\likes:\TV\prod\bill:\NP\fCenter(\likes\;\bill):\NP\impr\S$}
    \RightLabel{$\impr$E}
    \BIC{$\mary:\NP\prod(\likes:\TV\prod\bill:\NP)\fCenter((\likes\;\bill)\;\mary):\S$}
  \end{pfbox}
\end{center}
And so, by searching for a proof in our syntactic calculus, and
translating it to our semantic calculus, we get a meaning
representation for our input sequent.

%%% Local Variables:
%%% mode: latex
%%% TeX-master: t
%%% End:
