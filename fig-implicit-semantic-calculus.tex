\begin{figure}
  \begin{mdframed}
    \centering
    \begin{alignat*}{4}
      \text{Type}    \;&A,B&&\coloneqq \e\vsep\t\vsep A\ra B\\
      \text{Term}    \;&M,N&&\coloneqq x\vsep C\vsep\lambda x.M\vsep(M\;N)\\
      \text{Constant}\;&C &&\coloneqq
      {\forall}\vsep{\exists}\vsep{\neg}\vsep{\supset}\vsep{\land}\vsep{\lor}\vsep\ldots\footnotemark
    \end{alignat*}

    \begin{pfbox}
      \AXC{$(x : A)\in Γ$} \RightLabel{{Ax}} \UIC{$Γ\fCenter x : A$}
    \end{pfbox}

    \vspace*{\baselineskip}
    \begin{pfbox}
      \AXC{$Γ,x : A\fCenter M : B$} \RightLabel{$\ra${I}}
      \UIC{$Γ\fCenter \lambda x.M : A\ra B$}
    \end{pfbox}
    \begin{pfbox}
      \AXC{$Γ\fCenter M : A\ra B$} \AXC{$Γ\fCenter N : A$}
      \RightLabel{$\ra${E}} \BIC{$Γ\fCenter (M\;N) : B$}
    \end{pfbox}

    \vspace*{\baselineskip}
  \end{mdframed}
  \caption{A simple semantic calculus.}%
  \label{fig:implicit-semantic-calculus}
\end{figure}
%
\footnotetext{%
  We will assume any logical operator we need is defined as a
  constant in our calculus, with the expected type. We will write
  these in their usual form -- e.g. $M\land N$ or $\forall x.M$ -- but
  this should be taken as syntactic sugar for the logical constants,
  abstractions and applications.
}%
%%% Local Variables:
%%% mode: latex
%%% TeX-master: t
%%% End:
