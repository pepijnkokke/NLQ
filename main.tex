\documentclass[a4paper]{article}

\usepackage{cmap}%
\usepackage[LGR,T1]{fontenc}%
\usepackage[utf8]{inputenc}%
\usepackage{alphabeta}%
\usepackage[greek,english]{babel}%
\languageattribute{greek}{polutoniko}%
\usepackage{hyperref}%
\addto\extrasenglish{\def\figureautorefname{figure}}%
\usepackage{lmodern}%

\usepackage{xcolor}
\usepackage{mathtools}
\usepackage{mdframed}
\usepackage[inline]{enumitem}
\usepackage{natbib}
\usepackage{pdflscape}
\usepackage{stmaryrd}
\usepackage{comment}
\usepackage{ifthen}
\usepackage{tikz}
\usepackage{tikz-qtree}
\usepackage{textcomp}

\usepackage{amsmath, amscd, amsthm, amssymb, mathrsfs, amsfonts, wasysym}%
\let\oldemptyset\emptyset%
\let\emptyset\varnothing%
\usepackage{cancel}%

\usepackage{tabularx}%
\usepackage{multirow}%
\renewcommand{\arraystretch}{3}%

\usepackage{pict2e}
\usepackage{picture}

\newcommand{\varslash}{%
  \mathbin{\mathpalette\pictslash{{0}{1}}}%
}
\newcommand{\varbslash}{%
  \mathbin{\mathpalette\pictslash{{1}{-1}}}%
}

\makeatletter
\newcommand{\pictslash}[2]{%
  \vcenter{%
    \sbox0{$\m@th#1\varobslash$}\dimen0=.55\wd0
    \hbox to\wd 0{%
      \hfil\pictslash@aux#2\hfil
    }%
  }%
}
\newcommand{\pictslash@aux}[2]{%
    \begin{picture}(\dimen0,\dimen0)
    \roundcap
    \linethickness{.15ex}
    \put(0,#1\dimen0){\line(1,#2){\dimen0}}
    \end{picture}%
}
\makeatother

\def\downmapsto{\rotatebox[origin=c]{270}{\ensuremath{\mapsto}}}%

\usepackage{bussproofs}%
\EnableBpAbbreviations%
\def\fCenter  {\mathbin{\vdash}}%
\def\la       {\leftarrow}%
\def\ra       {\rightarrow}%
\def\impl     {\mathbin{\slash}}%
\def\impr     {\mathbin{\backslash}}%
\def\prod     {\bullet}%
\def\prodl    {\bullet}%
\def\prodr    {\bullet}%
\def\himpl    {\!\!\fatslash\:}%
\def\himpr    {\fatbslash}%
\def\hprod    {\mathbin{\circ}}%
\def\hprodl   {\mathbin{\circ}}%
\def\hprodr   {\mathbin{\circ}}%
\def\trace    {\ensuremath{\text{trace}}}%
\def\unit     {\ensuremath{\mathbf{I}}}%
\def\sq       {\ensuremath{\Box}}%
\def\di       {\ensuremath{\Diamond}}%
\def\diIfx    {\ensuremath{\hat{\di}}}%
\def\sqIfx    {\ensuremath{\hat{\sq}}}%
\def\diExt    {\rotatebox[origin=c]{180}{\diIfx}}%
\def\sqExt    {\rotatebox[origin=c]{180}{\sqIfx}}%
\def\vsep     {\ \vert\ }%
\def\e        {\ensuremath{\mathbf{e}}}%
\def\t        {\ensuremath{\mathbf{t}}}%
\def\lamET    {\ensuremath{\lambda^{\ra}_{\{\e,\t\}}}}%
\def\S        {\text{S}}%
\def\N        {\text{N}}%
\def\NP       {\text{NP}}%
\def\PP       {\text{PP}}%
\def\INF      {\text{INF}}%
\def\A        {\text{A}}%
\def\IV       {\text{IV}}%
\def\TV       {\text{TV}}%
\def\I        {\ensuremath{\mathbf{I}}}%
\def\B        {\ensuremath{\mathbf{B}}}%
\def\C        {\ensuremath{\mathbf{C}}}%
\def\john     {\ensuremath{\text{john}}}%
\def\mary     {\ensuremath{\text{mary}}}%
\def\bill     {\ensuremath{\text{bill}}}%
\def\walks    {\ensuremath{\text{walks}}}%
\def\damned   {\ensuremath{\text{damned}}}%
\def\dog      {\ensuremath{\text{dog}}}%
\def\likes    {\ensuremath{\text{likes}}}%
\def\sees     {\ensuremath{\text{sees}}}%
\def\leave    {\ensuremath{\text{leave}}}%
\def\wants    {\ensuremath{\text{wants}}}%
\def\everyone {\ensuremath{\text{everyone}}}%
\def\different{\ensuremath{\text{different}}}%
\def\someone  {\ensuremath{\text{someone}}}%
\def\served   {\ensuremath{\text{served}}}%
\def\a        {\ensuremath{\text{a}}}%
\def\waiter   {\ensuremath{\text{waiter}}}%
\def\JOHN     {\ensuremath{\mathbf{john}}}%
\def\PERSON   {\ensuremath{\mathbf{person}}}%
\def\WAITER   {\ensuremath{\mathbf{waiter}}}%
\def\WALK     {\ensuremath{\mathbf{walk}}}%
\def\LIKE     {\ensuremath{\mathbf{like}}}%
\def\SERVE    {\ensuremath{\mathbf{serve}}}%
\def\SEE      {\ensuremath{\mathbf{see}}}%
\def\DAMN     {\ensuremath{\mathbf{damn}}}%
\def\PAST     {\ensuremath{\mathbf{past}}}%
\def\DOG      {\ensuremath{\mathbf{dog}}}%
\def\WANT     {\ensuremath{\mathbf{want}}}%
\def\plug     {\ensuremath{\:\cdot\:[\:\cdot\:]}}%
\def\qdown    {\ensuremath{\mathbf{Q}{\downarrow}}}
\def\qup      {\ensuremath{\mathbf{Q}{\uparrow}}}

\newcommand{\q}       [1][{\cdot}]{\mathbf{Q}(#1)}%
\newcommand{\focus}   [1]{\boxed{#1}}%
\newcommand{\struct}  [1]{{\cdot}#1{\cdot}}%
\newcommand{\sub}     [3]{#1[#2/#3]}%
\newcommand{\tr}      [1][({\cdot})]{#1^*}%
\newcommand{\trd}     [1][({\cdot})]{#1^{**}}%
\newcommand{\case}    [4]{\text{case}\;#1\;\text{of}\;(#2, #3)\ra#4}%
\newcommand{\add}     [2]{#1 + #2}
\newcommand{\mathplus}[0]{+}

\renewcommand*{\&}{%
  \relax
  \ifmmode
    \mathbin{\char`\&}%
  \else
    \char`\&\relax
  \fi
}

\newenvironment{pfbox}[1][0.9]%
  {\gdef\scalefactor{#1} \leavevmode\hbox\bgroup}
  {\scalebox{\scalefactor}{\DisplayProof} \egroup}

\newenvironment{pfblock}[1][0.9]%
  {\gdef\scalefactor{#1}\begin{center}\proofSkipAmount \leavevmode}%
  {\scalebox{\scalefactor}{\DisplayProof}\proofSkipAmount \end{center} }

\newboolean{notes}%
\setboolean{notes}{true}%
\providecommand{\todo}[1]{%
  \ifthenelse%
  {\boolean{notes}}%
  {\textcolor{red}{\_}\marginpar{\color{red}#1}}%
  {}}
\providecommand{\note}[1]{%
  \ifthenelse%
  {\boolean{notes}}%
  {{\color{green}\textbf{NOTE:~}#1}}%
  {}}


% rules from the Haskell code
\newcommand{\AxR}[1]{\AXC{}\RightLabel{Ax$^L$}\UIC{$#1$}}%
\newcommand{\AxL}[1]{\AXC{}\RightLabel{Ax$^R$}\UIC{$#1$}}%
\newcommand{\UnfR}[1]{\RightLabel{Foc$^R'$}\UIC{$#1$}}%
\newcommand{\UnfL}[1]{\RightLabel{Foc$^L'$}\UIC{$#1$}}%
\newcommand{\FocR}[1]{\RightLabel{Foc$^R$}\UIC{$#1$}}%
\newcommand{\FocL}[1]{\RightLabel{Foc$^L$}\UIC{$#1$}}%
\newcommand{\WithL}[2]{\RightLabel{\&L$_{#1}$}\UIC{$#2$}}%
\newcommand{\WithR}[1]{\RightLabel{\&R}\BIC{$#1$}}%
\newcommand{\ImpRL}[1]{\RightLabel{$\impr$L}\BIC{$#1$}}%
\newcommand{\ImpRR}[1]{\RightLabel{$\impr$R}\UIC{$#1$}}%
\newcommand{\ImpLL}[1]{\RightLabel{$\impl$L}\BIC{$#1$}}%
\newcommand{\ImpLR}[1]{\RightLabel{$\impl$R}\UIC{$#1$}}%
\newcommand{\Res}[3]{%
  \RightLabel{%
    \ifstrequal{1}{#1}{%
      \ifstrequal{1}{#2}{%
        Res$\impr\prod$ % Res11
      }{%
        \ifstrequal{2}{#2}{%
          Res$\prod\impr$ % Res12
        }{%
          \ifstrequal{2}{#2}{%
            Res$\impl\prod$ % Res13
          }{%
            Res$\prod\impl$ % Res14
          }
        }
      }
    }{%
      \ifstrequal{1}{#2}{%
        Res$\sq\di$ % Res21
      }{%
        Res$\di\sq$ % Res22
      }
    }
  }
  \UIC{$#3$}
}%
\newcommand{\DiaL}[1]{\RightLabel{$\di$L}\UIC{$#1$}}%
\newcommand{\DiaR}[1]{\RightLabel{$\di$R}\UIC{$#1$}}%
\newcommand{\BoxL}[1]{\RightLabel{$\sq$L}\UIC{$#1$}}%
\newcommand{\BoxR}[1]{\RightLabel{$\sq$R}\UIC{$#1$}}%
\newcommand{\IfxRR}[1]{\RightLabel{$\Ifx$RR}\UIC{$#1$}}%
\newcommand{\IfxLR}[1]{\RightLabel{$\Ifx$LR}\UIC{$#1$}}%
\newcommand{\IfxLL}[1]{\RightLabel{$\Ifx$LL}\UIC{$#1$}}%
\newcommand{\IfxRL}[1]{\RightLabel{$\Ifx$RL}\UIC{$#1$}}%
\newcommand{\ExtRR}[1]{\RightLabel{$\Ext$RR}\UIC{$#1$}}%
\newcommand{\ExtLR}[1]{\RightLabel{$\Ext$LR}\UIC{$#1$}}%
\newcommand{\ExtLL}[1]{\RightLabel{$\Ext$LL}\UIC{$#1$}}%
\newcommand{\ExtRL}[1]{\RightLabel{$\Ext$RL}\UIC{$#1$}}%
\newcommand{\UnitRL}[1]{\RightLabel{$\I$L}\UIC{$#1$}}%
\newcommand{\UnitRR}[1]{\RightLabel{$\I$R}\UIC{$#1$}}%
\newcommand{\UnitRI}[1]{\RightLabel{$\I^-$}\UIC{$#1$}}%
\newcommand{\DnB}[1]{\RightLabel{$\B$}\UIC{$#1$}}%
\newcommand{\UpB}[1]{\RightLabel{$\B'$}\UIC{$#1$}}%
\newcommand{\DnC}[1]{\RightLabel{$\C$}\UIC{$#1$}}%
\newcommand{\UpC}[1]{\RightLabel{$\C'$}\UIC{$#1$}}%

\ifdefined\usetheme\usetheme{Boadilla}\fi
\ifdefined\usecolortheme\usecolortheme{seagull}\fi
\ifdefined\beamertemplatenavigationsymbolsempty\beamertemplatenavigationsymbolsempty\fi

\newtheorem{lemma}{Lemma}

% 1. simple semantic calculus;
% 2. simple syntactic calculus;
% 3. compositionality principle;
% 4. problems with compositionality;
% 5. extended semantic calculus;
% 6. quantifier raising and scope ambiguity;
% 7. continuation monad;
% 8. delimited continuations and indexed monads;
% 9. extended syntactic calculus;

\begin{document}

%\begin{figure}
  \centering
  \begin{minipage}{0.35\linewidth}%
    \centering
    \framebox{Morphological}\\
    $\downarrow$\\
    \framebox{Lexical}\\
    $\downarrow$\\
    \framebox{Syntactic}\\
    $\downarrow$\\
    \framebox{Semantic}\\
    $\downarrow$\\
    \framebox{Pragmatic}\\
  \end{minipage}
  \begin{minipage}{0.55\linewidth}%
    \centering
    ``Mary saw foxes.''\\
    $\downarrow$\\
    Mary see.PAST fox.PL\\
    $\downarrow$\\
    Mary:NP see:TV.PAST fox:NP.PL\\
    $\downarrow$\\
    Mary:NP [see:TV.PAST fox:NP.PL]\\
    $\downarrow$\\
    $\exists X. X \subseteq\mathbf{fox}\land\mathbf{past}(\mathbf{see}(\text{Mary},X))$\\
    $\downarrow$\\
    \ldots\\
  \end{minipage}
  \caption{An abstract pipeline for natural language understanding.}%
  \label{fig:nlu-pipeline}
\end{figure}

%%% Local Variables:
%%% mode: latex
%%% TeX-master: t
%%% End:

%\begin{figure}[hb]
  \begin{mdframed}
    \centering
    \begin{alignat*}{4}
      \text{Atom}       \;&α  &&\coloneqq \e\vsep\t\\
      \text{Type}       \;&A,B&&\coloneqq α\vsep A\ra B\\
      \text{Term}       \;&M,N&&\coloneqq x\vsep C\vsep\lambda x.M\vsep(M\;N)\\
      \text{Constant}   \;&C  &&\coloneqq
      {\forall}\vsep{\exists}\vsep{\neg}\vsep{\supset}\vsep{\land}\vsep{\lor}\vsep\ldots\\
      \text{Environment}\;&Γ  &&\text{set of typing assumptions of the form `$M : A$'}
    \end{alignat*}

    \begin{pfbox}
      \AXC{$(x : A)\in Γ$} \RightLabel{{Ax}} \UIC{$Γ\fCenter x : A$}
    \end{pfbox}
    \\[1\baselineskip]
    \begin{pfbox}
      \AXC{$Γ,x : A\fCenter M : B$} \RightLabel{$\ra${I}}
      \UIC{$Γ\fCenter \lambda x.M : A\ra B$}
    \end{pfbox}
    \begin{pfbox}
      \AXC{$Γ\fCenter M : A\ra B$} \AXC{$Γ\fCenter N : A$}
      \RightLabel{$\ra${E}} \BIC{$Γ\fCenter (M\;N) : B$}
    \end{pfbox}

    \vspace*{\baselineskip}
  \end{mdframed}
  \caption{\lamET, a simple semantic calculus.}%
  \label{fig:implicit-semantic-calculus}
\end{figure}
%
%%% Local Variables:
%%% mode: latex
%%% TeX-master: t
%%% End:

%\begin{figure}[hb]
  \begin{mdframed}
    \centering
    \begin{minipage}{0.6\linewidth}
      \begin{alignat*}{4}
        \text{Atom}     \;&α    &&\coloneqq \e\vsep\t\\
        \text{Type}     \;&A,B  &&\coloneqq α\vsep A\ra B\\
        \text{Structure}\;&Γ,Δ,Π&&\coloneqq A\vsep\emptyset\vsep Γ\prod Δ\\
        \text{Context}  \;&Σ    &&\coloneqq \Box\vsep Σ\prodl Δ\vsep Γ\prodr Σ
      \end{alignat*}
    \end{minipage}%
    \begin{minipage}{0.4\linewidth}
      \begin{align*}
        \Box [Γ]&\mapsto Γ\\
        (Σ\prodl Δ)[Γ]&\mapsto (Σ[Γ]\prod Δ)\\
        (Δ\prodr Σ)[Γ]&\mapsto (Δ\prod Σ[Γ])
      \end{align*}
    \end{minipage}

    \vspace*{\baselineskip}
    \begin{pfbox}[0.9]
      \AXC{}\RightLabel{{Ax}}\UIC{$A\fCenter A$}
    \end{pfbox}

    \vspace*{\baselineskip}
    \begin{pfbox}[0.9]
      \AXC{$Γ\prod A\fCenter B$}
      \RightLabel{$\ra${I}}
      \UIC{$Γ\fCenter A\ra B$}
    \end{pfbox}
    \begin{pfbox}[0.9]
      \AXC{$Γ\fCenter A\ra B$}
      \AXC{$Δ\fCenter A$}
      \RightLabel{$\ra${E}}
      \BIC{$Γ\prod Δ\fCenter B$}
    \end{pfbox}

    \vspace*{\baselineskip}
    \begin{pfbox}[0.9]
      \AXC{$Σ[Γ\prod \emptyset]\fCenter B$}
      \RightLabel{$\emptyset${E}}
      \UIC{$Σ[Γ]\fCenter B$}
    \end{pfbox}

    \vspace*{\baselineskip}
    \begin{pfbox}[0.9]
      \AXC{$Σ[A\prod A]\fCenter B$}
      \RightLabel{Cont.}
      \UIC{$Σ[A]\fCenter B$}
    \end{pfbox}
    \begin{pfbox}[0.9]
      \AXC{$Σ[Γ]\fCenter B$}
      \RightLabel{Weak.}
      \UIC{$Σ[Γ\prod Δ]\fCenter B$}
    \end{pfbox}

    \vspace*{\baselineskip}
    \begin{pfbox}[0.9]
      \AXC{$Σ[Δ\prod Γ]\fCenter B$}
      \RightLabel{Comm.}
      \UIC{$Σ[Γ\prod Δ]\fCenter B$}
    \end{pfbox}
    \begin{pfbox}[0.9]
      \AXC{$Σ[(Γ\prod Δ)\prod Π]\fCenter B$}
      \doubleLine\RightLabel{Ass.}
      \UIC{$Σ[Γ\prod (Δ\prod Π)]\fCenter B$}
    \end{pfbox}
    \vspace*{\baselineskip}
  \end{mdframed}
  \caption{Version of \lamET\ with explicit structural rules.}%
  \label{fig:explicit-semantic-calculus}
\end{figure}
%
%%% Local Variables:
%%% mode: latex
%%% TeX-master: t
%%% End:

%\begin{figure}
  \begin{mdframed}
    \centering
    \begin{minipage}{0.66\linewidth}
      \begin{alignat*}{4}
        \text{Atom}     \;&α  &&\coloneqq \S\vsep\N\vsep\NP\vsep\PP\vsep\INF\\
        \text{Type}     \;&A,B&&\coloneqq α\vsep A\impr B\vsep B\impl A\\
        \text{Structure}\;&Γ,Δ&&\coloneqq A\vsep Γ\prod Δ
      \end{alignat*}
    \end{minipage}%
    \begin{minipage}{0.33\linewidth}
      \begin{alignat*}{3}
        &\A \;&&\coloneqq \N\impl\N\\
        &\IV\;&&\coloneqq \NP\impr\S\\
        &\TV\;&&\coloneqq \IV\impl\NP
      \end{alignat*}
    \end{minipage}
    \\[1\baselineskip]
    \begin{pfbox}
      \AXC{} \RightLabel{{Ax}} \UIC{$A\fCenter A$}
    \end{pfbox}
    \\[1\baselineskip]
    \begin{pfbox}
      \AXC{$A\prodΓ\fCenter B$} \RightLabel{$\impr${I}}
      \UIC{$Γ\fCenter A\impr B$}
    \end{pfbox}
    \begin{pfbox}
      \AXC{$Γ\fCenter A$} \AXC{$Δ\fCenter A\impr B$} \RightLabel{$\impr${E}}
      \BIC{$Γ\prodΔ\fCenter B$}
    \end{pfbox}
    \\[1\baselineskip]
    \begin{pfbox}
      \AXC{$Γ\prod A\fCenter B$} \RightLabel{$\impl${I}}
      \UIC{$Γ\fCenter B\impl A$}
    \end{pfbox}
    \begin{pfbox}
      \AXC{$Γ\fCenter B\impl A$} \AXC{$Δ\fCenter A$} \RightLabel{$\impl${E}}
      \BIC{$Γ\prod Δ\fCenter B$}
    \end{pfbox}
    \vspace*{1\baselineskip}
  \end{mdframed}
  \caption{NL, a simple syntactic calculus.}%
  \label{fig:syntactic-calculus}
\end{figure}
%
\begin{figure}
  \begin{mdframed}
    \begin{minipage}{0.333\linewidth}
      \begin{alignat*}{4}
        &\tr[\S]         &&\mapsto\t              \\
        &\tr[\N]         &&\mapsto\e\ra\t         \\
        &\tr[\NP]        &&\mapsto\e              \\
        &\tr[\INF]       &&\mapsto\e\ra\t         \\
        &\tr[\PP]        &&\mapsto\e              \\
        &\tr[(A\impr B)] &&\mapsto\tr[A]\ra\tr[B] \\
        &\tr[(B\impl A)] &&\mapsto\tr[A]\ra\tr[B]
      \end{alignat*}
    \end{minipage}%
    \begin{minipage}{0.666\linewidth}
      \vspace*{1\baselineskip}
      \begin{prooftree}
        \AXC{}\RightLabel{Ax}\UIC{$x:A\fCenter x:A$}
      \end{prooftree}
      \begin{prooftree}
        \AXC{$x:A\prod Γ\fCenter M:B$}
        \RightLabel{$\impr$I}
        \UIC{$Γ\fCenter \lambda x.M:A\impr B$}
      \end{prooftree}
      \begin{prooftree}
        \AXC{$Γ\fCenter N:A$}
        \AXC{$Δ\fCenter M:A\impr B$}
        \RightLabel{$\impr$E}
        \BIC{$Γ\prod Δ\fCenter (M\;N):B$}
      \end{prooftree}
      \begin{prooftree}
        \AXC{$Γ\prod x:A\fCenter M:B$}
        \RightLabel{$\impl$I}
        \UIC{$Γ\fCenter \lambda x.M:B\impl A$}
      \end{prooftree}
      \begin{prooftree}
        \AXC{$Γ\fCenter M:B\impl A$}
        \AXC{$Δ\fCenter N:A$}
        \RightLabel{$\impl${E}}
        \BIC{$Γ\prod Δ\fCenter (M\;N):B$}
      \end{prooftree}
    \end{minipage}
    \vspace*{1\baselineskip}
  \end{mdframed}
  \caption{A translation from NL (\autoref{fig:syntactic-calculus}) to \lamET\ (\autoref{fig:explicit-semantic-calculus}).}%
  \label{fig:syntactic-calculus-to-explicit-lamET}
\end{figure}
%
%%% Local Variables:
%%% mode: latex
%%% TeX-master: t
%%% End:

\begin{figure}
  \begin{mdframed}
    \centering
    \begin{alignat*}{4}
      \text{Atom}&\;        &α   &\coloneqq \S\vsep\N\vsep\NP\vsep\PP\vsep\INF\\
      \text{Type}&\;        &A,B &\coloneqq α\vsep A\impr B\vsep B\impl A\\
      \text{Structure}&^+\; &Γ   &\coloneqq \cdot A\cdot\vsep Γ_1\prod Γ_2\\
      \text{Structure}&^-\; &Δ   &\coloneqq \cdot A\cdot\vsep Γ\imprΔ\vsep Δ\implΓ
    \end{alignat*}

    \begin{pfbox}
      \AXC{}
      \RightLabel{Ax}
      \UIC{$\struct{α}\fCenter\struct{α}$}
    \end{pfbox}
    \\[1\baselineskip]
    \begin{pfbox}
      \AXC{$Γ\fCenter\struct{A}$}
      \AXC{$\struct{B}\fCenter Δ$}
      \RightLabel{L$\impr$}
      \BIC{$\struct{A\impr B}\fCenter Γ\impr Δ$}
    \end{pfbox}
    \begin{pfbox}
      \AXC{$Γ\fCenter\struct{A}\impr\struct{B}$}
      \RightLabel{R$\impr$}
      \UIC{$Γ\fCenter\struct{A\impr B}$}
    \end{pfbox}
    \\[1\baselineskip]
    \begin{pfbox}
      \AXC{$Γ\fCenter\struct{A}$}
      \AXC{$\struct{B}\fCenter Δ$}
      \RightLabel{L$\impl$}
      \BIC{$\struct{B\impl A}\fCenter Δ\impl Γ$}
    \end{pfbox}
    \begin{pfbox}
      \AXC{$Γ\fCenter\struct{B}\impl\struct{A}$}
      \RightLabel{R$\impl$}
      \UIC{$Γ\fCenter\struct{B\impl A}$}
    \end{pfbox}
    \\[1\baselineskip]
    \begin{pfbox}
      \AXC{$Γ_2\fCenter Γ_1\impr Δ$}
      \doubleLine\RightLabel{Res$\impr\prod$}
      \UIC{$Γ_1\prod Γ_2\fCenter Δ$}
    \end{pfbox}
    \begin{pfbox}
      \AXC{$Γ_1\fCenter Δ\impl Γ_2$}
      \doubleLine\RightLabel{Res$\impl\prod$}
      \UIC{$Γ_1\prod Γ_2\fCenter Δ$}
    \end{pfbox}
    \vspace*{1\baselineskip}
  \end{mdframed}
  \caption{
    The syntactic calculus from~\autoref{fig:syntactic-calculus} as a
    display calculus.}%
  \label{fig:display-calculus}
\end{figure}
%
\begin{figure}
  \begin{mdframed}
    \centering
    \vspace*{1\baselineskip}
    Extension of semantic calculus from
    \autoref{fig:implicit-semantic-calculus}:
    \begin{alignat*}{4}
      \text{Type}\;&A,B&&\coloneqq\ldots\vsep A\times B\vsep\top\\
      \text{Term}\;&M,N&&\coloneqq\ldots\vsep (M, N)\vsep\case{M}{x}{y}{N}\vsep()
    \end{alignat*}

    \begin{pfbox}
      \AXC{$Γ\fCenter M : A$}
      \AXC{$Γ\fCenter N : B$}
      \RightLabel{$\times$I}
      \BIC{$Γ\fCenter (M, N) : A\times B$}
    \end{pfbox}
    \\[1\baselineskip]
    \begin{pfbox}
      \AXC{$Γ\fCenter M : A\times B$}
      \AXC{$Γ, x : A, y : B\fCenter N : C$}
      \RightLabel{$\times$E}
      \BIC{$Γ\fCenter \case{M}{x}{y}{N} : C$}
    \end{pfbox}
    \\[1\baselineskip]
    \begin{pfbox}
      \AXC{}\RightLabel{$\top$}\UIC{$Γ\fCenter () : \top$}
    \end{pfbox}
    \\[1\baselineskip]
    \hrulefill
    \\[1\baselineskip]
    Extension of semantic calculus from
    \autoref{fig:explicit-semantic-calculus}:
    \\[1\baselineskip]
    \begin{pfbox}
      \AXC{$Γ\fCenter A$}
      \AXC{$Δ\fCenter B$}
      \RightLabel{$\times$I}
      \BIC{$Γ\prod Δ\fCenter A\times B$}
    \end{pfbox}
    \\[1\baselineskip]
    \begin{pfbox}
      \AXC{$Γ\fCenter A\times B$}
      \AXC{$Δ\prod A\prod B\fCenter C$}
      \RightLabel{$\times$E}
      \BIC{$Γ\prod Δ\fCenter C$}
    \end{pfbox}
    \\[1\baselineskip]
    \begin{pfbox}
      \AXC{}\RightLabel{$\top$}\UIC{$\emptyset\fCenter \top$}
    \end{pfbox}
    \vspace*{1\baselineskip}
  \end{mdframed}
  \caption{An extension of the semantic calculi from
    \autoref{fig:implicit-semantic-calculus} and
    \autoref{fig:explicit-semantic-calculus}.}
  \label{fig:extension-products}
\end{figure}
%
%%% Local Variables:
%%% mode: latex
%%% TeX-master: t
%%% End:

%
\begin{landscape}
  \begin{figure}
  \begin{mdframed}
    \renewcommand{\arraystretch}{5}%
    \begin{tabular}{l c c c}
      \multirow{6}{*}{%
      \framebox{\(\!
      \begin{aligned}
        &\textit{Structures}\\
        &\tr[(\struct{A})]   &&\mapsto \tr[A]\\
        &\tr[(Γ_1\prod Γ_2)] &&\mapsto \tr[Γ_1]\prod\tr[Γ_2]\\
        \\
        &\tr[(\struct{A})]   &&\mapsto \tr[A]\\
        &\tr[(Δ\impl Γ)]     &&\mapsto \trd[Γ]\ra\tr[Δ]\\
        &\tr[(Γ\impr Δ)]     &&\mapsto \trd[Γ]\ra\tr[Δ]\\
        \\
        &\trd[(\struct{A})]   &&\mapsto \tr[A]\\
        &\trd[(Γ_1\prod Γ_2)] &&\mapsto \trd[Γ_1]\times\trd[Γ_2]\\
        \\
        &\textit{Sequents}\\
        &\tr[(Γ\fCenterΔ)]         &&\mapsto \tr[Γ]\fCenter\tr[Δ]
      \end{aligned}\)}}&
      \begin{pfbox}[0.9]
        \AXC{$Γ\fCenter\struct{A}$}
        \AXC{$\struct{B}\fCenter Δ$}
        \RightLabel{L$\impr$}
        \BIC{$\struct{A\impr B}\fCenter Γ\impr Δ$}
      \end{pfbox}
      &$\Longrightarrow$&
      \begin{pfbox}[0.9]
        \AXC{$\tr[B]\fCenter\tr[Δ]$}
        %\RightLabel{Weak.}
        %\UIC{$\tr[B]\prod\emptyset\fCenter\tr[B]\ra\tr[Δ]$}
        %\RightLabel{Comm.}
        %\UIC{$\emptyset\prod\tr[B]\fCenter\tr[Δ]$}
        \RightLabel{$\ra$I}
        \UIC{$\emptyset\fCenter\tr[B]\ra\tr[Δ]$}
        \AXC{}\RightLabel{Ax}\UIC{$\tr[A]\ra\tr[B]\fCenter\tr[A]\ra\tr[B]$}
        \AXC{$\tr[Γ]\fCenter\tr[A]$}
        \RightLabel{$\ra$E}
        \BIC{$\tr[A]\ra\tr[B]\prod\tr[Γ]\fCenter\tr[B]$}
        \RightLabel{$\ra$E}
        %\BIC{$\emptyset\prod(\tr[A]\ra\tr[B]\prod\tr[Γ])\fCenter\tr[Δ]$}
        %\RightLabel{Comm.}
        %\UIC{$(\tr[A]\ra\tr[B]\prod\tr[Γ])\prod\emptyset\fCenter\tr[Δ]$}
        %\RightLabel{$\emptyset$E}
        \BIC{$\tr[A]\ra\tr[B]\prod\tr[Γ]\fCenter\tr[Δ]$}
        \RightLabel{$\ra$I}
        \UIC{$\tr[A]\ra\tr[B]\fCenter\tr[Γ]\ra\tr[Δ]$}
      \end{pfbox}
      \\&
      \begin{pfbox}[0.9]
        \AXC{$Γ\fCenter\struct{A}$}
        \AXC{$\struct{B}\fCenter Δ$}
        \RightLabel{L$\impl$}
        \BIC{$\struct{B\impl A}\fCenter Δ\impl Γ$}
      \end{pfbox}
      &$\Longrightarrow$&
      \multicolumn{1}{c}{\text{(as above)}}
      \\&
      \begin{pfbox}[0.9]
        \AXC{$Γ_2\fCenter Γ_1\impr Δ$}
        \RightLabel{Res$\impr\prod$}
        \UIC{$Γ_1\prod Γ_2\fCenter Δ$}
      \end{pfbox}
      &$\Longrightarrow$&
      \begin{pfbox}[0.9]
        \AXC{$\tr[Γ_2]\fCenter\tr[Γ_1]\ra\tr[Δ]$}
        \AXC{}\RightLabel{Ax}\UIC{$\tr[Γ_1]\fCenter\tr[Γ_1]$}
        \RightLabel{$\ra$E}
        \BIC{$\tr[Γ_2]\prod\tr[Γ_1]\fCenter\tr[Δ]$}
        \RightLabel{Comm.}
        \UIC{$\tr[Γ_1]\prod\tr[Γ_2]\fCenter\tr[Δ]$}
      \end{pfbox}
      \\&
      \begin{pfbox}[0.9]
        \AXC{$Γ_1\prod Γ_2\fCenter Δ$}
        \RightLabel{Res$\prod\impr$}
        \UIC{$Γ_2\fCenter Γ_1\impr Δ$}
      \end{pfbox}
      &$\Longrightarrow$&
      \begin{pfbox}[0.9]
        \AXC{$\tr[Γ_1]\prod\tr[Γ_2]\fCenter\tr[Δ]$}
        \RightLabel{Comm.}
        \UIC{$\tr[Γ_2]\prod\tr[Γ_1]\fCenter\tr[Δ]$}
        \RightLabel{$\ra$I}
        \UIC{$\tr[Γ_2]\fCenter\tr[Γ_1]\ra\tr[Δ]$}
      \end{pfbox}
      \\&
      \begin{pfbox}[0.9]
        \AXC{$Γ_1\fCenter Δ\impl Γ_2$}
        \RightLabel{Res$\impl\prod$}
        \UIC{$Γ_1\prod Γ_2\fCenter Δ$}
      \end{pfbox}
      &$\Longrightarrow$&
      \begin{pfbox}[0.9]
        \AXC{$\tr[Γ_1]\fCenter\tr[Γ_2]\ra\tr[Δ]$}
        \AXC{}\RightLabel{Ax}\UIC{$\tr[Γ_2]\fCenter\tr[Γ_2]$}
        \RightLabel{$\ra$E}
        \BIC{$\tr[Γ_1]\prod\tr[Γ_2]\fCenter\tr[Δ]$}
      \end{pfbox}
      \\&
      \begin{pfbox}[0.9]
        \AXC{$Γ_1\prod Γ_2\fCenter Δ$}
        \RightLabel{Res$\prod\impl$}
        \UIC{$Γ_1\fCenter Δ\impl Γ_2$}
      \end{pfbox}
      &$\Longrightarrow$&
      \begin{pfbox}[0.9]
        \AXC{$\tr[Γ_1]\prod\tr[Γ_2]\fCenter\tr[Δ]$}
        \RightLabel{$\ra$I}
        \UIC{$\tr[Γ_1]\fCenter\tr[Γ_2]\ra\tr[Δ]$}
      \end{pfbox}
    \end{tabular}
    \vspace*{\baselineskip}
  \end{mdframed}
  \caption{Translation from display calculus to explicit \lamET.}
  \label{fig:display-calculus-to-explicit-lamET}
  \end{figure}
\end{landscape}
%
\begin{figure}
  \begin{mdframed}
    \centering
    \vspace*{1\baselineskip}
    \(\!
      \begin{aligned}
        &\text{Pol}(α)        &&\mapsto{-} \\
        &\text{Pol}(B\impl A) &&\mapsto{-}
      \end{aligned}
      \quad
      \begin{aligned}
        \\
        &\text{Pol}(A\impr B) &&\mapsto{-}
      \end{aligned}
    \)
    \\[1\baselineskip]
    \(\!
    \cancel{
      \AXC{}
      \RightLabel{Ax}
      \UIC{$\struct{α}\fCenter\struct{α}$}
      \DisplayProof
    }
    \)
    \\[1\baselineskip]
    \(\!
    \color{gray}
    \text{if}\ \text{Pol}(α) = {+}
    \left\lbrace
      \quad
      \begin{aligned}
        \\
        \AXC{}
        \RightLabel{Ax$^R$}
        \UIC{$\struct{α}\fCenter\focus{α}$}
        \DisplayProof
        \\[1\baselineskip]
      \end{aligned}
      \quad
      \normalcolor
      \middle\vert
      \normalcolor
      \quad
      \begin{aligned}
        \\
        \AXC{}
        \RightLabel{Ax$^L$}
        \UIC{$\focus{α}\fCenter\struct{α}$}
        \DisplayProof
        \\[1\baselineskip]
      \end{aligned}
      \quad
    \right\rbrace
    \normalcolor
    \text{if}\ \text{Pol}(α) = {-}
    \)
    \\[1\baselineskip]
    \(\!
    \text{if}\ \text{Pol}(A) = {+}
    \left\lbrace
      \quad
      \begin{aligned}
        \\
        \AXC{$Γ\fCenter\focus{A}$}
        \RightLabel{Foc$^R$}
        \UIC{$Γ\fCenter\struct{A}$}
        \DisplayProof
        \\[1\baselineskip]
        \AXC{$\struct{A}\fCenter Δ$}
        \RightLabel{Unf$^L$}
        \UIC{$\focus{A}\fCenterΔ$}
        \DisplayProof
        \\[1\baselineskip]
      \end{aligned}
      \quad
      \middle\vert
      \quad
      \begin{aligned}
        \\
        \AXC{$\focus{A}\fCenterΔ$}
        \RightLabel{Foc$^L$}
        \UIC{$\struct{A}\fCenter Δ$}
        \DisplayProof
        \\[1\baselineskip]
        \AXC{$Γ\fCenter\struct{A}$}
        \RightLabel{Unf$^R$}
        \UIC{$Γ\fCenter\focus{A}$}
        \DisplayProof
        \\[1\baselineskip]
      \end{aligned}
      \quad
    \right\rbrace
    \text{if}\ \text{Pol}(A) = {-}
    \)
    \\[1\baselineskip]
    \begin{pfbox}
      \AXC{$Γ\fCenter\focus{A}$}
      \AXC{$\focus{B}\fCenter Δ$}
      \RightLabel{L$\impr$}
      \BIC{$\focus{A\impr B}\fCenter Γ\impr Δ$}
    \end{pfbox}
    \begin{pfbox}
      \AXC{$Γ\fCenter\focus{A}$}
      \AXC{$\focus{B}\fCenter Δ$}
      \RightLabel{L$\impl$}
      \BIC{$\focus{B\impl A}\fCenter Δ\impl Γ$}
    \end{pfbox}
    \vspace*{1\baselineskip}
  \end{mdframed}
  \caption{Changes to the display calculus
    from~\autoref{fig:display-calculus}, implementing focusing.}
  \label{fig:focused-display-calculus}
\end{figure}
%
%%% Local Variables:
%%% mode: latex
%%% TeX-master: t
%%% End:

\begin{figure}[h]
  \begin{mdframed}
    \centering
    \[\text{Type}\;A,B\coloneqq\ldots\vsep A\& B\vsep A\oplus B\]
    \begin{pfbox}
      \AXC{$\struct{A}\fCenter Δ$}
      \RightLabel{L\&$_1$}
      \UIC{$\struct{A\& B}\fCenter Δ$}
    \end{pfbox}
    \begin{pfbox}
      \AXC{$\struct{B}\fCenter Δ$}
      \RightLabel{L\&$_2$}
      \UIC{$\struct{A\& B}\fCenter Δ$}
    \end{pfbox}
    \begin{pfbox}
      \AXC{$Γ\fCenter\struct{A}$}
      \AXC{$Γ\fCenter\struct{B}$}
      \RightLabel{R\&}
      \BIC{$Γ\fCenter\struct{A\& B}$}
    \end{pfbox}
    \\[1\baselineskip]
    \begin{pfbox}
      \AXC{$\struct{A}\fCenter Δ$}
      \AXC{$\struct{B}\fCenter Δ$}
      \RightLabel{L$\oplus$}
      \BIC{$\struct{A\oplus B}\fCenter Δ$}
    \end{pfbox}
    \begin{pfbox}
      \AXC{$Γ\fCenter\struct{A}$}
      \RightLabel{R$\oplus_1$}
      \UIC{$Γ\fCenter\struct{A\oplus B}$}
    \end{pfbox}
    \begin{pfbox}
      \AXC{$Γ\fCenter\struct{B}$}
      \RightLabel{R$\oplus_2$}
      \UIC{$Γ\fCenter\struct{A\oplus B}$}
    \end{pfbox}
    \\[1\baselineskip]
    \hrulefill
    \[
      \tr[(A\& B)] \mapsto \tr[A]\times\tr[B]
    \]
    \begin{pfblock}
      \AXC{$x:\struct{A}\fCenter M:Δ$}
      \RightLabel{L\&$_1$}
      \UIC{$z:\struct{A\& B}\fCenter \case{z}{x}{\_}{M}:Δ$}
    \end{pfblock}
    \begin{pfblock}
      \AXC{$y:\struct{B}\fCenter M:Δ$}
      \RightLabel{L\&$_2$}
      \UIC{$z:\struct{A\& B}\fCenter \case{z}{\_}{y}{M}:Δ$}
    \end{pfblock}
    \begin{pfblock}
      \AXC{$x:Γ\fCenter\struct{M:A}$}
      \AXC{$x:Γ\fCenter\struct{N:B}$}
      \RightLabel{R\&}
      \BIC{$x:Γ\fCenter\struct{(M,N):A\& B}$}
    \end{pfblock}
    \vspace*{0.5\baselineskip}
  \end{mdframed}
  \caption{
    Extension of calculus in \autoref{fig:nl-display-calculus} which supports ambiguity.}%
  \label{fig:extension-lexical-ambiguity}
\end{figure}

\begin{figure}[hb]
  \begin{mdframed}
    \centering
    \begin{minipage}{0.666\linewidth}
      \centering
      \begin{alignat*}{4}
        \text{Type}     &  \;&A,B&\coloneqq\ldots\vsep A\himpr B\vsep B\himpl A\vsep\q[A]\\
        \text{Structure}&^+\;&Γ  &\coloneqq\ldots\vsep Γ_1\hprod Γ_2\vsep\I\vsep\B\vsep\C\\
        \text{Structure}&^-\;&Δ  &\coloneqq\ldots\vsep Γ\himpr Δ\vsep Δ\himpl Γ
      \end{alignat*}
    \end{minipage}%
    \begin{minipage}{0.333\linewidth}
      \centering
      \begin{alignat*}{4}
        &\text{Pol}(A\himpr B) &&\mapsto{-}\\
        &\text{Pol}(B\himpl A) &&\mapsto{-}\\
        &\text{Pol}(\q[A])    &&\mapsto{+}
      \end{alignat*}
    \end{minipage}
    \\[1\baselineskip]
    (copy of rules for $\{\impr,\prod,\impl\}$ from
    \autoref{fig:display-calculus} for $\{\himpr,\hprod,\himpl\}$)
    \\[1\baselineskip]
    \begin{pfbox}
      \AXC{$\struct{A}\hprod\I\fCenter Δ$}
      \RightLabel{L\I}
      \UIC{$\struct{\q[A]}\fCenter Δ$}
    \end{pfbox}
    \begin{pfbox}
      \AXC{$Γ\fCenter\focus{B}$}
      \RightLabel{R\I}
      \UIC{$Γ\hprod\I\fCenter\focus{\q[B]}$}
    \end{pfbox}
    \begin{pfbox}
      \AXC{$Γ\fCenter Δ$}
      \RightLabel{$\I^-$}
      \UIC{$Γ\hprod\I\fCenter Δ$}
    \end{pfbox}
    \\[1\baselineskip]
    \begin{pfbox}
      \AXC{$Γ_1\prod(Γ_2\hprod Γ_3)\fCenter Δ$}
      \doubleLine\RightLabel{\B}
      \UIC{$Γ_2\hprod((\B\prod Γ_1)\prod Γ_3)\fCenter Δ$}
    \end{pfbox}
    \begin{pfbox}
      \AXC{$(Γ_1\hprod Γ_2)\prod Γ_3\fCenter Δ$}
      \doubleLine\RightLabel{\C}
      \UIC{$Γ_1\hprod((\C\prod Γ_2)\prod Γ_3)\fCenter Δ$}
    \end{pfbox}
    \\[1\baselineskip]
    \hrulefill
    \\[1\baselineskip]
    {
      \renewcommand{\arraystretch}{1.5}%
      \(\!
      \begin{array}{c c c}
        \multicolumn{3}{c}{\tr[({\q[A]})]\mapsto\tr[A]}\\
        \tr[\I]\mapsto\top      & \tr [\B]\mapsto\top     & \tr [\C]\mapsto\top\\
      \end{array}
      \)
    }
    \\[1\baselineskip]
    (copy of translations for $\{\impr,\prod,\impl\}$ from
    \autoref{fig:display-calculus-to-explicit-lamET} for
    $\{\himpr,\hprod,\himpl\}$)
    \\[1\baselineskip]
    \begin{pfbox}
      \AXC{$x:\struct{A}\hprod\I\fCenter M:Δ$}
      \RightLabel{L\I}
      \UIC{$y:\struct{\q[A]}\fCenter \sub{M}{(y,())}{x}:Δ$}
    \end{pfbox}
    \begin{pfbox}
      \AXC{$x:Γ\fCenter\focus{M:B}$}
      \RightLabel{R\I}
      \UIC{$y:Γ\hprod\I\fCenter\focus{\sub{M}{\fst{y}}{x}:\q[B]}$}
    \end{pfbox}
    \begin{pfbox}
      \AXC{$x:Γ\fCenter M:Δ$}
      \RightLabel{$\I^-$}
      \UIC{$y:Γ\hprod\I\fCenter \sub{M}{\fst{y}}{x}:Δ$}
    \end{pfbox}
    \\[1\baselineskip]
    (where $\fst{x}=\case{x}{y}{z}{y}$)
    \\[1\baselineskip]
    (\B\ and \C\ translate to various combinations of associativity,
    commutativity, $\emptyset$E and weakening)
    \\
    \vspace*{\baselineskip}
  \end{mdframed}
  \caption{
    Extension of calculus in \autoref{fig:display-calculus} which
    supports quantifier raising.}%
  \label{fig:extension-quantifier-raising}
\end{figure}

%%% Local Variables:
%%% mode: latex
%%% TeX-master: t
%%% End:

\begin{figure}[hb]
  \begin{mdframed}
    \centering
    \begin{minipage}{0.666\linewidth}
      \centering
      \begin{alignat*}{4}
        \text{Type}     &  \;&A,B&\coloneqq\ldots\vsep\di A\vsep\sq A\\
        \text{Structure}&^+\;&Γ  &\coloneqq\ldots\vsep\langle Γ\rangle\\
        \text{Structure}&^-\;&Δ  &\coloneqq\ldots\vsep[Δ]
      \end{alignat*}
    \end{minipage}%
    \begin{minipage}{0.333\linewidth}
      \centering
      \begin{alignat*}{4}
        &\text{Pol}(\di A) &&\mapsto{+}\\
        &\text{Pol}(\sq B) &&\mapsto{-}\\
      \end{alignat*}
    \end{minipage}
    \\[1\baselineskip]
    \begin{pfbox}
      \AXC{$\langle\struct{A}\rangle\fCenter Δ$}
      \RightLabel{L$\di$}
      \UIC{$\struct{\di A}\fCenter Δ$}
    \end{pfbox}
    \begin{pfbox}
      \AXC{$Γ\fCenter\focus{B}$}
      \RightLabel{R$\di$}
      \UIC{$\langle Γ\rangle\fCenter\focus{\di B}$}
    \end{pfbox}
    \\[1\baselineskip]
    \begin{pfbox}
      \AXC{$\focus{A}\fCenter Δ$}
      \RightLabel{L$\di$}
      \UIC{$\focus{\sq A}\fCenter[Δ]$}
    \end{pfbox}
    \begin{pfbox}
      \AXC{$Γ\fCenter[\struct{B}]$}
      \RightLabel{R$\sq$}
      \UIC{$Γ\fCenter\struct{\sq B}$}
    \end{pfbox}
    \\[1\baselineskip]
    \begin{pfbox}
      \AXC{$Γ\fCenter[Δ]$}
      \doubleLine\RightLabel{Res$\sq\di$}
      \UIC{$\langle Γ\rangle\fCenter Δ$}
    \end{pfbox}
    \\[1\baselineskip]
    \hrulefill
    \\[1\baselineskip]
    {
      \renewcommand{\arraystretch}{1.5}%
      \(
      \begin{array}{c c c}
        \tr [\di A]             \mapsto\tr [A]&
        \tr [\langle Γ \rangle] \mapsto\tr [Γ]&
        \trd[\langle Γ \rangle] \mapsto\trd[Γ]\\
        \tr [\sq A]             \mapsto\tr [A]&
        \tr [{[}Δ{]}]           \mapsto\tr [Δ]\\
      \end{array}
      \)
    }
    \\[1\baselineskip]
    (all rules translate to the identity)
    \vspace*{1\baselineskip}
  \end{mdframed}
  \caption{
    Extension of calculus in \autoref{fig:extension-quantifier-raising}
    which supports scope islands.}%
  \label{fig:extension-scope-islands}
\end{figure}

%%% Local Variables:
%%% mode: latex
%%% TeX-master: t
%%% End:


\begin{comment}
  \subsubsection*{Completeness of $\uparrow$ and $\downarrow$ w.r.t.\
    \textbf{B} and \textbf{C}}%
  We will assume the following definitions for contexts, the plugging
  operator $\cdot\:[\:\cdot\:]$, and the trace function:
  \begin{center}
    $\text{Context}\;Σ\coloneqq\Box\vsep Σ\prodl Γ\vsep Γ\prodr Σ$\\
    \begin{minipage}{0.45\linewidth}
      \begin{alignat*}{2}
        &\Box       \;&&[Γ']\mapsto Γ'\\
        &(Σ\prodl Γ)\;&&[Γ']\mapsto (Σ[Γ']\prod Γ)\\
        &(Γ\prodr Σ)\;&&[Γ']\mapsto (Γ\prod Σ[Γ'])
      \end{alignat*}
    \end{minipage}
    \begin{minipage}{0.45\linewidth}
      \begin{alignat*}{2}
        &\text{trace}(\Box)     \;&&\mapsto \mathbf{I}\\
        &\text{trace}(Σ\prodl Γ)\;&&\mapsto ((\mathbf{C}\prod \text{trace}(Σ))\prod Γ)\\
        &\text{trace}(Γ\prodr Σ)\;&&\mapsto ((\mathbf{B}\prod Γ)\prod
        \text{trace}(Σ))
      \end{alignat*}
    \end{minipage}
  \end{center}
  Given these definitions, we can show that the following rules for
  quantifier raising are derivable:
  \begin{center}
    \vspace*{0.5\baselineskip}
    \begin{pfbox}
      \AXC{$\text{trace}(Σ)\fCenter[A\himpr B]$} \AXC{$[C]\fCenter Δ$}
      \RightLabel{$\uparrow$}
      \BIC{$Σ[\cdot\mathbf{Q}(C\himpl(A\himpr B))\cdot]\fCenter Δ$}
    \end{pfbox}
    \begin{pfbox}
      \AXC{$Σ[\cdot A\cdot]\fCenter\cdot B\cdot$}
      \RightLabel{$\downarrow$}
      \UIC{$\text{trace}(Σ)\fCenter\cdot A\himpr B\cdot$}
    \end{pfbox}
    \vspace*{0.5\baselineskip}
  \end{center}
  And these rules can be combined to form one full quantifier
  movement, reducing the type $\mathbf{Q}(C\himpl(A\himpr B))$ to $A$,
  while changing the top-level type from $C$ to $B$:
  \begin{center}
    \vspace*{0.5\baselineskip}
    \begin{pfbox}
      \AXC{$\vdots$}\noLine\UIC{$Σ[\cdot A\cdot]\fCenter\cdot B\cdot$}
      \RightLabel{$\downarrow$}
      \UIC{$\text{trace}(Σ)\fCenter\cdot A\himpr B\cdot$}
      \RightLabel{Foc$^R$} \UIC{$\text{trace}(Σ)\fCenter[A\himpr B]$}
      \AXC{$\vdots$}\noLine\UIC{$[C]\fCenter Δ$}
      \RightLabel{$\uparrow$}
      \BIC{$Σ[\cdot\mathbf{Q}(C\himpl(A\himpr B))\cdot]\fCenter Δ$}
    \end{pfbox}
    \vspace*{0.5\baselineskip}
  \end{center}
  We would like to show that, in fragment of the logic which is used
  for natural language, the derived rules $\uparrow$ and $\downarrow$
  are complete w.r.t.\ the structural rules \textbf{B} and
  \textbf{C}. For this purpose, we assume that:
  \begin{itemize}
  \item%
    there will be no occurrences of hollow structural connectives
    ($\!\!\himpl$,$\hp rod$,$\himpr$), \textbf{B} or \textbf{C} in the
    final sequent---the presence of these indicates unresolved
    movement, which means the sentence is not pronounceable;
  \item%
    all occurrences of the quantifying licence \textbf{Q} will be of
    the form $\mathbf{Q}(C\himpl(A\himpr B))$.
  \end{itemize}
  Under these assumptions, we can derive that the only interesting
  proofs which involve quantifiers will be of the form:
$$
L\mathbf{I} \ra\text{move up} \ra L\!\!{\himpl} \ra\text{auxiliary
  rules} \ra R{\himpr} \ra\text{move down} \ra\mathbf{I}^-
$$
There are three important facts to note here:
\begin{enumerate}
\item\label{no-axiom-BC}%
  \textbf{B}'s and \textbf{C}'s are structures, and therefore cannot
  be eliminated by axioms;
\item\label{cannot-overtake-Q}%
  during upwards or downwards movement, the quantifier is always
  attached to a hollow product, and the \textbf{B} and \textbf{C}
  rules only allow a quantifier to move past a \textit{solid} product;
  therefore, no quantifier can ever move past another quantifier;
\item%
  from \ref{no-axiom-BC} and \ref{cannot-overtake-Q}, we can derive
  that \textbf{B}'s and \textbf{C}'s introduced by upwards movement of
  a quantifier can only be eliminated by downwards movement \textit{of
    that same quantifier}.
\end{enumerate}
\note{%
  One move important fact to take note of is that the \textbf{B} and
  \textbf{C} are set up in such a way that every quantifier is forced
  to return to its original location. If the quantifier stops halfway
  through it's movement, there will be \textbf{B}'s and \textbf{C}'s
  left over in the sequent---and since \textbf{B}'s and \textbf{C}'s
  are always introduced in a positive context, and polarity is
  maintained, they can never be eliminated by an axiom.\\
  During upwards (or downwards) movement, the main connective on the
  left-hand side is always the hollow product. At every step, there
  are three possibilities:
  \begin{enumerate}
  \item Move further up (or down) using \textbf{B} and \textbf{C};
  \item Use residuation to focus on the left-hand side of the hollow
    product.\\
    During upwards movement, the left-hand side is necessarily a
    quantifier, and therefore the only applicable rule is
    $L\!\!\himpl$, which will eliminate the $\himpl$, and therefore
    stop the upwards movement.\\
    During downwards movement, the left-hand side is always a
    formula. Due to the fact that right-hand side will contain
    \textbf{B} and \textbf{C}, which can only be eliminated by
    downwards movement, there will be no way to finish the proof but
    through reversing the residuation, thereby forming a loop.
  \item Use residuation to focus on the right-hand side of the hollow
    product.\\
    However, any proof steps that work on the right-hand side of the
    hollow product will commute with upwards and downwards movement,
    and can therefore take place before upwards movement, or after
    downwards movement.
  \end{enumerate}%
  As for the interaction of multiple quantifiers, this is where the
  second modality (hollow) starts to play a role, as it ensures that
  no quantifier can ever move past a quantifier which is still moving,
  as the rules \textbf{B} and \textbf{C} only allow quantifiers to
  move past \textit{solid} products. This also entails that
  quantifiers can never switch places, as this would require one to
  move past the other.  }
\end{comment}

\end{document}
