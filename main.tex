\documentclass[a4paper]{article}

\usepackage{cmap}%
\usepackage[LGR,T1]{fontenc}%
\usepackage[utf8]{inputenc}%
\usepackage{alphabeta}%
\usepackage[greek,english]{babel}%
\languageattribute{greek}{polutoniko}%
\usepackage{hyperref}%
\addto\extrasenglish{\def\figureautorefname{figure}}%
\usepackage{lmodern}%

\usepackage{xcolor}
\usepackage{mathtools}
\usepackage{mdframed}
\usepackage[inline]{enumitem}
\usepackage{natbib}
\usepackage{pdflscape}
\usepackage{stmaryrd}
\usepackage{comment}
\usepackage{ifthen}
\usepackage{tikz}
\usepackage{tikz-qtree}
\usepackage{textcomp}

\usepackage{amsmath, amscd, amsthm, amssymb, mathrsfs, amsfonts, wasysym}%
\let\oldemptyset\emptyset%
\let\emptyset\varnothing%
\usepackage{cancel}%

\usepackage{tabularx}%
\usepackage{multirow}%
\renewcommand{\arraystretch}{3}%

\usepackage{pict2e}
\usepackage{picture}

\newcommand{\varslash}{%
  \mathbin{\mathpalette\pictslash{{0}{1}}}%
}
\newcommand{\varbslash}{%
  \mathbin{\mathpalette\pictslash{{1}{-1}}}%
}

\makeatletter
\newcommand{\pictslash}[2]{%
  \vcenter{%
    \sbox0{$\m@th#1\varobslash$}\dimen0=.55\wd0
    \hbox to\wd 0{%
      \hfil\pictslash@aux#2\hfil
    }%
  }%
}
\newcommand{\pictslash@aux}[2]{%
    \begin{picture}(\dimen0,\dimen0)
    \roundcap
    \linethickness{.15ex}
    \put(0,#1\dimen0){\line(1,#2){\dimen0}}
    \end{picture}%
}
\makeatother

\def\downmapsto{\rotatebox[origin=c]{270}{\ensuremath{\mapsto}}}%

\usepackage{bussproofs}%
\EnableBpAbbreviations%
\def\fCenter  {\mathbin{\vdash}}%
\def\la       {\leftarrow}%
\def\ra       {\rightarrow}%
\def\impl     {\mathbin{\slash}}%
\def\impr     {\mathbin{\backslash}}%
\def\prod     {\bullet}%
\def\prodl    {\bullet}%
\def\prodr    {\bullet}%
\def\himpl    {\!\!\fatslash\:}%
\def\himpr    {\fatbslash}%
\def\hprod    {\mathbin{\circ}}%
\def\hprodl   {\mathbin{\circ}}%
\def\hprodr   {\mathbin{\circ}}%
\def\trace    {\ensuremath{\text{trace}}}%
\def\unit     {\ensuremath{\mathbf{I}}}%
\def\sq       {\ensuremath{\Box}}%
\def\di       {\ensuremath{\Diamond}}%
\def\diIfx    {\ensuremath{\hat{\di}}}%
\def\sqIfx    {\ensuremath{\hat{\sq}}}%
\def\diExt    {\rotatebox[origin=c]{180}{\diIfx}}%
\def\sqExt    {\rotatebox[origin=c]{180}{\sqIfx}}%
\def\vsep     {\ \vert\ }%
\def\e        {\ensuremath{\mathbf{e}}}%
\def\t        {\ensuremath{\mathbf{t}}}%
\def\lamET    {\ensuremath{\lambda^{\ra}_{\{\e,\t\}}}}%
\def\S        {\text{S}}%
\def\N        {\text{N}}%
\def\NP       {\text{NP}}%
\def\PP       {\text{PP}}%
\def\INF      {\text{INF}}%
\def\A        {\text{A}}%
\def\IV       {\text{IV}}%
\def\TV       {\text{TV}}%
\def\I        {\ensuremath{\mathbf{I}}}%
\def\B        {\ensuremath{\mathbf{B}}}%
\def\C        {\ensuremath{\mathbf{C}}}%
\def\john     {\ensuremath{\text{john}}}%
\def\mary     {\ensuremath{\text{mary}}}%
\def\bill     {\ensuremath{\text{bill}}}%
\def\walks    {\ensuremath{\text{walks}}}%
\def\damned   {\ensuremath{\text{damned}}}%
\def\dog      {\ensuremath{\text{dog}}}%
\def\likes    {\ensuremath{\text{likes}}}%
\def\sees     {\ensuremath{\text{sees}}}%
\def\leave    {\ensuremath{\text{leave}}}%
\def\wants    {\ensuremath{\text{wants}}}%
\def\everyone {\ensuremath{\text{everyone}}}%
\def\different{\ensuremath{\text{different}}}%
\def\someone  {\ensuremath{\text{someone}}}%
\def\served   {\ensuremath{\text{served}}}%
\def\a        {\ensuremath{\text{a}}}%
\def\waiter   {\ensuremath{\text{waiter}}}%
\def\JOHN     {\ensuremath{\mathbf{john}}}%
\def\PERSON   {\ensuremath{\mathbf{person}}}%
\def\WAITER   {\ensuremath{\mathbf{waiter}}}%
\def\WALK     {\ensuremath{\mathbf{walk}}}%
\def\LIKE     {\ensuremath{\mathbf{like}}}%
\def\SERVE    {\ensuremath{\mathbf{serve}}}%
\def\SEE      {\ensuremath{\mathbf{see}}}%
\def\DAMN     {\ensuremath{\mathbf{damn}}}%
\def\PAST     {\ensuremath{\mathbf{past}}}%
\def\DOG      {\ensuremath{\mathbf{dog}}}%
\def\WANT     {\ensuremath{\mathbf{want}}}%
\def\plug     {\ensuremath{\:\cdot\:[\:\cdot\:]}}%
\def\qdown    {\ensuremath{\mathbf{Q}{\downarrow}}}
\def\qup      {\ensuremath{\mathbf{Q}{\uparrow}}}

\newcommand{\q}       [1][{\cdot}]{\mathbf{Q}(#1)}%
\newcommand{\focus}   [1]{\boxed{#1}}%
\newcommand{\struct}  [1]{{\cdot}#1{\cdot}}%
\newcommand{\sub}     [3]{#1[#2/#3]}%
\newcommand{\tr}      [1][({\cdot})]{#1^*}%
\newcommand{\trd}     [1][({\cdot})]{#1^{**}}%
\newcommand{\case}    [4]{\text{case}\;#1\;\text{of}\;(#2, #3)\ra#4}%
\newcommand{\add}     [2]{#1 + #2}
\newcommand{\mathplus}[0]{+}

\renewcommand*{\&}{%
  \relax
  \ifmmode
    \mathbin{\char`\&}%
  \else
    \char`\&\relax
  \fi
}

\newenvironment{pfbox}[1][0.9]%
  {\gdef\scalefactor{#1} \leavevmode\hbox\bgroup}
  {\scalebox{\scalefactor}{\DisplayProof} \egroup}

\newenvironment{pfblock}[1][0.9]%
  {\gdef\scalefactor{#1}\begin{center}\proofSkipAmount \leavevmode}%
  {\scalebox{\scalefactor}{\DisplayProof}\proofSkipAmount \end{center} }

\newboolean{notes}%
\setboolean{notes}{true}%
\providecommand{\todo}[1]{%
  \ifthenelse%
  {\boolean{notes}}%
  {\textcolor{red}{\_}\marginpar{\color{red}#1}}%
  {}}
\providecommand{\note}[1]{%
  \ifthenelse%
  {\boolean{notes}}%
  {{\color{green}\textbf{NOTE:~}#1}}%
  {}}


% rules from the Haskell code
\newcommand{\AxR}[1]{\AXC{}\RightLabel{Ax$^L$}\UIC{$#1$}}%
\newcommand{\AxL}[1]{\AXC{}\RightLabel{Ax$^R$}\UIC{$#1$}}%
\newcommand{\UnfR}[1]{\RightLabel{Foc$^R'$}\UIC{$#1$}}%
\newcommand{\UnfL}[1]{\RightLabel{Foc$^L'$}\UIC{$#1$}}%
\newcommand{\FocR}[1]{\RightLabel{Foc$^R$}\UIC{$#1$}}%
\newcommand{\FocL}[1]{\RightLabel{Foc$^L$}\UIC{$#1$}}%
\newcommand{\WithL}[2]{\RightLabel{\&L$_{#1}$}\UIC{$#2$}}%
\newcommand{\WithR}[1]{\RightLabel{\&R}\BIC{$#1$}}%
\newcommand{\ImpRL}[1]{\RightLabel{$\impr$L}\BIC{$#1$}}%
\newcommand{\ImpRR}[1]{\RightLabel{$\impr$R}\UIC{$#1$}}%
\newcommand{\ImpLL}[1]{\RightLabel{$\impl$L}\BIC{$#1$}}%
\newcommand{\ImpLR}[1]{\RightLabel{$\impl$R}\UIC{$#1$}}%
\newcommand{\Res}[3]{%
  \RightLabel{%
    \ifstrequal{1}{#1}{%
      \ifstrequal{1}{#2}{%
        Res$\impr\prod$ % Res11
      }{%
        \ifstrequal{2}{#2}{%
          Res$\prod\impr$ % Res12
        }{%
          \ifstrequal{2}{#2}{%
            Res$\impl\prod$ % Res13
          }{%
            Res$\prod\impl$ % Res14
          }
        }
      }
    }{%
      \ifstrequal{1}{#2}{%
        Res$\sq\di$ % Res21
      }{%
        Res$\di\sq$ % Res22
      }
    }
  }
  \UIC{$#3$}
}%
\newcommand{\DiaL}[1]{\RightLabel{$\di$L}\UIC{$#1$}}%
\newcommand{\DiaR}[1]{\RightLabel{$\di$R}\UIC{$#1$}}%
\newcommand{\BoxL}[1]{\RightLabel{$\sq$L}\UIC{$#1$}}%
\newcommand{\BoxR}[1]{\RightLabel{$\sq$R}\UIC{$#1$}}%
\newcommand{\IfxRR}[1]{\RightLabel{$\Ifx$RR}\UIC{$#1$}}%
\newcommand{\IfxLR}[1]{\RightLabel{$\Ifx$LR}\UIC{$#1$}}%
\newcommand{\IfxLL}[1]{\RightLabel{$\Ifx$LL}\UIC{$#1$}}%
\newcommand{\IfxRL}[1]{\RightLabel{$\Ifx$RL}\UIC{$#1$}}%
\newcommand{\ExtRR}[1]{\RightLabel{$\Ext$RR}\UIC{$#1$}}%
\newcommand{\ExtLR}[1]{\RightLabel{$\Ext$LR}\UIC{$#1$}}%
\newcommand{\ExtLL}[1]{\RightLabel{$\Ext$LL}\UIC{$#1$}}%
\newcommand{\ExtRL}[1]{\RightLabel{$\Ext$RL}\UIC{$#1$}}%
\newcommand{\UnitRL}[1]{\RightLabel{$\I$L}\UIC{$#1$}}%
\newcommand{\UnitRR}[1]{\RightLabel{$\I$R}\UIC{$#1$}}%
\newcommand{\UnitRI}[1]{\RightLabel{$\I^-$}\UIC{$#1$}}%
\newcommand{\DnB}[1]{\RightLabel{$\B$}\UIC{$#1$}}%
\newcommand{\UpB}[1]{\RightLabel{$\B'$}\UIC{$#1$}}%
\newcommand{\DnC}[1]{\RightLabel{$\C$}\UIC{$#1$}}%
\newcommand{\UpC}[1]{\RightLabel{$\C'$}\UIC{$#1$}}%

\ifdefined\usetheme\usetheme{Boadilla}\fi
\ifdefined\usecolortheme\usecolortheme{seagull}\fi
\ifdefined\beamertemplatenavigationsymbolsempty\beamertemplatenavigationsymbolsempty\fi


\begin{document}

\section{Introduction}
\label{sec:introduction}

In this thesis, I will discuss the grammar logic NLQ, an extension of
the non-associative Lambek calculus, which is capable of analysing
quantifier movement, scope islands, infixation and extraction.

% This thesis is not meant to be a work of thorough linguistic
% analysis.
What I hope to do in this thesis is to \emph{extend} and
\emph{solidify} the logical vocabulary with which such linguistic
analyses can be made. In this, I will use the following guiding
principles:
\begin{itemize}
\item We are constructing a \emph{grammar} logic. Therefore, we only
  want features in our logic for which we can demonstrate a motivating
  example from natural language.
\item We are constructing a grammar \emph{logic}. Therefore, we will
  only accept extensions to our logic if we can show that they
  preserve our most important properties: we want our logic to be
  reflexive and transitive, and want a procedure for proof
  search that is both \emph{decidable} and \emph{complete}.
\end{itemize}
I am under no impression that the extensions I am proposing will be
the be-all and end-all of logical grammar, so another important point
in this thesis will be \emph{modularity}.
It is incredibly important to formulate extensions in a modular
manner, so that other logical grammarians are free to mix and
match extensions without having to worry about unforeseen
interactions. There are two key techniques for this:
\begin{enumerate*}[label=(\arabic*)]
\item
  we use display calculus to get a general procedure for
  cut-elimination (\autoref{sec:display-calculus}); and
\item
  we associate each syntactic extension with its own set of
  connectives (or \emph{modality}) and make sure that the inference
  rules in that extension only apply in the presence of these
  connectives (\autoref{sec:syntactic-approaches-to-scope}).
\end{enumerate*}

Another key point will be \emph{unique normal-forms}---in our proof
search procedure, we only want to find a single proof for each
interpretation that a sentence has. In our calculus, we will achieve
this using \emph{focusing}
(\autoref{sec:focusing-and-spurious-ambiguity}).

The last key point in this thesis will be \emph{verification}. It is
far too easy to make mistakes when writing down logical proofs in a
pen-and-paper style, or when manually typesetting them in
\LaTeX. Therefore, most of the claims I make in this thesis will be
backed up by a pair of verified implementations of the full version
of NLQ (i.e. the version using all discussed extensions).
These verifications can be found in appendices A and B and on GitHub.

In appendix A, we discuss a formalisation in Agda \citep{norell2009}.
We implement the grammar logic, prove some key properties, and give a
formal semantics in the form of a translation from proofs in NLQ to
Agda terms.

In appendix B, we discuss a formalisation in Haskell
\citep{marlow2010} using the singletons library
\citet{eisenberg2012}. In the full version, we implement the grammar
logic, implement proof search, and give a formal semantics in the form
of a translation into a subset of Haskell which includes meaning
postulates. However, because the implementations of the grammar logic
are nearly identical, we restrict our discussion of the Haskell
version to the interface provided by the library, and how to write
your own lexicon and example sentences.

Starting in \autoref{sec:what-is-type-logical-grammar}, we will give a
brief introduction to type-logical grammar in general, and to what I
consider to be the base type-logical grammar: the non-associative
Lambek calculus (NL) paired with a simple semantic lambda calculus
(\lamET).
In \autoref{sec:display-calculus}, we will discuss the display
calculus formulation of NL, and motivate our usage of display
calculus.
Then, in sections \ref{sec:lexical-ambiguity} and
\ref{sec:movement-and-quantifier-raising}, we will discuss several
extension to the base type-logical grammar.

\subsection{What is type-logical grammar?}
\label{sec:what-is-type-logical-grammar}

Before we address the question of what type-logical grammar is, let us
try and get an idea of what problem it is trying to solve. Have a look
at the abstract pipeline for natural language understanding (NLU) in
\autoref{fig:abstract-nlu-pipeline}.

\begin{figure}[hb]
  \begin{mdframed}
    \centering
    \begin{minipage}{0.35\linewidth}%
      \centering
      \framebox{Morphological}\\
      $\downarrow$\\
      \framebox{Lexical}\\
      $\downarrow$\\
      \framebox{Syntactic}\\
      $\downarrow$\\
      \framebox{Semantic}\\
      $\downarrow$\\
      \framebox{Pragmatic}\\
    \end{minipage}
    \begin{minipage}{0.55\linewidth}%
      \centering
      ``Mary saw foxes.''\\
      $\downarrow$\\
      Mary see.PAST fox.PL\\
      $\downarrow$\\
      Mary:NP see:TV.PAST fox:NP.PL\\
      $\downarrow$\\
      Mary:NP [see:TV.PAST fox:NP.PL]\\
      $\downarrow$\\
      $\exists X. X \subseteq\mathbf{fox}\land\mathbf{past}(\mathbf{see}(\text{Mary},X))$\\
      $\downarrow$\\
      \ldots\footnotemark\\
    \end{minipage}
  \end{mdframed}
  \caption{An abstract pipeline for natural language
    understanding.}%
  \label{fig:abstract-nlu-pipeline}
\end{figure}%
\footnotetext{%
  As the pragmatic function is all about integrating context (be it
  textual or environmental) into the meaning, it would not make sense
  to list something here.
}


To the left of the figure, you see the various phases or functions
commonly associated with an NLU-pipeline. To the right, you see the
inputs and outputs of these functions.
For instance, the morphological function will take an unanalysed
sentence, and return a sentence which is lemmatised. This entails that
all morphemes are made explicit---for instance, in the case of the
example in \autoref{fig:abstract-nlu-pipeline}, the previously
``implicit'' morphemes for past tense and plurality are added.

There is some disagreement on the exact role of type-logical grammars in
this pipeline. Ideally, type-logical grammars would play the role of
both the syntactic and the semantic function. However, the current
state of affairs in research is that often only the semantic function
is truly considered\footnote{%
  This statement is not true for \emph{associative} type-logical
  grammars, which fundamentally reject the tree structure of
  language---that is, they assume that the meaning of a sentence
  depends solely on the linear order of words, and not on some hidden
  tree structure.
}. This makes sense from a research perspective: we can refer to the
huge body of work on generative grammar to inform our choice for
sentence structure, and focus on assigning the right meaning to these
structures. This is also the approach we will also take in this
thesis---that is, we consider type-logical grammars to be the
function:
\begin{center}
  Mary:NP [see:TV.PAST fox:NP.PL]\\
  $\downarrow$\\
  \framebox{Type-Logical Grammar}\\
  $\downarrow$\\
  $\exists X.X\subseteq\mathbf{fox}\land\mathbf{past}(\mathbf{see}(\text{mary},X))$
\end{center}
That is, it is a function which, given some structured and typed input
which represents the syntactic structure of a sentence, returns the
meaning(s) associated with that sentence.

Given the presence of the phrase ``structured and typed'', we may
already suspect that type theory offers a fitting solution to this
problem. And indeed, under the guise of type-logical grammar, it
does. A type-logical grammar generally consists of three things:
\begin{enumerate}[label=(\arabic*)]
\item a syntactic calculus, set up in such a fashion that only
  grammatical sentences are well-typed, and for which an efficiently
  decidable procedure for proof-search exists;
\item a semantic calculus, used to represent the meanings of words and
  sentences; and
\item a translation from the syntactic to the semantic calculus.
\end{enumerate}
We interpret the part-of-speech tags in our input (NP, TV, etc.) as
types in the syntactic calculus, and combine these with the desired
type for the tree---usually \S\ for `sentence'---to form an input
sequent.
We then search for a proof of that sequent in the syntactic calculus,
and translate it to a term in the semantic calculus.
Once there, we interpret the morphemes (e.g.\ lemmas, \texttt{PAST},
\texttt{PL}, etc.) as terms in the semantic calculus.

In \autoref{sec:simple-type-logical-grammar}, we will have a look at
the base type-logical grammar, and give some examples of the process
of deriving sentence meaning.



\subsection{A simple type-logical grammar}
\label{sec:simple-type-logical-grammar}

The simplest type-logical grammar that comes to mind---drawing
heavily from Montague grammar and categorial grammar---is composed of
the simply-typed lambda calculus with atomic types \e\ and \t\ (\lamET) as
a semantic calculus, and the non-associative Lambek calculus
\citep[NL;][]{lambek1961} as a syntactic calculus.

The usual natural deduction formulation of \lamET\ can to be seen in
\autoref{fig:implicit-lamET}. It is a simple lambda
calculus, with atomic types \e\ (`entity') and \t\ (`truth-value').
In addition, we usually assume that any logical operator or
word-meanings we need is defined as a constant of the appropriate
type. For instance, $\forall$ is a constant of type $(\e\ra\t)\ra\t$,
and `john' is a constant of type \e. Note that we will sometimes write
logical operators in their usual notation, e.g.\ $M\wedge N$ or
$\forall x.M$, but this should be taken as syntactic sugar, in the
case of our examples rewriting to $(({\wedge}\;M)\;N)$ and
$\forall\;(\lambda{x}.M)$, respectively. Additionally, we will
occasionally write e.g.\ $\e\e\t$ instead of $\e\ra\e\ra\t$, or
$(\e\t)\t$ instead of $(\e\ra\t)\ra\t$, using adjacency to mean
implication.

\begin{figure}[h]
  \begin{mdframed}
    \centering
    \begin{alignat*}{4}
      \text{Atom}       \;&α  &&\coloneqq \e\vsep\t\\
      \text{Type}       \;&A,B&&\coloneqq α\vsep A\ra B\\
      \text{Term}       \;&M,N&&\coloneqq x\vsep C\vsep\lambda x.M\vsep(M\;N)\\
      \text{Constant}   \;&C  &&\coloneqq
      {\forall}\vsep{\exists}\vsep{\neg}\vsep{\supset}\vsep{\land}\vsep{\lor}\vsep\ldots\\
      \text{Environment}\;&Γ  &&\text{set of typing assumptions of the form `$M : A$'}
    \end{alignat*}

    \begin{pfbox}
      \AXC{$(x : A)\in Γ$} \RightLabel{{Ax}} \UIC{$Γ\fCenter x : A$}
    \end{pfbox}
    \\[1\baselineskip]
    \begin{pfbox}
      \AXC{$Γ,x : A\fCenter M : B$} \RightLabel{$\ra${I}}
      \UIC{$Γ\fCenter \lambda x.M : A\ra B$}
    \end{pfbox}
    \begin{pfbox}
      \AXC{$Γ\fCenter M : A\ra B$} \AXC{$Γ\fCenter N : A$}
      \RightLabel{$\ra${E}} \BIC{$Γ\fCenter (M\;N) : B$}
    \end{pfbox}

    \todo{Reference to Montague?}

    \vspace*{1\baselineskip}
  \end{mdframed}
  \caption{\lamET, a simple semantic calculus.}%
  \label{fig:implicit-lamET}
\end{figure}
%

Using this calculus as a semantics function directly would
over-generate, e.g.\ for the sequent $\{\text{john}:\e,
\text{likes}:\e\ra\e\ra\t, \text{mary}:\e\}\fCenter\t$ we can derive
$((\text{likes}\; \text{john})\; \text{mary})$, $((\text{likes}\;
\text{mary})\; \text{john})$, $((\text{likes}\; \text{mary})\;
\text{mary})$ and $((\text{likes}\; \text{john})\; \text{john})$.
The reason for this is, of course, that the set structure used in this
formulation is much too expressive for natural language grammar.

If we want more control over the structure of our terms, a good first
step is to move to a purely syntactic formulation, where all the
structural properties are made explicit in the calculus itself; this
has been done in \autoref{fig:explicit-lamET}. We have
replaced the set by a (possibly empty) binary tree, spanned by the
structural product `$\prod$'. We have also included a number of new
structural rules, which implement the structure of a set: $\emptyset$E
and $\emptyset$I allow us to have an empty antecedent; contraction and
weakening tell us that we can use formulas multiple times or not at
all; and with commutativity and associativity we can change the order
of the formulas any way we like.

\begin{figure}[h]
  \begin{mdframed}
    \centering
    \begin{minipage}{0.6\linewidth}
      \begin{alignat*}{4}
        \text{Atom}     \;&α    &&\coloneqq \e\vsep\t\\
        \text{Type}     \;&A,B  &&\coloneqq α\vsep A\ra B\\
        \text{Structure}\;&Γ,Δ,Π&&\coloneqq A\vsep\emptyset\vsep Γ\prod Δ\\
        \text{Context}  \;&Σ    &&\coloneqq \Box\vsep Σ\prodl Δ\vsep Γ\prodr Σ
      \end{alignat*}
    \end{minipage}%
    \begin{minipage}{0.4\linewidth}
      \begin{align*}
        \Box [Γ]&\mapsto Γ\\
        (Σ\prodl Δ)[Γ]&\mapsto (Σ[Γ]\prod Δ)\\
        (Δ\prodr Σ)[Γ]&\mapsto (Δ\prod Σ[Γ])
      \end{align*}
    \end{minipage}

    \vspace*{\baselineskip}
    \begin{pfbox}
      \AXC{}\RightLabel{{Ax}}\UIC{$A\fCenter A$}
    \end{pfbox}

    \vspace*{\baselineskip}
    \begin{pfbox}
      \AXC{$Γ\prod A\fCenter B$}
      \RightLabel{$\ra${I}}
      \UIC{$Γ\fCenter A\ra B$}
    \end{pfbox}
    \begin{pfbox}
      \AXC{$Γ\fCenter A\ra B$}
      \AXC{$Δ\fCenter A$}
      \RightLabel{$\ra${E}}
      \BIC{$Γ\prod Δ\fCenter B$}
    \end{pfbox}

    \vspace*{\baselineskip}
    \begin{pfbox}
      \AXC{$Σ[Γ\prod \emptyset]\fCenter B$}
      \RightLabel{$\emptyset${E}}
      \UIC{$Σ[Γ]\fCenter B$}
    \end{pfbox}
    \begin{pfbox}
      \AXC{$Σ[Γ]\fCenter B$}
      \RightLabel{$\emptyset${I}}
      \UIC{$Σ[Γ\prod \emptyset]\fCenter B$}
    \end{pfbox}

    \vspace*{\baselineskip}
    \begin{pfbox}
      \AXC{$Σ[A\prod A]\fCenter B$}
      \RightLabel{Cont.}
      \UIC{$Σ[A]\fCenter B$}
    \end{pfbox}
    \begin{pfbox}
      \AXC{$Σ[Γ]\fCenter B$}
      \RightLabel{Weak.}
      \UIC{$Σ[Γ\prod A]\fCenter B$}
    \end{pfbox}

    \vspace*{\baselineskip}
    \begin{pfbox}
      \AXC{$Σ[Δ\prod Γ]\fCenter B$}
      \RightLabel{Comm.}
      \UIC{$Σ[Γ\prod Δ]\fCenter B$}
    \end{pfbox}
    \begin{pfbox}
      \AXC{$Σ[(Γ\prod Δ)\prod Π]\fCenter B$}
      \doubleLine\RightLabel{Ass.}
      \UIC{$Σ[Γ\prod (Δ\prod Π)]\fCenter B$}
    \end{pfbox}
    \vspace*{\baselineskip}
  \end{mdframed}
  \caption{\lamET, with explicit structural rules.}%
  \label{fig:explicit-lamET}
\end{figure}
%
%%% Local Variables:
%%% mode: latex
%%% TeX-master: t
%%% End:


Note that, in order to define these structural rules, we had to define
the notion of a `context'---a structure with \emph{exactly one} hole
in it---and a plugging function `\plug'---a function which inserts a
structure into that hole. The reason for this is that we have to be
able to apply commutativity and associativity \emph{anywhere} in the
structure to be able to freely change the order (and
bracketing).\footnote{%
  The contexts are not strictly necessary for $\emptyset$E,
  contraction and weakening, since we can already move any formula
  anywhere we want, but they make the proof system much more usable
  and greatly decrease the length of proofs that need to use any of
  these structural rules.
}

It is not hard to convince yourself that the implicit and explicit
versions of \lamET\ are equivalent---though we will refrain from
giving the full proof here.
Because of this equivalence, we can use the term language from
\autoref{fig:implicit-lamET} for the explicit version of
\lamET.
The term labelling of the logical rules is exactly the same. The
structural rules only manipulate structures, and therefore do not
change the terms. The only exception to this is contraction, for which
the term labelling is as follows:
\begin{prooftree}
  \AXC{$Σ[y : A\prod z : A]\fCenter M : B$}
  \RightLabel{Cont.}
  \UIC{$Σ[x : A]\fCenter M[x/y][x/z] : B$}
\end{prooftree}
Contraction takes a term with two variables of the same type, and
contracts them using substitution, which is defined as usual:
\begin{alignat*}{3}
  &x             &&[N/y] \mapsto
  \begin{cases}
    N, &\text{if}\;x=y\\
    x, &\text{otherwise}
  \end{cases}
  \\
  &C             &&[N/y] \mapsto C\\
  &(\lambda x.M) &&[N/y] \mapsto
  \begin{cases}
    \lambda x.M[N/y], &\text{if}\;x=y\\
    \lambda x.M,      &\text{otherwise}
  \end{cases}
  \\
  &(M\;M')       &&[N/y] \mapsto (M[N/y]\;M'[N/y])
\end{alignat*}

Using our explicit semantic calculus, we can construct our syntactic
calculus in three simple steps:
\begin{enumerate}
\item%
  we drop \emph{all} structural rules;
\item%
  since the implication `$\ra$' can now only take arguments directly
  from the left, we add a second implication `$\la$' which can only
  take arguments from the right---by convention, implications in this
  system are written as `$\impr$' and `$\impl$' (pronounced ``under''
  and ``over'') with the argument type written \emph{under} the slash;
\item%
  we replace the atomic semantic types \e\ and \t\ by atomic syntactic
  types, reminiscent of part-of-speech tags---in this case, we will
  use S (`sentence'), NP (`noun phrase'), N (`noun'), PP
  (`prepositional phrase') and INF (`infinitive');
\end{enumerate}
The resulting system can be seen in \autoref{fig:nl-natural-deduction},
defined along with some definitions for common part-of-speech tags,
i.e.\ \A\ (`adjective'), \IV\ (`intransitive verb') and \TV\
(`transitive verb').

\begin{figure}[h]
  \begin{mdframed}
    \centering
    \begin{minipage}{0.66\linewidth}
      \begin{alignat*}{4}
        \text{Atom}     \;&α  &&\coloneqq \S\vsep\N\vsep\NP\vsep\INF\\
        \text{Type}     \;&A,B&&\coloneqq α\vsep A\impr B\vsep B\impl A\\
        \text{Structure}\;&Γ,Δ&&\coloneqq A\vsep Γ\prod Δ
      \end{alignat*}
    \end{minipage}%
    \begin{minipage}{0.33\linewidth}
      \begin{alignat*}{3}
        &\A \;&&\coloneqq \N\impl\N\\
        &\IV\;&&\coloneqq \NP\impr\S\\
        &\TV\;&&\coloneqq \IV\impl\NP
      \end{alignat*}
    \end{minipage}
    \\[1\baselineskip]
    \begin{pfbox}
      \AXC{} \RightLabel{{Ax}} \UIC{$A\fCenter A$}
    \end{pfbox}
    \\[1\baselineskip]
    \begin{pfbox}
      \AXC{$A\prodΓ\fCenter B$} \RightLabel{$\impr${I}}
      \UIC{$Γ\fCenter A\impr B$}
    \end{pfbox}
    \begin{pfbox}
      \AXC{$Γ\fCenter A$} \AXC{$Δ\fCenter A\impr B$} \RightLabel{$\impr${E}}
      \BIC{$Γ\prodΔ\fCenter B$}
    \end{pfbox}
    \\[1\baselineskip]
    \begin{pfbox}
      \AXC{$Γ\prod A\fCenter B$} \RightLabel{$\impl${I}}
      \UIC{$Γ\fCenter B\impl A$}
    \end{pfbox}
    \begin{pfbox}
      \AXC{$Γ\fCenter B\impl A$} \AXC{$Δ\fCenter A$} \RightLabel{$\impl${E}}
      \BIC{$Γ\prod Δ\fCenter B$}
    \end{pfbox}
    \vspace*{1\baselineskip}
  \end{mdframed}
  \caption{NL \citep{lambek1961} in natural deduction style.}%
  \label{fig:nl-natural-deduction}
\end{figure}
%
\begin{figure}[h]
  \begin{mdframed}
    \begin{minipage}{0.333\linewidth}
      \begin{alignat*}{4}
        &\tr[\S]         &&\mapsto\t              \\
        &\tr[\N]         &&\mapsto\e\ra\t         \\
        &\tr[\NP]        &&\mapsto\e              \\
        &\tr[\INF]       &&\mapsto\e\ra\t         \\
        &\tr[(A\impr B)] &&\mapsto\tr[A]\ra\tr[B] \\
        &\tr[(B\impl A)] &&\mapsto\tr[A]\ra\tr[B]
      \end{alignat*}
    \end{minipage}%
    \begin{minipage}{0.666\linewidth}
      \vspace*{1\baselineskip}
      \begin{prooftree}
        \AXC{}\RightLabel{Ax}\UIC{$x:A\fCenter x:A$}
      \end{prooftree}
      \begin{prooftree}
        \AXC{$x:A\prod Γ\fCenter M:B$}
        \RightLabel{$\impr$I}
        \UIC{$Γ\fCenter \lambda x.M:A\impr B$}
      \end{prooftree}
      \begin{prooftree}
        \AXC{$Γ\fCenter N:A$}
        \AXC{$Δ\fCenter M:A\impr B$}
        \RightLabel{$\impr$E}
        \BIC{$Γ\prod Δ\fCenter (M\;N):B$}
      \end{prooftree}
      \begin{prooftree}
        \AXC{$Γ\prod x:A\fCenter M:B$}
        \RightLabel{$\impl$I}
        \UIC{$Γ\fCenter \lambda x.M:B\impl A$}
      \end{prooftree}
      \begin{prooftree}
        \AXC{$Γ\fCenter M:B\impl A$}
        \AXC{$Δ\fCenter N:A$}
        \RightLabel{$\impl${E}}
        \BIC{$Γ\prod Δ\fCenter (M\;N):B$}
      \end{prooftree}
    \end{minipage}
    \vspace*{1\baselineskip}
  \end{mdframed}
  \caption{NL (\autoref{fig:nl-natural-deduction}) with term
    labelling in \lamET\ (\autoref{fig:implicit-lamET}).}%
  \label{fig:nl-natural-deduction-to-lamET}
\end{figure}


Dropping \emph{all} structural rules may seem unnecessary, but there
is a good motivation for each rule.  For example, in the presence of
commutativity, there is no way to distinguish between ``Mary walks''
and ``walks Mary''; under weakening, we can add any word anywhere in a
grammatical sentence, and the sentence will remain grammatical---
e.g.\ ``Mary banana walks''; and with contraction, we can remove
consecutive words with the same type---which means that ``John read
a fantastic blue book'' could be taken to mean the same thing as
``John read a blue book''.

With respect to associativity, \citet[][p.\ 167]{lambek1961} mentions
that ``the most natural assignments of types to English words [would]
admit many pseudo-sentences as grammatical, e.g.\ %
\begin{center}
  (*)~\itshape John is poor sad. John likes poor him. Who works and
  John rests?
\end{center}
More examples, including specific derivations, of ungrammatical
sentences that would be admitted in the presence of associativity and
the empty structure can be found in \citet[p.\ 33, 105-106]{moot2012}.

Note that we use the product-free version of NL. The reason for this
is that we have no use for the product in this thesis. Should you need
the product, however, it is very easily added:
\begin{center}
  \begin{pfbox}
    \AXC{$\Gamma\fCenter{A}$}
    \AXC{$\Delta\fCenter{B}$}
    \RightLabel{L$\otimes$}
    \BIC{$\Gamma\prod\Delta\fCenter{A\otimes{B}}$}
  \end{pfbox}
  \begin{pfbox}
    \AXC{$\Gamma\fCenter{A\otimes{B}}$}
    \AXC{$\Sigma[A\prod{B}]\fCenter{C}$}
    \RightLabel{R$\otimes$}
    \BIC{$\Sigma[\Gamma]\fCenter{C}$}
  \end{pfbox}
\end{center}

The last component we need for our simple type-logical grammar is a
translation from our syntactic calculus to our semantic calculus,
which consists of:
\begin{enumerate}[label=(\arabic*)]
\item
  a function $\tr$, translating the types in NL to types in \lamET; and
\item
  a set of rewrite rules, that rewrite proofs in NL to proofs in \lamET.
\end{enumerate}
However, in the interest of brevity, we will often give this second
translation directly as a term labelling. For instance, in
\autoref{fig:nl-natural-deduction-to-lamET}, we give the
translation on terms by directly labelling the rules of the syntactic
calculus with semantic terms. Because there is a one-to-one
correspondence between lambda terms and proofs, this is perfectly
unambiguous.

Note that we have chosen the particular translation for atomic types
in \autoref{fig:nl-natural-deduction-to-lamET} because it
aligns well with the remainder of this thesis. However, there are
different ways to define this translation---most notably,
\citepos{montague1973} worst-case generalisation for NPs, which
interprets them as having the type $(\e\t)\t$.

Now that we have a full type-logical grammar, let's give an example
analysis of the sentence ``Mary likes Bill''. We assume the
morphological, lexical and syntactic phases have been taken care of,
which leaves us with the following endsequent:
\[
  \text{mary}:\NP\prod(\text{likes}:\TV\prod\text{bill}:\NP)\;\fCenter\;?:\S
\]
Fortunately, proof search is decidable for this system, so we can
simply search the space of all possible proofs of this sequent. As it
turns out, the only proof is:
\begin{center}
  \vspace*{-1\baselineskip}
  \begin{pfbox}[0.8]
    \AXC{}\RightLabel{Ax}\UIC{$\text{mary}:\NP\fCenter\text{mary}:\NP$}
    \AXC{}\RightLabel{Ax}\UIC{$\text{likes}:\TV\fCenter\text{likes}:(\NP\impr\S)\impl\NP$}
    \AXC{}\RightLabel{Ax}\UIC{$\text{bill}:\NP\fCenter\text{bill}:\NP$}
    \RightLabel{$\impl$E}
    \BIC{$\text{likes}:\TV\prod\text{bill}:\NP\fCenter(\text{likes}\;\text{bill}):\NP\impr\S$}
    \RightLabel{$\impr$E}
    \BIC{$\text{mary}:\NP\prod(\text{likes}:\TV\prod\text{bill}:\NP)\fCenter((\text{likes}\;\text{bill})\;\text{mary}):\S$}
  \end{pfbox}
\end{center}
And so, by searching for a proof in our syntactic calculus (bottom-up)
and then adding in the term labelling (top-down) we derive a
function-argument structure for our sentence. Usually, we include
another step in this process, where we insert the lexical definitions
for the words. For the above example, these are:\footnote{%
  We use bold-face to distinguish between the variables associated
  with each word, and the meaning postulates we use in our semantics.
}
\[
  \begin{aligned}
    &\text{mary}  &&= \MARY\\
    &\text{john}  &&= \JOHN\\
    &\text{likes} &&= \lambda{y}.\lambda{x}.\LIKE(x,y)
  \end{aligned}
\]
After inserting these definitions, and $\beta$-reducing, we get:
\[
  \LIKE(\JOHN,\MARY)
\]
Because we are usually only interested in the resulting
function-argument structure and the associated semantics, for the
remainder of this thesis we will summarise the above translations as
follows:
\begin{center}
  \begin{pfbox}
    \AXC{}\RightLabel{Ax}\UIC{$\NP\fCenter\NP$}
    \AXC{}\RightLabel{Ax}\UIC{$\TV\fCenter(\NP\impr\S)\impl\NP$}
    \AXC{}\RightLabel{Ax}\UIC{$\NP\fCenter\NP$}
    \RightLabel{$\impl$E}
    \BIC{$\TV\prod\NP\fCenter(\NP\impr\S$}
    \RightLabel{$\impr$E}
    \BIC{$\NP\prod(\TV\prod\NP)\fCenter\S$}
  \end{pfbox}
  \vspace*{-1\baselineskip}
  \begin{gather*}
    \downmapsto
    \\
    ((\text{likes}\;\text{bill})\;\text{mary})
    \\
    \downmapsto
    \\
    \LIKE(\JOHN,\MARY)
  \end{gather*}
\end{center}


\subsection{Sequent calculus and proof search}
\label{sec:sequent-calculus-and-proof-search}

In the previous section, we glossed over the issue of proof
search. This is problematic, because the natural deduction formulation
of the syntactic calculus we presented in
\autoref{fig:nl-natural-deduction} is not especially suited to proof
search. \citeauthor{lambek1961} originally developed a sequent
calculus for NL, which \emph{does} have a practical procedure for
proof search. In \autoref{fig:nl-sequent-calculus} we present the
product-free version of \citepos{lambek1961} sequent calculus.

\begin{figure}[h]
  \begin{mdframed}
    \centering
    \begin{minipage}{0.6\linewidth}
      \begin{alignat*}{4}
        \text{Atom}     \;&α  &&\coloneqq \S\vsep\N\vsep\NP\vsep\INF\\
        \text{Type}     \;&A,B&&\coloneqq α\vsep A\impr B\vsep B\impl A\\
        \text{Structure}\;&Γ,Δ&&\coloneqq A\vsep Γ\prod Δ
      \end{alignat*}
    \end{minipage}%
    \begin{minipage}{0.4\linewidth}
      \begin{align*}
        \Box [Γ]&\mapsto Γ\\
        (Σ\prodl Δ)[Γ]&\mapsto (Σ[Γ]\prod Δ)\\
        (Δ\prodr Σ)[Γ]&\mapsto (Δ\prod Σ[Γ])
      \end{align*}
    \end{minipage}

    \vspace*{1\baselineskip}
    \begin{pfbox}
      \AXC{}
      \RightLabel{Ax}
      \UIC{$A\fCenter{A}$}
    \end{pfbox}
    \\[1\baselineskip]
    \begin{pfbox}
      \AXC{$\Sigma[B]\fCenter{C}$}
      \AXC{$\Gamma\fCenter{A}$}
      \RightLabel{L$\impr$}
      \BIC{$\Sigma[\Gamma\prod(A\impr{B})]\fCenter{C}$}
    \end{pfbox}
    \begin{pfbox}
      \AXC{$A\prod\Gamma\fCenter{B}$}
      \RightLabel{R$\impr$}
      \UIC{$\Gamma\fCenter{A\impr{B}}\fCenter{C}$}
    \end{pfbox}
    \\[1\baselineskip]
    \begin{pfbox}
      \AXC{$\Sigma[B]\fCenter{C}$}
      \AXC{$\Gamma\fCenter{A}$}
      \RightLabel{L$\impl$}
      \BIC{$\Sigma[(B\impl{A})\prod\Gamma]$}
    \end{pfbox}
    \begin{pfbox}
      \AXC{$\Gamma\prod{A}\fCenter{B}$}
      \RightLabel{R$\impr$}
      \UIC{$\Gamma\fCenter{B\impl{A}}$}
    \end{pfbox}
    \vspace*{1\baselineskip}

  \end{mdframed}
  \caption{NL \citep{lambek1961} in sequent calculus style.}
  \label{fig:nl-sequent-calculus}
\end{figure}


One important property of sequent calculus is the \emph{sub-formula}
property---the property that a derivation of a sequent uses only
proper sub-formulas of the formulas in that sequent.
As a direct consequence of this property, we generally get an
algorithm for proof search which is both easy to implement, and
complete. This algorithm is backward-chaining proof search: we
\begin{enumerate*}[label=(\arabic*)]
\item start with the desired endsequent;
\item branch, applying each rule that can be applied; and
\item repeat.
\end{enumerate*}
This algorithm is trivially complete, because we try all rules. It is
also trivially guaranteed to terminate, since a derivation can only
use sub-formulas of the formulas in the conclusion---at each
successive step, the number of available formulas becomes strictly
smaller, and so we will eventually run out of formulas.

The sequent calculus formulation is equivalent to the natural
deduction formulation from \autoref{fig:nl-natural-deduction}.
This is trivial to prove once you have a procedure for cut-elimination
\citep[see][p.\ 107]{moot2012}. Therefore, we are still able to
translate to \lamET, and obtain an interpretation. However, in the
next section we will discuss the alternative to this sequent calculus
that we will use, so we will forgo this exercise.


\section{Display calculus and focused proof search}
\label{sec:display-calculus}

In the previous section, we glossed over the issue of proof
search. However, the natural deduction formulation of the syntactic
calculus we presented in \autoref{fig:syntactic-calculus} is not
especially suited to proof search. \citet{lambek1958} originally
developed a sequent calculus for NL, which does have a practical proof
search algorithm.

In this section, we will develop a display calculus
\citep{belnap1982} for NL. We will start out by motivating our choice
for display calculus. Then we will present a display calculus for NL
based on work by \citet{bernardi2010,gore1998}.
In \autoref{sec:translation-to-lamET} we will relate our display
calculus back to the framework discussed in
\autoref{sec:introduction}, by defining a translation from our display
calculus back to \lamET.
And finally, in \autoref{sec:focusing-and-spurious-ambiguity}, we will
then conclude this section by discussing the problem of spurious
ambiguity, and address this problem by developing an extension to
display calculus, using polarities and focusing
\citep{girard1991,bastenhof2012}.

\subsection{Why use display calculus?}
\label{sec:why-use-display-calculus}
There are a few key advantages to using display calculus. First of
all, display calculus generalises sequent calculus. What this means is
that if something is a display calculus, it has all the properties
commonly associated with sequent calculus. Amongst others, display
calculus has the property that we are looking for: it has an easy to
implement, complete algorithm for proof search---backward-chaining
proof search.

However, display calculus is more than sequent calculus. One of the
main theorems regarding sequent calculus---Gentzen's `Hauptsatz'--is
the proof of cut-elimination. Whereas for sequent calculus, this
theorem has to be proved separately for each instance, display
calculus has a generic proof of cut-elimination, which holds whenever
the calculus obeys certain easy to check conditions.

One last reason is that display calculus is, due to the way in which
it is usually formulated, relatively easy to formalise.

Below we will discuss these arguments in favour of display calculus in
more detail.

\paragraph{Practical proof search procedure}
One important property of sequent calculus, is the \emph{sub-formula}
property---the property that a derivation of a sequent uses only
sub-formulas of the formulas in that sequent.
As a direct consequence of this property, we generally get an
algorithm for proof search which is both easy to implement, and
complete. This algorithm is backward-chaining proof search: we
\begin{enumerate*}[label=(\arabic*)]
\item start with the desired sequent;
\item branch, applying each rule that can be applied; and
\item repeat.
\end{enumerate*}
This algorithm is trivially complete, because we try all rules. It is
also trivially guaranteed to terminate, since a derivation can only
use sub-formulas of the formulas in the conclusion---the number of
available formulas is strictly smaller than the number of formulas in
the end sequent, and so we will eventually run out of formulas.

While all display calculi have the sub-formula property, they do not
necessarily have the \emph{sub-structure} property---the property that
the derivation of a sequent can only use sub-structures of the
structure in that sequent. While the original formulation of NL
\citep{lambek1961} has this property, it makes sense not to require it
from all display calculi, since many logics depend crucially on rules
that do not have this property.
However, this does mean we will have to take special care that the
structural rules we introduce will not break our guarantee of
termination. In practice, this also means that we will have to adjust
the algorithm for proof search.

\todo{Rewrite this section to be about residuation in general, and
  about a change in the search algorithm for display calculus. And
  only reference \autoref{fig:display-calculus} as an example.}
As an example, take the residuation rules given in
\autoref{fig:display-calculus}: these rules do not have the
sub-structure property, and---since they can be applied either
way---they can form loops.
However, in this particular case, these loops are benign, as they are
guaranteed to always return to the \emph{same} sequent. We can
preserve our termination guarantee by adding loop-checking to our
proof search algorithm. We do this by
\begin{enumerate*}[label=(\arabic*)]
\item passing along a set of visited sequents;
\item stopping the proof search if we ever visit the same sequent
  twice; and
\item emptying out this set if we make progress---where progress means
  eliminating a connective.
\end{enumerate*}
This extension also preserves completeness, since any proof that has a
loop in it can be trivially rewritten to a proof without a loop by
cutting out the loop.

The problem then remains to avoid structural rules---or combinations
of structural rules---which can cause a growing loop, in which no
sequent is visited more than once. We will discuss this further in
\todo{Reference section on Barker's NL$_{QR}$.}

\paragraph{Generic proof of cut-elimination}
Another important property of display calculus is the generic proof of
cut-elimination.
A proof of cut-elimination means that every proof which uses the cut
rule can be rewritten to a proof that does not use the cut rule:
\begin{prooftree}
  \AXC{$Γ\fCenter A$}
  \AXC{$A\fCenter Δ$}
  \RightLabel{Cut}
  \BIC{$Γ\fCenter Δ$}
\end{prooftree}
This is important, amongst other reasons, because a logic has to admit
the cut rule by definition. However, if we were to include cut as an
explicit rule, we would no longer be able to use backward-chaining
proof search; the cut rule can always be applied, and introduces a
unknown formula $A$.

Another reason why cut is important is because it embodies a
linguistic intuition that many of us have: the idea that if you have a
sentence which contains a noun phrase---e.g. `a book' in ``Mary read
a book''---and we have some other phrase of which we known that it is
also a noun phrase---e.g. ``the tallest man''---then we should be
able to substitute that second noun phrase for the first, and the
result should still be a grammatical sentence---e.g. ``Mary read the
tallest man.''

It should be clear that it is always important for the cut rule to be
admissible. However, in practice, one often has to give a separate
proof of cut-elimination for every logic. The generic proof of
cut-elimination for display calculus, however, states that if a
calculus obeys certain conditions \citep[see][]{gore1998}, the cut
rule is admissible.
This makes it an invaluable tool for research. In this thesis, we will
discuss several extensions to the non-associative Lambek calculus.
Because we know that each of these extensions respects the rules of
display calculus, we can be sure that any combination of then will
have a proof of cut-elimination, without having to prove this even
once.

\paragraph{Easy to formalise}
One last property of display calculus that is useful in formalising
the calculus, is the fact that display calculus does not rely on the
mechanisms of contexts and plugging functions, as used in
\autoref{fig:explicit-semantic-calculus} and the usual sequent
calculus formulation of NL.
These mechanisms are sometimes touted for simplifying the presentation
of proofs on paper, and for decreasing the complexity of proof
search---the idea being that there are fewer rules to apply.

However, they greatly complicate formal meta-logical proofs using, for
instance, proof assistants such as Coq or Agda.
For some intuition as to why, note that using contexts generally
inserts an application of the plugging function `\plug' in the
\emph{conclusions} of inference rules. This means that, in order to
do, for instance, a proof by induction on the structure of the
sequent, one has a much harder time proving which rules can lead to
this sequent.
In dependently-typed programming, the equivalent is inserting
function applications in the return types of the constructors of
datatypes. In his implementation of verified binary search trees,
\citet{mcbride2014} notes that this is bad design, as it leads to an
increased proof burden.

To make matters worse, it is not trivial to see if these mechanisms
actually \emph{do} decrease the size of the proofs. Undoubtedly, there
are fewer rule applications, but the flipside of this is that each
rule application involving a context must now implicitly be decorated
with that context.
In a similar vein, it is hard to see whether these mechanisms reduce
amount of work to be done during proof search. While there are indeed
fewer rules, each of these rules can now be applied under a variety of
contexts.
This last point hints at another advantage of not using contexts: it
allows for the proof search algorithm to be truly trivial, as we can
say a rule applies if its conclusion can be unified with the current
proof obligation, and do not have to check all possible contexts under
which this unification could succeed.


\subsection{NL as a display calculus}
\label{sec:nl-as-a-display-calculus}

In \autoref{fig:display-calculus}, we present the display calculus
version of NL. It features the same atoms and types as in
\autoref{fig:syntactic-calculus}, but structures have been
expanded: there are now positive and negative structures, and
residuation rules to navigate those. These two work together to
guarantee the \emph{display property}---the property that any
sub-structure can be made the sole structure in either the antecedent
or the succedent, depending on its polarity. For instance, below we
use residuation to isolate the object NP on the left-hand side:
\begin{center}
  \begin{pfbox}
    \AXC{$\vdots$}\noLine
    \UIC{$\struct{\NP}\prod(\struct{\TV}\prod\struct{\underline{\NP}})\fCenter\struct{\S}$}
    \RightLabel{Res$\prod\impr$}
    \UIC{$\struct{\TV}\prod\struct{\underline{\NP}}\fCenter\struct{\NP}\impr\struct{\S}$}
    \RightLabel{Res$\prod\impl$}
    \UIC{$\struct{\underline{\NP}}\fCenter(\struct{\NP}\impr\struct{\S})\impl\struct{\TV}$}
  \end{pfbox}
\end{center}
In order for our calculus to be a valid display calculus, it obeys
eight simple conditions. Of these conditions, the only one that
involves any proof burden is the last---adapted from \citet{gore1998}:
\begin{quote}
  If there are inference rules $ρ_1$ and $ρ_2$ with respective
  conclusions $Γ\fCenter\struct{A}$ and $\struct{A}\fCenter Δ$
  and if {Cut} is applied to yield $Γ\fCenter Δ$ then, either
  $Γ\fCenter Δ$ is identical to $Γ\fCenter\struct{A}$ or to
  $\struct{A}\fCenter Δ$; or it is possible to pass from the premises
  of $ρ_1$ and $ρ_2$ to $Γ\fCenter Δ$ by means of inferences falling
  under {Cut} where the cut-formula is always a proper sub-formula of
  $A$.
\end{quote}
In other words, we have to show that---in our logic---we can rewrite a
cut on matching left and right rules to smaller cuts on proper
sub-formulas of the cut-formula. For L$\impr$ and R$\impr$, this is
done as follows:
\begin{center}
  \begin{pfbox}
    \AXC{$\vdots$}\noLine\UIC{$Π\fCenter\struct{A}\impr\struct{B}$}
    \RightLabel{R$\impr$}
    \UIC{$Π\fCenter\struct{A\impr B}$}
    \AXC{$\vdots$}\noLine\UIC{$Γ\fCenter\struct{A}$}
    \AXC{$\vdots$}\noLine\UIC{$\struct{B}\fCenter Δ$}
    \RightLabel{L$\impr$}
    \BIC{$\struct{A\impr B}\fCenter Γ\impr Δ$}
    \RightLabel{Cut}
    \BIC{$Π\fCenter Γ\impr Δ$}
  \end{pfbox}
  \\[1\baselineskip] $\Longrightarrow$ \\
  \begin{pfbox}
    \AXC{$\vdots$}\noLine\UIC{$Γ\fCenter\struct{A}$}
    \AXC{$\vdots$}\noLine\UIC{$Π\fCenter\struct{A}\impr\struct{B}$}
    \RightLabel{Res$\impr\prod$}
    \UIC{$\struct{A}\prodΠ\fCenter\struct{B}$}
    \AXC{$\vdots$}\noLine\UIC{$\struct{B}\fCenter Δ$}
    \RightLabel{Cut}
    \BIC{$\struct{A}\prod Π\fCenter Δ$}
    \RightLabel{Res$\prod\impl$}
    \UIC{$\struct{A}\fCenter Δ\impl Π$}
    \RightLabel{Cut}
    \BIC{$Γ\fCenter Δ\impl Π$}
    \RightLabel{Res$\impl\prod$}
    \UIC{$Γ\prod Π\fCenter Δ$}
    \RightLabel{Res$\prod\impr$}
    \UIC{$Π\fCenter Γ\impr Δ$}
  \end{pfbox}
\end{center}
And likewise for L$\impl$ and R$\impl$.

Another change that was made in this display calculus, is that the
axiom has been restricted to atoms. This does not mean our logic no
longer has the identity, since it is derivable by simple induction
over the structure of the formula. For instance, in the case of an
identity on $A\impr B$:
\begin{center}
  \begin{pfbox}
    \AXC{$\vdots$}\noLine\UIC{$\struct{A}\fCenter\struct{A}$}
    \AXC{$\vdots$}\noLine\UIC{$\struct{B}\fCenter\struct{B}$}
    \RightLabel{L$\impr$}
    \BIC{$\struct{A\impr B}\fCenter\struct{A}\impr\struct{B}$}
    \RightLabel{R$\impr$}
    \UIC{$\struct{A\impr B}\fCenter\struct{A\impr B}$}
  \end{pfbox}
\end{center}
Instead, the change was made to avoid spurious ambiguity. \emph{If}
the calculus were to have a full identity, then there would be
\emph{two} proofs of the identity over, for instance \IV:
\begin{center}
  \begin{pfbox}
    \AXC{}
    \RightLabel{Ax}
    \UIC{$\struct{\NP\impr\S}\fCenter\struct{\NP\impr\S}$}
  \end{pfbox}
  \begin{pfbox}
    \AXC{}\RightLabel{Ax}\UIC{$\struct{\NP}\fCenter\struct{\NP}$}
    \AXC{}\RightLabel{Ax}\UIC{$\struct{\S}\fCenter\struct{\S}$}
    \RightLabel{L$\impr$}
    \BIC{$\struct{\NP\impr\S}\fCenter\struct{\NP}\impr\struct{\S}$}
    \RightLabel{R$\impr$}
    \UIC{$\struct{\NP\impr\S}\fCenter\struct{\NP\impr\S}$}
  \end{pfbox}
\end{center}
This is problematic. When we search for proofs, ideally we only want
different proofs if then sentence has different meanings, but the two
proofs above do not appear to have a radically different structure. In
fact, when we derive the identity by induction, both expand to the
same proof. The problem of spurious ambiguity is further discussed in
\autoref{sec:focusing-and-spurious-ambiguity}.

\subsection{Translation to \lamET}
\label{sec:translation-to-lamET}
In the previous sections, we defined a display calculus which is
equivalent to our natural deduction formulation of NL from
\autoref{sec:introduction}. However, there is still one thing missing
from our new implementation: a translation to \lamET.
We could make the equivalence between display NL and natural deduction
explicit, and use our old translation to \lamET, but in the interest
of clarity, we choose to give a direct translation instead. This
translation is presented in
\autoref{fig:display-calculus-to-explicit-lamET}.

There is one problem in translating display NL to \lamET---or to
natural deduction NL, for that matter: the display structures are more
expressive. To solve this problem, we remedy the distinction between
positive and negative structures. We translate positive
structure to structures, but translate negative structures---which are
built out of implications---to formulas instead. This neatly ensures
that the ``single formula succedent'' restriction on intuitionistic
logic is obeyed.

Though it does occurs in neither the premises nor the conclusions of
the rules of display NL, positive structures \emph{can} also occur
under negative structures. We cannot simply translate these to
structures, as this would be a type error. Therefore, we define a
second translation function `$\trd$' which maps positive structures to
formulas. Because this translation maps the structural product `$\prod$'
to the product type `$\times$', we have to extend \lamET with support
for product types. This is done in \autoref{fig:extension-products}.

\vspace*{1\baselineskip}

Now that we once again have a complete type-logical grammar, let us
take a look at our running example:
\begin{center}
  \begin{pfbox}[0.8]
    \AXC{$\struct{\mary:\NP}
      \prod(\struct{\likes:\TV}
      \prod\struct{\bill:\NP})\;
      \fCenter\;?:\struct{\S}$}
  \end{pfbox}
\end{center}%
One valid derivation of the following sequent---keeping in mind that
TV is short for $(\NP\impr\S)\impl\NP$---is:
\begin{center}
  \begin{pfbox}
    \AXC{}\RightLabel{Ax}\UIC{$\struct{\NP}\fCenter\struct{\NP}$}
    \AXC{}\RightLabel{Ax}\UIC{$\struct{\S}\fCenter\struct{\S}$}
    \RightLabel{L$\impr$}
    \BIC{$\struct{\NP\impr\S}\fCenter\struct{\NP}\impr\struct{\S}$}
    \AXC{}\RightLabel{Ax}\UIC{$\struct{\NP}\fCenter\struct{\NP}$}
    \RightLabel{L$\impl$}
    \BIC{$\struct{(\NP\impr\S)\impl\NP}\fCenter(\struct{\NP}\impr\struct{\S})\impl\struct{\NP}$}
    \RightLabel{Res$\impl\prod$}
    \UIC{$\struct{(\NP\impr\S)\impl\NP}\prod\struct{\NP}\fCenter\struct{\NP}\impr\struct{\S}$}
    \RightLabel{Res$\impr\prod$}
    \UIC{$\struct{\NP}\prod(\struct{(\NP\impr\S)\impl\NP}\prod\struct{\NP})\fCenter\struct{\S}$}
  \end{pfbox}
\end{center}
Using the translation from
\autoref{fig:display-calculus-to-explicit-lamET}, this translates to
the following proof (and term) in \lamET:
\begin{center}
  \hspace*{-1.8em}
  \begin{pfbox}[0.8]
    \AXC{}\RightLabel{Ax}\UIC{$\t\fCenter\t$}
    \RightLabel{$\ra$I}
    \UIC{$\emptyset\fCenter\t\t$}
    \AXC{}\RightLabel{Ax}\UIC{$\e\t\fCenter\e\t$}
    \AXC{}\RightLabel{Ax}\UIC{$\e\fCenter\e$}
    \RightLabel{$\ra$E}
    \BIC{$\e\t\prod\e\fCenter\t$}
    \RightLabel{$\ra$E}
    \BIC{$\e\t\prod\e\fCenter\t$}
    \RightLabel{$\ra$I}
    \UIC{$\e\t\fCenter\e\t$}
    \RightLabel{$\ra$I}
    \UIC{$\emptyset\fCenter(\e\t)\e\t$}
    \AXC{}\RightLabel{Ax}\UIC{$\e\e\t\fCenter\e\e\t$}
    \AXC{}\RightLabel{Ax}\UIC{$\e\fCenter\e$}
    \RightLabel{$\ra$E}
    \BIC{$\e\e\t\prod\e\fCenter\e\t$}
    \RightLabel{$\ra$E}
    \BIC{$\e\e\t\prod\e\fCenter\e\t$}
    \RightLabel{$\ra$I}
    \UIC{$\e\e\t\fCenter\e\e\t$}
    \AXC{}\RightLabel{Ax}\UIC{$\e\fCenter\e$}
    \RightLabel{$\ra$E}
    \BIC{$\e\e\t\prod\e\fCenter\e\t$}
    \AXC{}\RightLabel{Ax}\UIC{$\e\fCenter\e$}
    \RightLabel{$\ra$E}
    \BIC{$(\e\e\t\prod\e)\prod\e\fCenter\t$}
    \RightLabel{Comm.}
    \UIC{$\e\prod(\e\e\t\prod\e)\fCenter\t$}
  \end{pfbox}
  \\[1\baselineskip]
  $(((\lambda x.(\lambda k.\lambda y.(\lambda z.z)\;(k\;y))\;(\likes\;x))\;\bill)\;\mary)$
\end{center}
Admittedly, this looks somewhat unwieldy, but it $\beta$-reduces
neatly to:
\begin{center}
  \begin{pfbox}
    \AXC{}\RightLabel{Ax}\UIC{$\e\e\t\fCenter\e\e\t$}
    \AXC{}\RightLabel{Ax}\UIC{$\e\fCenter\e$}
    \RightLabel{$\ra$E}
    \BIC{$\e\e\t\prod\e\fCenter\e\t$}
    \AXC{}\RightLabel{Ax}\UIC{$\e\fCenter\e$}
    \RightLabel{$\ra$E}
    \BIC{$(\e\e\t\prod\e)\prod\e\fCenter\t$}
    \RightLabel{Comm.}
    \UIC{$\e\prod(\e\e\t\prod\e)\fCenter\t$}
  \end{pfbox}
  \\[1\baselineskip]
  $((\likes\;\bill)\;\mary)$
\end{center}

\subsection{Focusing and spurious ambiguity}
\label{sec:focusing-and-spurious-ambiguity}

In \autoref{sec:nl-as-a-display-calculus}, we briefly touched upon
spurious ambiguity.

% \footnote{%
%   \citet{bastenhof2012} develops the techniques for focused proof
%   search in order to obtain an elegant CPS translation for the
%   Lambek-Grishin calculus \citep{moortgat2009}.
%   In this thesis, we will not use a CPS translation, and in fact argue
%   against it in \todo{WRITE THIS}. Because of this, our usage of the
%   techniques -- in particular, the assignment of polarities to atomic
%   types -- will be different.
% }.

\begin{figure}
  \begin{mdframed}
    \centering
    \begin{alignat*}{4}
      \text{Atom}&\;        &α   &\coloneqq \S\vsep\N\vsep\NP\vsep\PP\vsep\INF\\
      \text{Type}&\;        &A,B &\coloneqq α\vsep A\impr B\vsep B\impl A\\
      \text{Structure}&^+\; &Γ   &\coloneqq \cdot A\cdot\vsep Γ_1\prod Γ_2\\
      \text{Structure}&^-\; &Δ   &\coloneqq \cdot A\cdot\vsep Γ\imprΔ\vsep Δ\implΓ
    \end{alignat*}

    \begin{pfbox}
      \AXC{}
      \RightLabel{Ax}
      \UIC{$\struct{α}\fCenter\struct{α}$}
    \end{pfbox}
    \\[1\baselineskip]
    \begin{pfbox}
      \AXC{$Γ\fCenter\struct{A}$}
      \AXC{$\struct{B}\fCenter Δ$}
      \RightLabel{L$\impr$}
      \BIC{$\struct{A\impr B}\fCenter Γ\impr Δ$}
    \end{pfbox}
    \begin{pfbox}
      \AXC{$Γ\fCenter\struct{A}\impr\struct{B}$}
      \RightLabel{R$\impr$}
      \UIC{$Γ\fCenter\struct{A\impr B}$}
    \end{pfbox}
    \\[1\baselineskip]
    \begin{pfbox}
      \AXC{$Γ\fCenter\struct{A}$}
      \AXC{$\struct{B}\fCenter Δ$}
      \RightLabel{L$\impl$}
      \BIC{$\struct{B\impl A}\fCenter Δ\impl Γ$}
    \end{pfbox}
    \begin{pfbox}
      \AXC{$Γ\fCenter\struct{B}\impl\struct{A}$}
      \RightLabel{R$\impl$}
      \UIC{$Γ\fCenter\struct{B\impl A}$}
    \end{pfbox}
    \\[1\baselineskip]
    \begin{pfbox}
      \AXC{$Γ_2\fCenter Γ_1\impr Δ$}
      \doubleLine\RightLabel{Res$\impr\prod$}
      \UIC{$Γ_1\prod Γ_2\fCenter Δ$}
    \end{pfbox}
    \begin{pfbox}
      \AXC{$Γ_1\fCenter Δ\impl Γ_2$}
      \doubleLine\RightLabel{Res$\impl\prod$}
      \UIC{$Γ_1\prod Γ_2\fCenter Δ$}
    \end{pfbox}
    \vspace*{1\baselineskip}
  \end{mdframed}
  \caption{
    The syntactic calculus from~\autoref{fig:syntactic-calculus} as a
    display calculus.}%
  \label{fig:display-calculus}
\end{figure}
%
\begin{figure}
  \begin{mdframed}
    \centering
    \vspace*{1\baselineskip}
    Extension of semantic calculus from
    \autoref{fig:implicit-semantic-calculus}:
    \begin{alignat*}{4}
      \text{Type}\;&A,B&&\coloneqq\ldots\vsep A\times B\vsep\top\\
      \text{Term}\;&M,N&&\coloneqq\ldots\vsep (M, N)\vsep\case{M}{x}{y}{N}\vsep()
    \end{alignat*}

    \begin{pfbox}
      \AXC{$Γ\fCenter M : A$}
      \AXC{$Γ\fCenter N : B$}
      \RightLabel{$\times$I}
      \BIC{$Γ\fCenter (M, N) : A\times B$}
    \end{pfbox}
    \\[1\baselineskip]
    \begin{pfbox}
      \AXC{$Γ\fCenter M : A\times B$}
      \AXC{$Γ, x : A, y : B\fCenter N : C$}
      \RightLabel{$\times$E}
      \BIC{$Γ\fCenter \case{M}{x}{y}{N} : C$}
    \end{pfbox}
    \\[1\baselineskip]
    \begin{pfbox}
      \AXC{}\RightLabel{$\top$}\UIC{$Γ\fCenter () : \top$}
    \end{pfbox}
    \\[1\baselineskip]
    \hrulefill
    \\[1\baselineskip]
    Extension of semantic calculus from
    \autoref{fig:explicit-semantic-calculus}:
    \\[1\baselineskip]
    \begin{pfbox}
      \AXC{$Γ\fCenter A$}
      \AXC{$Δ\fCenter B$}
      \RightLabel{$\times$I}
      \BIC{$Γ\prod Δ\fCenter A\times B$}
    \end{pfbox}
    \\[1\baselineskip]
    \begin{pfbox}
      \AXC{$Γ\fCenter A\times B$}
      \AXC{$Δ\prod A\prod B\fCenter C$}
      \RightLabel{$\times$E}
      \BIC{$Γ\prod Δ\fCenter C$}
    \end{pfbox}
    \\[1\baselineskip]
    \begin{pfbox}
      \AXC{}\RightLabel{$\top$}\UIC{$\emptyset\fCenter \top$}
    \end{pfbox}
    \vspace*{1\baselineskip}
  \end{mdframed}
  \caption{An extension of the semantic calculi from
    \autoref{fig:implicit-semantic-calculus} and
    \autoref{fig:explicit-semantic-calculus}.}
  \label{fig:extension-products}
\end{figure}
%
%%% Local Variables:
%%% mode: latex
%%% TeX-master: t
%%% End:

%
\begin{landscape}
  \begin{figure}
  \begin{mdframed}
    \renewcommand{\arraystretch}{5}%
    \begin{tabular}{l c c c}
      \multirow{6}{*}{%
      \framebox{\(\!
      \begin{aligned}
        &\textit{Structures}\\
        &\tr[(\struct{A})]   &&\mapsto \tr[A]\\
        &\tr[(Γ_1\prod Γ_2)] &&\mapsto \tr[Γ_1]\prod\tr[Γ_2]\\
        \\
        &\tr[(\struct{A})]   &&\mapsto \tr[A]\\
        &\tr[(Δ\impl Γ)]     &&\mapsto \trd[Γ]\ra\tr[Δ]\\
        &\tr[(Γ\impr Δ)]     &&\mapsto \trd[Γ]\ra\tr[Δ]\\
        \\
        &\trd[(\struct{A})]   &&\mapsto \tr[A]\\
        &\trd[(Γ_1\prod Γ_2)] &&\mapsto \trd[Γ_1]\times\trd[Γ_2]\\
        \\
        &\textit{Sequents}\\
        &\tr[(Γ\fCenterΔ)]         &&\mapsto \tr[Γ]\fCenter\tr[Δ]
      \end{aligned}\)}}&
      \begin{pfbox}[0.9]
        \AXC{$Γ\fCenter\struct{A}$}
        \AXC{$\struct{B}\fCenter Δ$}
        \RightLabel{L$\impr$}
        \BIC{$\struct{A\impr B}\fCenter Γ\impr Δ$}
      \end{pfbox}
      &$\Longrightarrow$&
      \begin{pfbox}[0.9]
        \AXC{$\tr[B]\fCenter\tr[Δ]$}
        %\RightLabel{Weak.}
        %\UIC{$\tr[B]\prod\emptyset\fCenter\tr[B]\ra\tr[Δ]$}
        %\RightLabel{Comm.}
        %\UIC{$\emptyset\prod\tr[B]\fCenter\tr[Δ]$}
        \RightLabel{$\ra$I}
        \UIC{$\emptyset\fCenter\tr[B]\ra\tr[Δ]$}
        \AXC{}\RightLabel{Ax}\UIC{$\tr[A]\ra\tr[B]\fCenter\tr[A]\ra\tr[B]$}
        \AXC{$\tr[Γ]\fCenter\tr[A]$}
        \RightLabel{$\ra$E}
        \BIC{$\tr[A]\ra\tr[B]\prod\tr[Γ]\fCenter\tr[B]$}
        \RightLabel{$\ra$E}
        %\BIC{$\emptyset\prod(\tr[A]\ra\tr[B]\prod\tr[Γ])\fCenter\tr[Δ]$}
        %\RightLabel{Comm.}
        %\UIC{$(\tr[A]\ra\tr[B]\prod\tr[Γ])\prod\emptyset\fCenter\tr[Δ]$}
        %\RightLabel{$\emptyset$E}
        \BIC{$\tr[A]\ra\tr[B]\prod\tr[Γ]\fCenter\tr[Δ]$}
        \RightLabel{$\ra$I}
        \UIC{$\tr[A]\ra\tr[B]\fCenter\tr[Γ]\ra\tr[Δ]$}
      \end{pfbox}
      \\&
      \begin{pfbox}[0.9]
        \AXC{$Γ\fCenter\struct{A}$}
        \AXC{$\struct{B}\fCenter Δ$}
        \RightLabel{L$\impl$}
        \BIC{$\struct{B\impl A}\fCenter Δ\impl Γ$}
      \end{pfbox}
      &$\Longrightarrow$&
      \multicolumn{1}{c}{\text{(as above)}}
      \\&
      \begin{pfbox}[0.9]
        \AXC{$Γ_2\fCenter Γ_1\impr Δ$}
        \RightLabel{Res$\impr\prod$}
        \UIC{$Γ_1\prod Γ_2\fCenter Δ$}
      \end{pfbox}
      &$\Longrightarrow$&
      \begin{pfbox}[0.9]
        \AXC{$\tr[Γ_2]\fCenter\tr[Γ_1]\ra\tr[Δ]$}
        \AXC{}\RightLabel{Ax}\UIC{$\tr[Γ_1]\fCenter\tr[Γ_1]$}
        \RightLabel{$\ra$E}
        \BIC{$\tr[Γ_2]\prod\tr[Γ_1]\fCenter\tr[Δ]$}
        \RightLabel{Comm.}
        \UIC{$\tr[Γ_1]\prod\tr[Γ_2]\fCenter\tr[Δ]$}
      \end{pfbox}
      \\&
      \begin{pfbox}[0.9]
        \AXC{$Γ_1\prod Γ_2\fCenter Δ$}
        \RightLabel{Res$\prod\impr$}
        \UIC{$Γ_2\fCenter Γ_1\impr Δ$}
      \end{pfbox}
      &$\Longrightarrow$&
      \begin{pfbox}[0.9]
        \AXC{$\tr[Γ_1]\prod\tr[Γ_2]\fCenter\tr[Δ]$}
        \RightLabel{Comm.}
        \UIC{$\tr[Γ_2]\prod\tr[Γ_1]\fCenter\tr[Δ]$}
        \RightLabel{$\ra$I}
        \UIC{$\tr[Γ_2]\fCenter\tr[Γ_1]\ra\tr[Δ]$}
      \end{pfbox}
      \\&
      \begin{pfbox}[0.9]
        \AXC{$Γ_1\fCenter Δ\impl Γ_2$}
        \RightLabel{Res$\impl\prod$}
        \UIC{$Γ_1\prod Γ_2\fCenter Δ$}
      \end{pfbox}
      &$\Longrightarrow$&
      \begin{pfbox}[0.9]
        \AXC{$\tr[Γ_1]\fCenter\tr[Γ_2]\ra\tr[Δ]$}
        \AXC{}\RightLabel{Ax}\UIC{$\tr[Γ_2]\fCenter\tr[Γ_2]$}
        \RightLabel{$\ra$E}
        \BIC{$\tr[Γ_1]\prod\tr[Γ_2]\fCenter\tr[Δ]$}
      \end{pfbox}
      \\&
      \begin{pfbox}[0.9]
        \AXC{$Γ_1\prod Γ_2\fCenter Δ$}
        \RightLabel{Res$\prod\impl$}
        \UIC{$Γ_1\fCenter Δ\impl Γ_2$}
      \end{pfbox}
      &$\Longrightarrow$&
      \begin{pfbox}[0.9]
        \AXC{$\tr[Γ_1]\prod\tr[Γ_2]\fCenter\tr[Δ]$}
        \RightLabel{$\ra$I}
        \UIC{$\tr[Γ_1]\fCenter\tr[Γ_2]\ra\tr[Δ]$}
      \end{pfbox}
    \end{tabular}
    \vspace*{\baselineskip}
  \end{mdframed}
  \caption{Translation from display calculus to explicit \lamET.}
  \label{fig:display-calculus-to-explicit-lamET}
  \end{figure}
\end{landscape}
%
\begin{figure}
  \begin{mdframed}
    \centering
    \vspace*{1\baselineskip}
    \(\!
      \begin{aligned}
        &\text{Pol}(α)        &&\mapsto{-} \\
        &\text{Pol}(B\impl A) &&\mapsto{-}
      \end{aligned}
      \quad
      \begin{aligned}
        \\
        &\text{Pol}(A\impr B) &&\mapsto{-}
      \end{aligned}
    \)
    \\[1\baselineskip]
    \(\!
    \cancel{
      \AXC{}
      \RightLabel{Ax}
      \UIC{$\struct{α}\fCenter\struct{α}$}
      \DisplayProof
    }
    \)
    \\[1\baselineskip]
    \(\!
    \color{gray}
    \text{if}\ \text{Pol}(α) = {+}
    \left\lbrace
      \quad
      \begin{aligned}
        \\
        \AXC{}
        \RightLabel{Ax$^R$}
        \UIC{$\struct{α}\fCenter\focus{α}$}
        \DisplayProof
        \\[1\baselineskip]
      \end{aligned}
      \quad
      \normalcolor
      \middle\vert
      \normalcolor
      \quad
      \begin{aligned}
        \\
        \AXC{}
        \RightLabel{Ax$^L$}
        \UIC{$\focus{α}\fCenter\struct{α}$}
        \DisplayProof
        \\[1\baselineskip]
      \end{aligned}
      \quad
    \right\rbrace
    \normalcolor
    \text{if}\ \text{Pol}(α) = {-}
    \)
    \\[1\baselineskip]
    \(\!
    \text{if}\ \text{Pol}(A) = {+}
    \left\lbrace
      \quad
      \begin{aligned}
        \\
        \AXC{$Γ\fCenter\focus{A}$}
        \RightLabel{Foc$^R$}
        \UIC{$Γ\fCenter\struct{A}$}
        \DisplayProof
        \\[1\baselineskip]
        \AXC{$\struct{A}\fCenter Δ$}
        \RightLabel{Unf$^L$}
        \UIC{$\focus{A}\fCenterΔ$}
        \DisplayProof
        \\[1\baselineskip]
      \end{aligned}
      \quad
      \middle\vert
      \quad
      \begin{aligned}
        \\
        \AXC{$\focus{A}\fCenterΔ$}
        \RightLabel{Foc$^L$}
        \UIC{$\struct{A}\fCenter Δ$}
        \DisplayProof
        \\[1\baselineskip]
        \AXC{$Γ\fCenter\struct{A}$}
        \RightLabel{Unf$^R$}
        \UIC{$Γ\fCenter\focus{A}$}
        \DisplayProof
        \\[1\baselineskip]
      \end{aligned}
      \quad
    \right\rbrace
    \text{if}\ \text{Pol}(A) = {-}
    \)
    \\[1\baselineskip]
    \begin{pfbox}
      \AXC{$Γ\fCenter\focus{A}$}
      \AXC{$\focus{B}\fCenter Δ$}
      \RightLabel{L$\impr$}
      \BIC{$\focus{A\impr B}\fCenter Γ\impr Δ$}
    \end{pfbox}
    \begin{pfbox}
      \AXC{$Γ\fCenter\focus{A}$}
      \AXC{$\focus{B}\fCenter Δ$}
      \RightLabel{L$\impl$}
      \BIC{$\focus{B\impl A}\fCenter Δ\impl Γ$}
    \end{pfbox}
    \vspace*{1\baselineskip}
  \end{mdframed}
  \caption{Changes to the display calculus
    from~\autoref{fig:display-calculus}, implementing focusing.}
  \label{fig:focused-display-calculus}
\end{figure}
%
%%% Local Variables:
%%% mode: latex
%%% TeX-master: t
%%% End:

\begin{figure}[h]
  \begin{mdframed}
    \centering
    \[\text{Type}\;A,B\coloneqq\ldots\vsep A\& B\vsep A\oplus B\]
    \begin{pfbox}
      \AXC{$\struct{A}\fCenter Δ$}
      \RightLabel{L\&$_1$}
      \UIC{$\struct{A\& B}\fCenter Δ$}
    \end{pfbox}
    \begin{pfbox}
      \AXC{$\struct{B}\fCenter Δ$}
      \RightLabel{L\&$_2$}
      \UIC{$\struct{A\& B}\fCenter Δ$}
    \end{pfbox}
    \begin{pfbox}
      \AXC{$Γ\fCenter\struct{A}$}
      \AXC{$Γ\fCenter\struct{B}$}
      \RightLabel{R\&}
      \BIC{$Γ\fCenter\struct{A\& B}$}
    \end{pfbox}
    \\[1\baselineskip]
    \begin{pfbox}
      \AXC{$\struct{A}\fCenter Δ$}
      \AXC{$\struct{B}\fCenter Δ$}
      \RightLabel{L$\oplus$}
      \BIC{$\struct{A\oplus B}\fCenter Δ$}
    \end{pfbox}
    \begin{pfbox}
      \AXC{$Γ\fCenter\struct{A}$}
      \RightLabel{R$\oplus_1$}
      \UIC{$Γ\fCenter\struct{A\oplus B}$}
    \end{pfbox}
    \begin{pfbox}
      \AXC{$Γ\fCenter\struct{B}$}
      \RightLabel{R$\oplus_2$}
      \UIC{$Γ\fCenter\struct{A\oplus B}$}
    \end{pfbox}
    \\[1\baselineskip]
    \hrulefill
    \[
      \tr[(A\& B)] \mapsto \tr[A]\times\tr[B]
    \]
    \begin{pfblock}
      \AXC{$x:\struct{A}\fCenter M:Δ$}
      \RightLabel{L\&$_1$}
      \UIC{$z:\struct{A\& B}\fCenter \case{z}{x}{\_}{M}:Δ$}
    \end{pfblock}
    \begin{pfblock}
      \AXC{$y:\struct{B}\fCenter M:Δ$}
      \RightLabel{L\&$_2$}
      \UIC{$z:\struct{A\& B}\fCenter \case{z}{\_}{y}{M}:Δ$}
    \end{pfblock}
    \begin{pfblock}
      \AXC{$x:Γ\fCenter\struct{M:A}$}
      \AXC{$x:Γ\fCenter\struct{N:B}$}
      \RightLabel{R\&}
      \BIC{$x:Γ\fCenter\struct{(M,N):A\& B}$}
    \end{pfblock}
    \vspace*{0.5\baselineskip}
  \end{mdframed}
  \caption{
    Extension of calculus in \autoref{fig:nl-display-calculus} which supports ambiguity.}%
  \label{fig:extension-lexical-ambiguity}
\end{figure}

\begin{figure}[hb]
  \begin{mdframed}
    \centering
    \begin{minipage}{0.666\linewidth}
      \centering
      \begin{alignat*}{4}
        \text{Type}     &  \;&A,B&\coloneqq\ldots\vsep A\himpr B\vsep B\himpl A\vsep\q[A]\\
        \text{Structure}&^+\;&Γ  &\coloneqq\ldots\vsep Γ_1\hprod Γ_2\vsep\I\vsep\B\vsep\C\\
        \text{Structure}&^-\;&Δ  &\coloneqq\ldots\vsep Γ\himpr Δ\vsep Δ\himpl Γ
      \end{alignat*}
    \end{minipage}%
    \begin{minipage}{0.333\linewidth}
      \centering
      \begin{alignat*}{4}
        &\text{Pol}(A\himpr B) &&\mapsto{-}\\
        &\text{Pol}(B\himpl A) &&\mapsto{-}\\
        &\text{Pol}(\q[A])    &&\mapsto{+}
      \end{alignat*}
    \end{minipage}
    \\[1\baselineskip]
    (copy of rules for $\{\impr,\prod,\impl\}$ from
    \autoref{fig:display-calculus} for $\{\himpr,\hprod,\himpl\}$)
    \\[1\baselineskip]
    \begin{pfbox}
      \AXC{$\struct{A}\hprod\I\fCenter Δ$}
      \RightLabel{L\I}
      \UIC{$\struct{\q[A]}\fCenter Δ$}
    \end{pfbox}
    \begin{pfbox}
      \AXC{$Γ\fCenter\focus{B}$}
      \RightLabel{R\I}
      \UIC{$Γ\hprod\I\fCenter\focus{\q[B]}$}
    \end{pfbox}
    \begin{pfbox}
      \AXC{$Γ\fCenter Δ$}
      \RightLabel{$\I^-$}
      \UIC{$Γ\hprod\I\fCenter Δ$}
    \end{pfbox}
    \\[1\baselineskip]
    \begin{pfbox}
      \AXC{$Γ_1\prod(Γ_2\hprod Γ_3)\fCenter Δ$}
      \doubleLine\RightLabel{\B}
      \UIC{$Γ_2\hprod((\B\prod Γ_1)\prod Γ_3)\fCenter Δ$}
    \end{pfbox}
    \begin{pfbox}
      \AXC{$(Γ_1\hprod Γ_2)\prod Γ_3\fCenter Δ$}
      \doubleLine\RightLabel{\C}
      \UIC{$Γ_1\hprod((\C\prod Γ_2)\prod Γ_3)\fCenter Δ$}
    \end{pfbox}
    \\[1\baselineskip]
    \hrulefill
    \\[1\baselineskip]
    {
      \renewcommand{\arraystretch}{1.5}%
      \(\!
      \begin{array}{c c c}
        \multicolumn{3}{c}{\tr[({\q[A]})]\mapsto\tr[A]}\\
        \tr[\I]\mapsto\top      & \tr [\B]\mapsto\top     & \tr [\C]\mapsto\top\\
      \end{array}
      \)
    }
    \\[1\baselineskip]
    (copy of translations for $\{\impr,\prod,\impl\}$ from
    \autoref{fig:display-calculus-to-explicit-lamET} for
    $\{\himpr,\hprod,\himpl\}$)
    \\[1\baselineskip]
    \begin{pfbox}
      \AXC{$x:\struct{A}\hprod\I\fCenter M:Δ$}
      \RightLabel{L\I}
      \UIC{$y:\struct{\q[A]}\fCenter \sub{M}{(y,())}{x}:Δ$}
    \end{pfbox}
    \begin{pfbox}
      \AXC{$x:Γ\fCenter\focus{M:B}$}
      \RightLabel{R\I}
      \UIC{$y:Γ\hprod\I\fCenter\focus{\sub{M}{\fst{y}}{x}:\q[B]}$}
    \end{pfbox}
    \begin{pfbox}
      \AXC{$x:Γ\fCenter M:Δ$}
      \RightLabel{$\I^-$}
      \UIC{$y:Γ\hprod\I\fCenter \sub{M}{\fst{y}}{x}:Δ$}
    \end{pfbox}
    \\[1\baselineskip]
    (where $\fst{x}=\case{x}{y}{z}{y}$)
    \\[1\baselineskip]
    (\B\ and \C\ translate to various combinations of associativity,
    commutativity, $\emptyset$E and weakening)
    \\
    \vspace*{\baselineskip}
  \end{mdframed}
  \caption{
    Extension of calculus in \autoref{fig:display-calculus} which
    supports quantifier raising.}%
  \label{fig:extension-quantifier-raising}
\end{figure}

%%% Local Variables:
%%% mode: latex
%%% TeX-master: t
%%% End:

%\begin{figure}[hb]
  \begin{mdframed}
    \centering
    \begin{minipage}{0.666\linewidth}
      \centering
      \begin{alignat*}{4}
        \text{Type}     &  \;&A,B&\coloneqq\ldots\vsep\di A\vsep\sq A\\
        \text{Structure}&^+\;&Γ  &\coloneqq\ldots\vsep\langle Γ\rangle\\
        \text{Structure}&^-\;&Δ  &\coloneqq\ldots\vsep[Δ]
      \end{alignat*}
    \end{minipage}%
    \begin{minipage}{0.333\linewidth}
      \centering
      \begin{alignat*}{4}
        &\text{Pol}(\di A) &&\mapsto{+}\\
        &\text{Pol}(\sq B) &&\mapsto{-}\\
      \end{alignat*}
    \end{minipage}
    \\[1\baselineskip]
    \begin{pfbox}
      \AXC{$\langle\struct{A}\rangle\fCenter Δ$}
      \RightLabel{L$\di$}
      \UIC{$\struct{\di A}\fCenter Δ$}
    \end{pfbox}
    \begin{pfbox}
      \AXC{$Γ\fCenter\focus{B}$}
      \RightLabel{R$\di$}
      \UIC{$\langle Γ\rangle\fCenter\focus{\di B}$}
    \end{pfbox}
    \\[1\baselineskip]
    \begin{pfbox}
      \AXC{$\focus{A}\fCenter Δ$}
      \RightLabel{L$\di$}
      \UIC{$\focus{\sq A}\fCenter[Δ]$}
    \end{pfbox}
    \begin{pfbox}
      \AXC{$Γ\fCenter[\struct{B}]$}
      \RightLabel{R$\sq$}
      \UIC{$Γ\fCenter\struct{\sq B}$}
    \end{pfbox}
    \\[1\baselineskip]
    \begin{pfbox}
      \AXC{$Γ\fCenter[Δ]$}
      \doubleLine\RightLabel{Res$\sq\di$}
      \UIC{$\langle Γ\rangle\fCenter Δ$}
    \end{pfbox}
    \\[1\baselineskip]
    \hrulefill
    \\[1\baselineskip]
    {
      \renewcommand{\arraystretch}{1.5}%
      \(
      \begin{array}{c c c}
        \tr [\di A]             \mapsto\tr [A]&
        \tr [\langle Γ \rangle] \mapsto\tr [Γ]&
        \trd[\langle Γ \rangle] \mapsto\trd[Γ]\\
        \tr [\sq A]             \mapsto\tr [A]&
        \tr [{[}Δ{]}]           \mapsto\tr [Δ]\\
      \end{array}
      \)
    }
    \\[1\baselineskip]
    (all rules translate to the identity)
    \vspace*{1\baselineskip}
  \end{mdframed}
  \caption{
    Extension of calculus in \autoref{fig:extension-quantifier-raising}
    which supports scope islands.}%
  \label{fig:extension-scope-islands}
\end{figure}

%%% Local Variables:
%%% mode: latex
%%% TeX-master: t
%%% End:

%\begin{figure}[h]
  \begin{mdframed}
    \centering
    \begin{minipage}{0.66\linewidth}
      \begin{alignat*}{4}
        \text{Type}     &  \;&A,B&\coloneqq \ldots\vsep\diDn A\vsep\sqDn A\\
        \text{Structure}&^+\;&Γ  &\coloneqq \ldots\vsep\diDn Γ\\
        \text{Structure}&^-\;&Δ  &\coloneqq \ldots\vsep\sqDn Δ
      \end{alignat*}
    \end{minipage}%
    \begin{minipage}{0.33\linewidth}
      \begin{alignat*}{3}
        &A \downharpoonleft  B\;&&\coloneqq\;(\diDn\sqDn{A})\impr{B}\\
        &B \downharpoonright A\;&&\coloneqq\;{B}\impl(\diDn\sqDn{A})
      \end{alignat*}
    \end{minipage}
    \\[1\baselineskip]
    (copy of rules for $\{\di,\sq\}$ from
    \autoref{fig:extension-scope-islands} for $\{\diDn,\sqDn\}$)
    \\[1\baselineskip]
    \begin{pfbox}
      \AXC{$Γ_1\prod (Γ_2\prod\diDn Γ_3)\fCenter Δ$}
      \RightLabel{RR\diDn}
      \UIC{$(Γ_1\prod Γ_2)\prod\diDn Γ_3\fCenter Δ$}
    \end{pfbox}
    \begin{pfbox}
      \AXC{$(Γ_1\prod\diDn Γ_3)\prod Γ_2\fCenter Δ$}
      \RightLabel{LR\diDn}
      \UIC{$(Γ_1\prod Γ_2)\prod\diDn Γ_3\fCenter Δ$}
    \end{pfbox}
    \\[1\baselineskip]
    \begin{pfbox}
      \AXC{$(\diDn Γ_3\prod Γ_2)\prod Γ_1\fCenter Δ$}
      \RightLabel{LL\diDn}
      \UIC{$\diDn Γ_3\prod (Γ_2\prod Γ_1)\fCenter Δ$}
    \end{pfbox}
    \begin{pfbox}
      \AXC{$Γ_2\prod (\diDn Γ_3\prod Γ_1)\fCenter Δ$}
      \RightLabel{RL\diDn}
      \UIC{$\diDn Γ_3\prod (Γ_2\prod Γ_1)\fCenter Δ$}
    \end{pfbox}
    \\[1\baselineskip]
    \hrulefill
    \\[1\baselineskip]
    (copy of translations for $\{\di,\sq\}$ from
    \autoref{fig:extension-scope-islands} for $\{\diDn,\sqDn\}$)
    \\[1\baselineskip]
    ({RR\diDn}, {LR\diDn}, {LL\diDn} and {RL\diDn} translate to
    various combinations of associativity and commutativity)
    \\[1\baselineskip]
  \end{mdframed}
  \caption{Extension of calculus in \autoref{fig:nl-display-calculus} which supports infixation.}
  \label{fig:extension-infixation}
\end{figure}

%\begin{figure}[hb]
  \begin{mdframed}
    \centering
    \begin{minipage}{0.66\linewidth}
      \begin{alignat*}{4}
        \text{Type}     &  \;&A,B&\coloneqq \ldots\vsep\diUp A\vsep\sqUp A\\
        \text{Structure}&^+\;&Γ  &\coloneqq \ldots\vsep\diUp Γ\\
        \text{Structure}&^-\;&Δ  &\coloneqq \ldots\vsep\sqUp Δ
      \end{alignat*}
    \end{minipage}%
    \begin{minipage}{0.33\linewidth}
      \begin{alignat*}{3}
        &A \upharpoonleft  B\;&&\coloneqq\;\diUp\sqUp(A\impr B)\\
        &B \upharpoonright A\;&&\coloneqq\;\diUp\sqUp(B\impl A)
      \end{alignat*}
    \end{minipage}
    \\
    \vspace*{\baselineskip}%
    (copy of rules for $\{\di,\sq\}$ from
    \autoref{fig:extension-scope-islands} for $\{\diUp,\sqUp\}$)
    \\
    \vspace*{\baselineskip}%
    \begin{pfbox}
      \AXC{$(Γ_1\prodΓ_2)\prod\diUpΓ_3\fCenterΔ$}
      \RightLabel{RR\diUp}
      \UIC{$Γ_1\prod(Γ_2\prod\diUpΓ_3)\fCenterΔ$}
    \end{pfbox}
    \begin{pfbox}
      \AXC{$(Γ_1\prod\diUpΓ_3)\prodΓ_2\fCenterΔ$}
      \RightLabel{LR\diUp}
      \UIC{$Γ_1\prod(Γ_2\prod\diUpΓ_3)\fCenterΔ$}
    \end{pfbox}

    \vspace*{\baselineskip}%
    \begin{pfbox}
      \AXC{$\diUpΓ_1\prod(Γ_2\prodΓ_3)\fCenterΔ$}
      \RightLabel{LL\diUp}
      \UIC{$(\diUpΓ_1\prodΓ_2)\prodΓ_3\fCenterΔ$}
    \end{pfbox}
    \begin{pfbox}
      \AXC{$\diUpΓ_1\prod(Γ_2\prodΓ_3)\fCenterΔ$}
      \RightLabel{RL\diUp}
      \UIC{$Γ_2\prod(\diUpΓ_1\prodΓ_3)\fCenterΔ$}
    \end{pfbox}
    \\
    \vspace*{\baselineskip}
    \hrulefill
    \\
    \vspace*{\baselineskip}
    (copy of translations for $\{\di,\sq\}$ from
    \autoref{fig:extension-scope-islands} for $\{\diUp,\sqUp\}$)
    \\
    \vspace*{\baselineskip}
    ({RR\diUp}, {LR\diUp}, {LL\diUp} and {RL\diUp} translate to
    various combinations of associativity and commutativity)
    \\
    \vspace*{\baselineskip}
  \end{mdframed}
  \caption{Extension of calculus in \autoref{fig:display-calculus}
    which supports extraction.}
  \label{fig:extension-extraction}
\end{figure}
%
%%% Local Variables:
%%% mode: latex
%%% TeX-master: t
%%% End:


% - implicit semantic calculus;
% - explicit semantic calculus;
% - syntactic calculus;
% - display calculus;
% - compositionality principle;
% - problems with compositionality;
% - quantifier raising and scope ambiguity;
% - continuation monad & delimited continuations;
% - extension: lexical ambiguity;
% - extension: quantifier raising
%   * treatment of some & every;
%   * treatment of same & different;
%   * treatment of plurals;
% - extension: scope islands;

\section{Future work}

\paragraph*{Forward-Chaining Proof Search}
In \autoref{sec:what-is-type-logical-grammar} it was mentioned that
most research focuses on implementing what I call the `semantic
function' (i.e. interpreting). This is a good approach for research:
we can limit ourselves to sequent calculus, which has pleasant
properties, refer to the huge body of work on generative grammar to
inform our choice for sentence structure, and simply focus on making
these known structures derivable. However, in order to be feasible in
a practical system, one must also implement what I call the `syntactic
function' (i.e. parsing). One way of doing this is by switching to
forward-chaining proof search, i.e. by constructing all possible
sentences based on the given words, and filtering on those which are
both pronounceable and maintain the correct word-order.
\todo{Ask Michael for a reference.}

\begin{sidewaysfigure}
  \begin{mdframed}
  \begin{tabularx}{1.0\linewidth}{l c c r}
    \begin{pfbox}[0.7]
      \AXC{}\RightLabel{Ax}\UIC{$A\fCenter A$}
    \end{pfbox}
    &$\Longrightarrow$&
    \begin{pfbox}[0.7]
      \AXC{}\RightLabel{Ax}\UIC{$Π,A\fCenter A$}
      \RightLabel{$\ra$I}
      \UIC{$Π\fCenter A\ra A$}
    \end{pfbox}
    &
    \multirow{3}{*}{\fbox{\scriptsize\(\!
      \begin{aligned}
        &\tr[A]             &&\mapsto A                  \\
        &\tr[\emptyset]     &&\mapsto \top               \\
        &\tr[(Γ\prod Δ)]    &&\mapsto \tr[Γ]\times\tr[Δ] \\
        &\tr[(Γ\fCenter A)] &&\mapsto Π\fCenter\tr[Γ]\ra A
        \quad\text{\footnotesize(for arbitrary Π)}
      \end{aligned}
    \)}}\\

    \begin{pfbox}[0.7]
      \AXC{$Γ\prod A\fCenter B$}
      \RightLabel{$\ra${I}}
      \UIC{$Γ\fCenter A\ra B$}
    \end{pfbox}
    &$\Longrightarrow$&
    \begin{pfbox}[0.7]
      \AXC{$Π,\tr[Γ],A\fCenter\tr[Γ]\times A\ra B$}
      \AXC{}\RightLabel{Ax}\UIC{$Π,\tr[Γ],A\fCenter\tr[Γ]$}
      \AXC{}\RightLabel{Ax}\UIC{$Π,\tr[Γ],A\fCenter A$}
      \RightLabel{$\times$I}
      \BIC{$Π,\tr[Γ],A\fCenter\tr[Γ]\times A$}
      \RightLabel {$\ra$E}
      \BIC{$Π,\tr[Γ],A\fCenter B$}
      \RightLabel{$\ra$I}
      \UIC{$Π,\tr[Γ]\fCenter A\ra B$}
      \RightLabel{$\ra$I}
      \UIC{$Π\fCenter\tr[Γ]\ra A\ra B$}
    \end{pfbox}
    \\

    \begin{pfbox}[0.7]
      \AXC{$Γ\fCenter A\ra B$}
      \AXC{$Δ\fCenter A$}
      \RightLabel{$\ra${E}}
      \BIC{$Γ\prod Δ\fCenter B$}
    \end{pfbox}
    &$\Longrightarrow$&
    \\
    \multicolumn{4}{r}{%
    \begin{pfbox}[0.7]
      \AXC{}\RightLabel{Ax}\UIC{$Π,\tr[Γ]\times\tr[Δ]\fCenter\tr[Γ]\times\tr[Δ]$}
      \AXC{$Π,\tr[Γ]\times\tr[Δ],\tr[Γ],\tr[Δ]\fCenter \tr[Γ]\ra A\ra B$}
      \AXC{}\RightLabel{Ax}\UIC{$Π,\tr[Γ]\times\tr[Δ],\tr[Γ],\tr[Δ]\fCenter\tr[Γ]$}
      \RightLabel{$\ra$E}
      \BIC{$Π,\tr[Γ]\times\tr[Δ],\tr[Γ],\tr[Δ]\fCenter A\ra B$}
      \AXC{$Π,\tr[Γ]\times\tr[Δ],\tr[Γ],\tr[Δ]\fCenter \tr[Δ]\ra A$}
      \AXC{}\RightLabel{Ax}\UIC{$Π,\tr[Γ]\times\tr[Δ],\tr[Γ],\tr[Δ]\fCenter \tr[Δ]$}
      \RightLabel{$\ra$E}
      \BIC{$Π,\tr[Γ]\times\tr[Δ],\tr[Γ],\tr[Δ]\fCenter A$}
      \RightLabel{$\ra$E}
      \BIC{$Π,\tr[Γ]\times\tr[Δ],\tr[Γ],\tr[Δ]\fCenter B$}
      \RightLabel{$\times$E}
      \BIC{$Π,\tr[Γ]\times\tr[Δ]\fCenter B$}
      \RightLabel{$\ra$I}
      \UIC{$Π\fCenter\tr[Γ]\times\tr[Δ]\ra B$}
    \end{pfbox}
    }
    \\

    \begin{pfbox}[0.7]
      \AXC{$Γ\fCenter A$}
      \AXC{$Δ\fCenter B$}
      \RightLabel{$\times$I}
      \BIC{$Γ\prod Δ\fCenter A\times B$}
    \end{pfbox}
    &$\Longrightarrow$&
    \\
    \multicolumn{4}{r}{%
    \begin{pfbox}[0.7]
      \AXC{}\RightLabel{Ax}\UIC{$Π,\tr[Γ]\times\tr[Δ]\fCenter\tr[Γ]\times\tr[Δ]$}
      \AXC{$Π,\tr[Γ]\times\tr[Δ],\tr[Γ],\tr[Δ]\fCenter\tr[Γ]\ra A$}
      \AXC{}\RightLabel{Ax}\UIC{$Π,\tr[Γ]\times\tr[Δ],\tr[Γ],\tr[Δ]\fCenter\tr[Γ]$}
      \RightLabel{$\ra$E}
      \BIC{$Π,\tr[Γ]\times\tr[Δ],\tr[Γ],\tr[Δ]\fCenter A$}
      \AXC{$Π,\tr[Γ]\times\tr[Δ],\tr[Γ],\tr[Δ]\fCenter\tr[Δ]\ra B$}
      \AXC{}\RightLabel{Ax}\UIC{$Π,\tr[Γ]\times\tr[Δ],\tr[Γ],\tr[Δ]\fCenter\tr[Δ]$}
      \RightLabel{$\ra$E}
      \BIC{$Π,\tr[Γ]\times\tr[Δ],\tr[Γ],\tr[Δ]\fCenter B$}
      \RightLabel{$\times$I}
      \BIC{$Π,\tr[Γ]\times\tr[Δ],\tr[Γ],\tr[Δ]\fCenter A\times B$}
      \RightLabel{$\times$E}
      \BIC{$Π,\tr[Γ]\times\tr[Δ]\fCenter A\times B$}
      \RightLabel{$\ra$I}
      \UIC{$Π\fCenter\tr[Γ]\times\tr[Δ]\ra A\times B$}
    \end{pfbox}
    }
    \\

    \begin{pfbox}[0.7]
      \AXC{$Γ\fCenter A\times B$}
      \AXC{$Δ\prod A\prod B\fCenter C$}
      \RightLabel{$\times$E}
      \BIC{$Γ\prod Δ\fCenter C$}
    \end{pfbox}
    &$\Longrightarrow$&
    \\
    \multicolumn{4}{r}{%
    \begin{pfbox}[0.7]
      \AXC{}\RightLabel{Ax}\UIC{$Π,\tr[Γ]\times\tr[Δ]\fCenter\tr[Γ]\times\tr[Δ]$}
      \AXC{$Π,\tr[Γ]\times\tr[Δ],\tr[Γ],\tr[Δ]\fCenter\tr[Γ]\ra A\times B$}
      \AXC{}\RightLabel{Ax}\UIC{$Π,\tr[Γ]\times\tr[Δ],\tr[Γ],\tr[Δ]\fCenter\tr[Γ]$}
      \RightLabel{$\ra$E}
      \BIC{$Π,\tr[Γ]\times\tr[Δ],\tr[Γ],\tr[Δ]\fCenter A\times B$}
      \AXC{$Π,\tr[Γ]\times\tr[Δ],\tr[Γ],\tr[Δ],A,B\fCenter\tr[Δ]\ra C$}
      \AXC{}\RightLabel{Ax}\UIC{$Π,\tr[Γ]\times\tr[Δ],\tr[Γ],\tr[Δ],A,B\fCenter\tr[Δ]$}
      \RightLabel{$\ra$E}
      \BIC{$Π,\tr[Γ]\times\tr[Δ],\tr[Γ],\tr[Δ],A,B\fCenter C$}
      \RightLabel{$\times$E}
      \BIC{$Π,\tr[Γ]\times\tr[Δ],\tr[Γ],\tr[Δ]\fCenter C$}
      \RightLabel{$\times$E}
      \BIC{$Π,\tr[Γ]\times\tr[Δ]\fCenter C$}
      \RightLabel{$\ra$I}
      \UIC{$Π\fCenter\tr[Γ]\times\tr[Δ]\ra C$}
    \end{pfbox}
    }
  \end{tabularx}%
  \vspace*{\baselineskip}
  \end{mdframed}%
  \caption{Translation from explicit to implicit version of \lamET\ as
    used in the Haskell implementation.}%
  \label{fig:explicit-to-implicit}
\end{sidewaysfigure}%
%
%%% Local Variables:
%%% mode: latex
%%% TeX-master: t
%%% End:


\bibliographystyle{apalike}%
\bibliography{main}%

\end{document}
