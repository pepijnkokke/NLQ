\section{Display calculus and focused proof search}
\label{sec:display-calculus}

In the previous section, we glossed over the issue of proof
search. However, the natural deduction formulation of the syntactic
calculus we presented in \autoref{fig:syntactic-calculus} is not
especially suited to proof search. Lambek originally developed a
sequent calculus for NL, which does have a practical proof search
algorithm.

In this section, we will develop a display calculus
\citep{belnap1982} for NL. We will start out by motivating our choice
for display calculus. Then we will present a display calculus for NL
based on work by \citet{moortgat2012,gore1998}.
In \autoref{sec:translation-to-lamET} we will relate our display
calculus back to the framework discussed in
\autoref{sec:introduction}, by defining a translation from our display
calculus back to \lamET.
And finally, in \autoref{sec:focusing-and-spurious-ambiguity}, we will
then conclude this section by discussing the problem of spurious
ambiguity, and address this problem by developing an extension to
display calculus, using polarities and focusing
\citep{girard1991,bastenhof2011}.

\subsection{Why use display calculus?}
\label{sec:why-use-display-calculus}
There are a few key advantages to using display calculus. First of
all, display calculus generalises sequent calculus. What this means is
that if something is a display calculus, it has all the properties
commonly associated with sequent calculus. Amongst others, display
calculus has the property that we are looking for: it has an easy to
implement, complete algorithm for proof search.

However, display calculus is more than sequent calculus. One of the
main theorems regarding sequent calculus---Gentzen's `Hauptsatz'---is
the proof of cut-elimination. Whereas for sequent calculus, this
theorem has to be proved separately for each instance, display
calculus has a generic proof of cut-elimination, which holds whenever
the calculus obeys certain easy to check conditions.

One last reason is that display calculus is, due to the way in which
it is usually formulated, relatively easy to formalise.

Below we will discuss these arguments in favour of display calculus in
more detail.

\paragraph{Practical proof search procedure}
One important property of sequent calculus, is the \emph{sub-formula}
property---the property that a derivation of a sequent uses only
proper sub-formulas of the formulas in that sequent.
As a direct consequence of this property, we generally get an
algorithm for proof search which is both easy to implement, and
complete. This algorithm is backward-chaining proof search: we
\begin{enumerate*}[label=(\arabic*)]
\item start with the desired sequent;
\item branch, applying each rule that can be applied; and
\item repeat.
\end{enumerate*}
This algorithm is trivially complete, because we try all rules. It is
also trivially guaranteed to terminate, since a derivation can only
use sub-formulas of the formulas in the conclusion---the number of
available formulas is strictly smaller than the number of formulas in
the end sequent, and so we will eventually run out of formulas.

While all display calculi have the sub-formula property, they do not
necessarily have the \emph{sub-structure} property---the property that
the derivation of a sequent can only use proper sub-structures of the
structure in that sequent. While the original formulation of NL has
this property, it makes sense not to require it from all display
calculi, since many logics depend crucially on inference rules that do
not have this property.
However, this does mean we will have to take special care that the
structural rules we introduce will not break the guarantee of
termination.

Theoretically, in order to be able to use naive backwards-chaining
proof search, our structural rules have to have the sub-structure
property. In practice, this is unfeasible for display calculi.
As an example of why, have a look at the residuation rules---for
instance, those in \autoref{fig:display-calculus}. Such rules are
crucial to display calculus---we will discuss why in the following
section---but they do not have the sub-structure property. In fact,
since they can be applied either way around, it is easy to see how
they can cause problems with termination.
This non-termination, however, is benign; these rules can only cause
\emph{loops}---any non-termination caused by them is guaranteed to
return to the same sequent.
We can extend our proof search procedure with loop-checking to cope
with this. We do this by
\begin{enumerate*}[label=(\arabic*)]
\item passing along a set of visited sequents;
\item stopping the proof search if we ever visit the same sequent
  twice; and
\item emptying out this set if we make progress---where progress means
  eliminating a connective.
\end{enumerate*}
Such an extension also preserves completeness, since any proof that
has a loop in it can be trivially rewritten to a proof without a loop by
cutting out the loop.

The problem then remains to avoid structural rules---or combinations
thereof---which can cause a divergence in which \emph{no} sequent is
visited more than once. We will discuss this further in
\todo{Reference section on Barker's NL$_{QR}$.}

\paragraph{Generic proof of cut-elimination}
Another important property of display calculus is the generic proof of
cut-elimination.
A proof of cut-elimination means that every proof which uses the cut
rule can be rewritten to a proof that does not use the cut rule:
\begin{prooftree}
  \AXC{$Γ\fCenter A$}
  \AXC{$A\fCenter Δ$}
  \RightLabel{Cut}
  \BIC{$Γ\fCenter Δ$}
\end{prooftree}
This is important, amongst other reasons, because a logic has to admit
the cut rule by definition. However, if we were to include cut as an
explicit rule, we would no longer be able to use backward-chaining
proof search; the cut rule can always be applied, and introduces a
unknown formula $A$.

Another reason why cut is important is because it embodies a
linguistic intuition that many of us have: the idea that if you have a
sentence which contains a noun phrase---e.g.\ `a book' in ``Mary read
a book''---and we have some other phrase of which we known that it is
also a noun phrase---e.g.\ ``the tallest man''---then we should be
able to substitute that second noun phrase for the first, and the
result should still be a grammatical sentence---e.g.\ ``Mary read the
tallest man.''

It should be clear that it is always important for the cut rule to be
admissible. However, in practice, one often has to give a separate
proof of cut-elimination for every logic. The generic proof of
cut-elimination for display calculus, however, states that if a
calculus obeys certain conditions \citep[see][]{gore1998}, the cut
rule is admissible.
This makes it an invaluable tool for research. In this thesis, we will
discuss several extensions to the non-associative Lambek calculus.
Because we know that each of these extensions respects the rules of
display calculus, we can be sure that any combination of then will
have a proof of cut-elimination, without having to prove this even
once.

\paragraph{Easy to formalise}
One last property of display calculus that is useful in formalising
the calculus, is the fact that display calculus does not rely on the
mechanisms of contexts and plugging functions, as used in
\autoref{fig:explicit-semantic-calculus} and the usual sequent
calculus formulation of NL.
These mechanisms are sometimes touted for simplifying the presentation
of proofs on paper, and for decreasing the complexity of proof
search---the idea being that there are fewer rules to apply.

However, they greatly complicate formal meta-logical proofs using, for
instance, proof assistants such as Coq or Agda.
For some intuition as to why, note that using contexts generally
inserts an application of the plugging function `\plug' in the
\emph{conclusions} of inference rules. This means that, in order to
do, for instance, a proof by induction on the structure of the
sequent, one has a much harder time proving which rules can lead to
this sequent.
In dependently-typed programming, the equivalent is inserting
function applications in the return types of the constructors of
datatypes. In their implementation of verified binary search trees,
\citet{mcbride2014} notes that this is bad design, as it leads to an
increased proof burden.

To make matters worse, it is not trivial to see if these mechanisms
actually \emph{do} decrease the size of the proofs. Undoubtedly, there
are fewer rule applications, but the flipside of this is that each
rule application involving a context must now implicitly be decorated
with that context.
In a similar vein, it is hard to see whether these mechanisms reduce
amount of work to be done during proof search. While there are indeed
fewer rules, each of these rules can now be applied under a variety of
contexts.
This last point hints at another advantage of not using contexts: it
allows for the proof search algorithm to be truly trivial, as we can
say a rule applies if its conclusion can be unified with the current
proof obligation, and do not have to check all possible contexts under
which this unification could succeed.


\subsection{NL as a display calculus}
\label{sec:nl-as-a-display-calculus}

\begin{figure}
  \begin{mdframed}
    \centering
    \begin{alignat*}{4}
      \text{Atom}&\;        &α   &\coloneqq \S\vsep\N\vsep\NP\vsep\PP\vsep\INF\\
      \text{Type}&\;        &A,B &\coloneqq α\vsep A\impr B\vsep B\impl A\\
      \text{Structure}&^+\; &Γ   &\coloneqq \cdot A\cdot\vsep Γ_1\prod Γ_2\\
      \text{Structure}&^-\; &Δ   &\coloneqq \cdot A\cdot\vsep Γ\imprΔ\vsep Δ\implΓ
    \end{alignat*}

    \begin{pfbox}
      \AXC{}
      \RightLabel{Ax}
      \UIC{$\struct{α}\fCenter\struct{α}$}
    \end{pfbox}
    \\[1\baselineskip]
    \begin{pfbox}
      \AXC{$Γ\fCenter\struct{A}$}
      \AXC{$\struct{B}\fCenter Δ$}
      \RightLabel{L$\impr$}
      \BIC{$\struct{A\impr B}\fCenter Γ\impr Δ$}
    \end{pfbox}
    \begin{pfbox}
      \AXC{$Γ\fCenter\struct{A}\impr\struct{B}$}
      \RightLabel{R$\impr$}
      \UIC{$Γ\fCenter\struct{A\impr B}$}
    \end{pfbox}
    \\[1\baselineskip]
    \begin{pfbox}
      \AXC{$Γ\fCenter\struct{A}$}
      \AXC{$\struct{B}\fCenter Δ$}
      \RightLabel{L$\impl$}
      \BIC{$\struct{B\impl A}\fCenter Δ\impl Γ$}
    \end{pfbox}
    \begin{pfbox}
      \AXC{$Γ\fCenter\struct{B}\impl\struct{A}$}
      \RightLabel{R$\impl$}
      \UIC{$Γ\fCenter\struct{B\impl A}$}
    \end{pfbox}
    \\[1\baselineskip]
    \begin{pfbox}
      \AXC{$Γ_2\fCenter Γ_1\impr Δ$}
      \doubleLine\RightLabel{Res$\impr\prod$}
      \UIC{$Γ_1\prod Γ_2\fCenter Δ$}
    \end{pfbox}
    \begin{pfbox}
      \AXC{$Γ_1\fCenter Δ\impl Γ_2$}
      \doubleLine\RightLabel{Res$\impl\prod$}
      \UIC{$Γ_1\prod Γ_2\fCenter Δ$}
    \end{pfbox}
    \vspace*{1\baselineskip}
  \end{mdframed}
  \caption{
    The syntactic calculus from~\autoref{fig:syntactic-calculus} as a
    display calculus.}%
  \label{fig:display-calculus}
\end{figure}
%
\begin{figure}
  \begin{mdframed}
    \centering
    \vspace*{1\baselineskip}
    Extension of semantic calculus from
    \autoref{fig:implicit-semantic-calculus}:
    \begin{alignat*}{4}
      \text{Type}\;&A,B&&\coloneqq\ldots\vsep A\times B\vsep\top\\
      \text{Term}\;&M,N&&\coloneqq\ldots\vsep (M, N)\vsep\case{M}{x}{y}{N}\vsep()
    \end{alignat*}

    \begin{pfbox}
      \AXC{$Γ\fCenter M : A$}
      \AXC{$Γ\fCenter N : B$}
      \RightLabel{$\times$I}
      \BIC{$Γ\fCenter (M, N) : A\times B$}
    \end{pfbox}
    \\[1\baselineskip]
    \begin{pfbox}
      \AXC{$Γ\fCenter M : A\times B$}
      \AXC{$Γ, x : A, y : B\fCenter N : C$}
      \RightLabel{$\times$E}
      \BIC{$Γ\fCenter \case{M}{x}{y}{N} : C$}
    \end{pfbox}
    \\[1\baselineskip]
    \begin{pfbox}
      \AXC{}\RightLabel{$\top$}\UIC{$Γ\fCenter () : \top$}
    \end{pfbox}
    \\[1\baselineskip]
    \hrulefill
    \\[1\baselineskip]
    Extension of semantic calculus from
    \autoref{fig:explicit-semantic-calculus}:
    \\[1\baselineskip]
    \begin{pfbox}
      \AXC{$Γ\fCenter A$}
      \AXC{$Δ\fCenter B$}
      \RightLabel{$\times$I}
      \BIC{$Γ\prod Δ\fCenter A\times B$}
    \end{pfbox}
    \\[1\baselineskip]
    \begin{pfbox}
      \AXC{$Γ\fCenter A\times B$}
      \AXC{$Δ\prod A\prod B\fCenter C$}
      \RightLabel{$\times$E}
      \BIC{$Γ\prod Δ\fCenter C$}
    \end{pfbox}
    \\[1\baselineskip]
    \begin{pfbox}
      \AXC{}\RightLabel{$\top$}\UIC{$\emptyset\fCenter \top$}
    \end{pfbox}
    \vspace*{1\baselineskip}
  \end{mdframed}
  \caption{An extension of the semantic calculi from
    \autoref{fig:implicit-semantic-calculus} and
    \autoref{fig:explicit-semantic-calculus}.}
  \label{fig:extension-products}
\end{figure}
%
%%% Local Variables:
%%% mode: latex
%%% TeX-master: t
%%% End:

%
\begin{landscape}
  \begin{figure}
  \begin{mdframed}
    \renewcommand{\arraystretch}{5}%
    \begin{tabular}{l c c c}
      \multirow{6}{*}{%
      \framebox{\(\!
      \begin{aligned}
        &\textit{Structures}\\
        &\tr[(\struct{A})]   &&\mapsto \tr[A]\\
        &\tr[(Γ_1\prod Γ_2)] &&\mapsto \tr[Γ_1]\prod\tr[Γ_2]\\
        \\
        &\tr[(\struct{A})]   &&\mapsto \tr[A]\\
        &\tr[(Δ\impl Γ)]     &&\mapsto \trd[Γ]\ra\tr[Δ]\\
        &\tr[(Γ\impr Δ)]     &&\mapsto \trd[Γ]\ra\tr[Δ]\\
        \\
        &\trd[(\struct{A})]   &&\mapsto \tr[A]\\
        &\trd[(Γ_1\prod Γ_2)] &&\mapsto \trd[Γ_1]\times\trd[Γ_2]\\
        \\
        &\textit{Sequents}\\
        &\tr[(Γ\fCenterΔ)]         &&\mapsto \tr[Γ]\fCenter\tr[Δ]
      \end{aligned}\)}}&
      \begin{pfbox}[0.9]
        \AXC{$Γ\fCenter\struct{A}$}
        \AXC{$\struct{B}\fCenter Δ$}
        \RightLabel{L$\impr$}
        \BIC{$\struct{A\impr B}\fCenter Γ\impr Δ$}
      \end{pfbox}
      &$\Longrightarrow$&
      \begin{pfbox}[0.9]
        \AXC{$\tr[B]\fCenter\tr[Δ]$}
        %\RightLabel{Weak.}
        %\UIC{$\tr[B]\prod\emptyset\fCenter\tr[B]\ra\tr[Δ]$}
        %\RightLabel{Comm.}
        %\UIC{$\emptyset\prod\tr[B]\fCenter\tr[Δ]$}
        \RightLabel{$\ra$I}
        \UIC{$\emptyset\fCenter\tr[B]\ra\tr[Δ]$}
        \AXC{}\RightLabel{Ax}\UIC{$\tr[A]\ra\tr[B]\fCenter\tr[A]\ra\tr[B]$}
        \AXC{$\tr[Γ]\fCenter\tr[A]$}
        \RightLabel{$\ra$E}
        \BIC{$\tr[A]\ra\tr[B]\prod\tr[Γ]\fCenter\tr[B]$}
        \RightLabel{$\ra$E}
        %\BIC{$\emptyset\prod(\tr[A]\ra\tr[B]\prod\tr[Γ])\fCenter\tr[Δ]$}
        %\RightLabel{Comm.}
        %\UIC{$(\tr[A]\ra\tr[B]\prod\tr[Γ])\prod\emptyset\fCenter\tr[Δ]$}
        %\RightLabel{$\emptyset$E}
        \BIC{$\tr[A]\ra\tr[B]\prod\tr[Γ]\fCenter\tr[Δ]$}
        \RightLabel{$\ra$I}
        \UIC{$\tr[A]\ra\tr[B]\fCenter\tr[Γ]\ra\tr[Δ]$}
      \end{pfbox}
      \\&
      \begin{pfbox}[0.9]
        \AXC{$Γ\fCenter\struct{A}$}
        \AXC{$\struct{B}\fCenter Δ$}
        \RightLabel{L$\impl$}
        \BIC{$\struct{B\impl A}\fCenter Δ\impl Γ$}
      \end{pfbox}
      &$\Longrightarrow$&
      \multicolumn{1}{c}{\text{(as above)}}
      \\&
      \begin{pfbox}[0.9]
        \AXC{$Γ_2\fCenter Γ_1\impr Δ$}
        \RightLabel{Res$\impr\prod$}
        \UIC{$Γ_1\prod Γ_2\fCenter Δ$}
      \end{pfbox}
      &$\Longrightarrow$&
      \begin{pfbox}[0.9]
        \AXC{$\tr[Γ_2]\fCenter\tr[Γ_1]\ra\tr[Δ]$}
        \AXC{}\RightLabel{Ax}\UIC{$\tr[Γ_1]\fCenter\tr[Γ_1]$}
        \RightLabel{$\ra$E}
        \BIC{$\tr[Γ_2]\prod\tr[Γ_1]\fCenter\tr[Δ]$}
        \RightLabel{Comm.}
        \UIC{$\tr[Γ_1]\prod\tr[Γ_2]\fCenter\tr[Δ]$}
      \end{pfbox}
      \\&
      \begin{pfbox}[0.9]
        \AXC{$Γ_1\prod Γ_2\fCenter Δ$}
        \RightLabel{Res$\prod\impr$}
        \UIC{$Γ_2\fCenter Γ_1\impr Δ$}
      \end{pfbox}
      &$\Longrightarrow$&
      \begin{pfbox}[0.9]
        \AXC{$\tr[Γ_1]\prod\tr[Γ_2]\fCenter\tr[Δ]$}
        \RightLabel{Comm.}
        \UIC{$\tr[Γ_2]\prod\tr[Γ_1]\fCenter\tr[Δ]$}
        \RightLabel{$\ra$I}
        \UIC{$\tr[Γ_2]\fCenter\tr[Γ_1]\ra\tr[Δ]$}
      \end{pfbox}
      \\&
      \begin{pfbox}[0.9]
        \AXC{$Γ_1\fCenter Δ\impl Γ_2$}
        \RightLabel{Res$\impl\prod$}
        \UIC{$Γ_1\prod Γ_2\fCenter Δ$}
      \end{pfbox}
      &$\Longrightarrow$&
      \begin{pfbox}[0.9]
        \AXC{$\tr[Γ_1]\fCenter\tr[Γ_2]\ra\tr[Δ]$}
        \AXC{}\RightLabel{Ax}\UIC{$\tr[Γ_2]\fCenter\tr[Γ_2]$}
        \RightLabel{$\ra$E}
        \BIC{$\tr[Γ_1]\prod\tr[Γ_2]\fCenter\tr[Δ]$}
      \end{pfbox}
      \\&
      \begin{pfbox}[0.9]
        \AXC{$Γ_1\prod Γ_2\fCenter Δ$}
        \RightLabel{Res$\prod\impl$}
        \UIC{$Γ_1\fCenter Δ\impl Γ_2$}
      \end{pfbox}
      &$\Longrightarrow$&
      \begin{pfbox}[0.9]
        \AXC{$\tr[Γ_1]\prod\tr[Γ_2]\fCenter\tr[Δ]$}
        \RightLabel{$\ra$I}
        \UIC{$\tr[Γ_1]\fCenter\tr[Γ_2]\ra\tr[Δ]$}
      \end{pfbox}
    \end{tabular}
    \vspace*{\baselineskip}
  \end{mdframed}
  \caption{Translation from display calculus to explicit \lamET.}
  \label{fig:display-calculus-to-explicit-lamET}
  \end{figure}
\end{landscape}
%
\begin{figure}
  \begin{mdframed}
    \centering
    \vspace*{1\baselineskip}
    \(\!
      \begin{aligned}
        &\text{Pol}(α)        &&\mapsto{-} \\
        &\text{Pol}(B\impl A) &&\mapsto{-}
      \end{aligned}
      \quad
      \begin{aligned}
        \\
        &\text{Pol}(A\impr B) &&\mapsto{-}
      \end{aligned}
    \)
    \\[1\baselineskip]
    \(\!
    \cancel{
      \AXC{}
      \RightLabel{Ax}
      \UIC{$\struct{α}\fCenter\struct{α}$}
      \DisplayProof
    }
    \)
    \\[1\baselineskip]
    \(\!
    \color{gray}
    \text{if}\ \text{Pol}(α) = {+}
    \left\lbrace
      \quad
      \begin{aligned}
        \\
        \AXC{}
        \RightLabel{Ax$^R$}
        \UIC{$\struct{α}\fCenter\focus{α}$}
        \DisplayProof
        \\[1\baselineskip]
      \end{aligned}
      \quad
      \normalcolor
      \middle\vert
      \normalcolor
      \quad
      \begin{aligned}
        \\
        \AXC{}
        \RightLabel{Ax$^L$}
        \UIC{$\focus{α}\fCenter\struct{α}$}
        \DisplayProof
        \\[1\baselineskip]
      \end{aligned}
      \quad
    \right\rbrace
    \normalcolor
    \text{if}\ \text{Pol}(α) = {-}
    \)
    \\[1\baselineskip]
    \(\!
    \text{if}\ \text{Pol}(A) = {+}
    \left\lbrace
      \quad
      \begin{aligned}
        \\
        \AXC{$Γ\fCenter\focus{A}$}
        \RightLabel{Foc$^R$}
        \UIC{$Γ\fCenter\struct{A}$}
        \DisplayProof
        \\[1\baselineskip]
        \AXC{$\struct{A}\fCenter Δ$}
        \RightLabel{Unf$^L$}
        \UIC{$\focus{A}\fCenterΔ$}
        \DisplayProof
        \\[1\baselineskip]
      \end{aligned}
      \quad
      \middle\vert
      \quad
      \begin{aligned}
        \\
        \AXC{$\focus{A}\fCenterΔ$}
        \RightLabel{Foc$^L$}
        \UIC{$\struct{A}\fCenter Δ$}
        \DisplayProof
        \\[1\baselineskip]
        \AXC{$Γ\fCenter\struct{A}$}
        \RightLabel{Unf$^R$}
        \UIC{$Γ\fCenter\focus{A}$}
        \DisplayProof
        \\[1\baselineskip]
      \end{aligned}
      \quad
    \right\rbrace
    \text{if}\ \text{Pol}(A) = {-}
    \)
    \\[1\baselineskip]
    \begin{pfbox}
      \AXC{$Γ\fCenter\focus{A}$}
      \AXC{$\focus{B}\fCenter Δ$}
      \RightLabel{L$\impr$}
      \BIC{$\focus{A\impr B}\fCenter Γ\impr Δ$}
    \end{pfbox}
    \begin{pfbox}
      \AXC{$Γ\fCenter\focus{A}$}
      \AXC{$\focus{B}\fCenter Δ$}
      \RightLabel{L$\impl$}
      \BIC{$\focus{B\impl A}\fCenter Δ\impl Γ$}
    \end{pfbox}
    \vspace*{1\baselineskip}
  \end{mdframed}
  \caption{Changes to the display calculus
    from~\autoref{fig:display-calculus}, implementing focusing.}
  \label{fig:focused-display-calculus}
\end{figure}
%
%%% Local Variables:
%%% mode: latex
%%% TeX-master: t
%%% End:


We present the display calculus for NL in
\autoref{fig:display-calculus}. It features the same atoms and types
as in \autoref{fig:syntactic-calculus}, but structures have been
expanded: there are now positive and negative structures---with one
structural connective for each logical connective---and residuation
rules to navigate them. These two work together to guarantee the
\emph{display property}---the property that any sub-structure can be
made the sole structure in either the antecedent or the succedent,
depending on its polarity. For instance, below we use residuation to
isolate the object NP on the left-hand side:\footnote{
  Inverted applications of the residuation rules are marked by
  switching the operators---e.g. by writing Res$\prod\impr$ instead of
  Res$\impr\prod$.
}
\begin{center}
  \begin{pfbox}
    \AXC{$\vdots$}\noLine
    \UIC{$\struct{\NP}\prod(\struct{\TV}\prod\struct{\underline{\NP}})\fCenter\struct{\S}$}
    \RightLabel{Res$\prod\impr$}
    \UIC{$\struct{\TV}\prod\struct{\underline{\NP}}\fCenter\struct{\NP}\impr\struct{\S}$}
    \RightLabel{Res$\prod\impl$}
    \UIC{$\struct{\underline{\NP}}\fCenter(\struct{\NP}\impr\struct{\S})\impl\struct{\TV}$}
  \end{pfbox}
\end{center}

Another change that was made in this display calculus, is that the
axiom has been restricted to atoms. This does not mean our logic no
longer has the identity, since it is derivable by simple induction
over the structure of the formula. For instance, in the case of an
identity on $A\impr B$:
\begin{center}
  \begin{pfbox}
    \AXC{$\vdots$}\noLine\UIC{$\struct{A}\fCenter\struct{A}$}
    \AXC{$\vdots$}\noLine\UIC{$\struct{B}\fCenter\struct{B}$}
    \RightLabel{L$\impr$}
    \BIC{$\struct{A\impr B}\fCenter\struct{A}\impr\struct{B}$}
    \RightLabel{R$\impr$}
    \UIC{$\struct{A\impr B}\fCenter\struct{A\impr B}$}
  \end{pfbox}
\end{center}
Instead, the change was made to avoid spurious ambiguity. If the
calculus \emph{were} to have a full identity, then there would be, for
instance, \emph{two} proofs of the identity over \IV:
\begin{center}
  \begin{pfbox}
    \AXC{}
    \RightLabel{Ax}
    \UIC{$\struct{\NP\impr\S}\fCenter\struct{\NP\impr\S}$}
  \end{pfbox}
  \begin{pfbox}
    \AXC{}\RightLabel{Ax}\UIC{$\struct{\NP}\fCenter\struct{\NP}$}
    \AXC{}\RightLabel{Ax}\UIC{$\struct{\S}\fCenter\struct{\S}$}
    \RightLabel{L$\impr$}
    \BIC{$\struct{\NP\impr\S}\fCenter\struct{\NP}\impr\struct{\S}$}
    \RightLabel{R$\impr$}
    \UIC{$\struct{\NP\impr\S}\fCenter\struct{\NP\impr\S}$}
  \end{pfbox}
\end{center}
This is problematic. Generally, we only want to have two proofs for
the same sequent when that sequent is associated with an ambiguous
sentence. But the associated sentences---e.g.\ ``Mary left''---are not
at all ambiguous. In fact, when we use the derived identity described
above, the first proof expands to the second one. The problem of
spurious ambiguity is further discussed in
\autoref{sec:focusing-and-spurious-ambiguity}.

In order for our calculus to be a valid display calculus, it needs to
obey eight simple conditions. Of these conditions, the only one that
involves any proof burden is \textbf{C8}---adapted from
\citet{gore1998}:
\begin{quote}
  If there are inference rules $ρ_1$ and $ρ_2$ with respective
  conclusions $Γ\fCenter\struct{A}$ and $\struct{A}\fCenter Δ$
  and if {Cut} is applied to yield $Γ\fCenter Δ$ then, either
  $Γ\fCenter Δ$ is identical to $Γ\fCenter\struct{A}$ or to
  $\struct{A}\fCenter Δ$; or it is possible to pass from the premises
  of $ρ_1$ and $ρ_2$ to $Γ\fCenter Δ$ by means of inferences falling
  under {Cut} where the cut-formula is always a proper sub-formula of
  $A$.
\end{quote}
In other words, we have to show that we can rewrite cuts on matching
left and right rules to smaller cuts on proper sub-formulas of the
cut-formula. For L$\impr$ and R$\impr$, this is done as follows:
\begin{center}
  \begin{pfbox}
    \AXC{$\vdots$}\noLine\UIC{$Π\fCenter\struct{A}\impr\struct{B}$}
    \RightLabel{R$\impr$}
    \UIC{$Π\fCenter\struct{A\impr B}$}
    \AXC{$\vdots$}\noLine\UIC{$Γ\fCenter\struct{A}$}
    \AXC{$\vdots$}\noLine\UIC{$\struct{B}\fCenter Δ$}
    \RightLabel{L$\impr$}
    \BIC{$\struct{A\impr B}\fCenter Γ\impr Δ$}
    \RightLabel{Cut}
    \BIC{$Π\fCenter Γ\impr Δ$}
  \end{pfbox}
  \\[1\baselineskip] $\Longrightarrow$ \\
  \begin{pfbox}
    \AXC{$\vdots$}\noLine\UIC{$Γ\fCenter\struct{A}$}
    \AXC{$\vdots$}\noLine\UIC{$Π\fCenter\struct{A}\impr\struct{B}$}
    \RightLabel{Res$\impr\prod$}
    \UIC{$\struct{A}\prodΠ\fCenter\struct{B}$}
    \AXC{$\vdots$}\noLine\UIC{$\struct{B}\fCenter Δ$}
    \RightLabel{Cut}
    \BIC{$\struct{A}\prod Π\fCenter Δ$}
    \RightLabel{Res$\prod\impl$}
    \UIC{$\struct{A}\fCenter Δ\impl Π$}
    \RightLabel{Cut}
    \BIC{$Γ\fCenter Δ\impl Π$}
    \RightLabel{Res$\impl\prod$}
    \UIC{$Γ\prod Π\fCenter Δ$}
    \RightLabel{Res$\prod\impr$}
    \UIC{$Π\fCenter Γ\impr Δ$}
  \end{pfbox}
\end{center}
And likewise for L$\impl$ and R$\impl$.


\subsection{Terms for \lamET}
\label{sec:translation-to-lamET}


In the previous sections, we defined a display calculus which is
equivalent to our natural deduction formulation of NL from
\autoref{sec:introduction}. However, there is still one thing missing
from our new implementation: terms.

We could translate display NL to natural deduction NL, and use the
term labelling that is the result of that translation. However, in
later sections we will extend display NL to be more expressive, and we
do not want to be forced to update the natural deduction formulation
as well. In addition, the extra indirection would complicate matters
too much. We therefore choose to give a direct translation from
display NL to lambda calculus.

\begin{figure}
  \begin{mdframed}
    \centering
    \vspace*{1\baselineskip}
    Extension of semantic calculus from
    \autoref{fig:implicit-semantic-calculus}:
    \begin{alignat*}{4}
      \text{Type}\;&A,B&&\coloneqq\ldots\vsep A\times B\vsep\top\\
      \text{Term}\;&M,N&&\coloneqq\ldots\vsep (M, N)\vsep\case{M}{x}{y}{N}\vsep()
    \end{alignat*}

    \begin{pfbox}
      \AXC{$Γ\fCenter M : A$}
      \AXC{$Γ\fCenter N : B$}
      \RightLabel{$\times$I}
      \BIC{$Γ\fCenter (M, N) : A\times B$}
    \end{pfbox}
    \\[1\baselineskip]
    \begin{pfbox}
      \AXC{$Γ\fCenter M : A\times B$}
      \AXC{$Γ, x : A, y : B\fCenter N : C$}
      \RightLabel{$\times$E}
      \BIC{$Γ\fCenter \case{M}{x}{y}{N} : C$}
    \end{pfbox}
    \\[1\baselineskip]
    \begin{pfbox}
      \AXC{}\RightLabel{$\top$}\UIC{$Γ\fCenter () : \top$}
    \end{pfbox}
    \\[1\baselineskip]
    \hrulefill
    \\[1\baselineskip]
    Extension of semantic calculus from
    \autoref{fig:explicit-semantic-calculus}:
    \\[1\baselineskip]
    \begin{pfbox}
      \AXC{$Γ\fCenter A$}
      \AXC{$Δ\fCenter B$}
      \RightLabel{$\times$I}
      \BIC{$Γ\prod Δ\fCenter A\times B$}
    \end{pfbox}
    \\[1\baselineskip]
    \begin{pfbox}
      \AXC{$Γ\fCenter A\times B$}
      \AXC{$Δ\prod A\prod B\fCenter C$}
      \RightLabel{$\times$E}
      \BIC{$Γ\prod Δ\fCenter C$}
    \end{pfbox}
    \\[1\baselineskip]
    \begin{pfbox}
      \AXC{}\RightLabel{$\top$}\UIC{$\emptyset\fCenter \top$}
    \end{pfbox}
    \vspace*{1\baselineskip}
  \end{mdframed}
  \caption{An extension of the semantic calculi from
    \autoref{fig:implicit-semantic-calculus} and
    \autoref{fig:explicit-semantic-calculus}.}
  \label{fig:extension-products}
\end{figure}
%
%%% Local Variables:
%%% mode: latex
%%% TeX-master: t
%%% End:


There is one problem in translating display NL to the lambda
calculus: the structures for display NL are much more expressive. There
are structural connectives for implication, and since each logical
connective must have a structural equivalent, we will certainly add
more structural connectives in later sections. For this reason, we
choose to translate all structures to types. The one downside to this
is that we must translate the product `$\prod$' to the product type
`$\times$', and insert the necessary machinery to pack and unpack
these products.
In \autoref{fig:extension-products}, we extend our semantic calculus
with products. Anticipating future needs, we also extend it with
units.

\begin{figure}[hb]
  \begin{mdframed}
    \centering
    \begin{minipage}{0.333\linewidth}
      \begin{alignat*}{4}
        &\textit{Structures}\\
        &\tr[(\struct{A})] &&\mapsto \tr[A]\\
        &\tr[(Γ_1\prod Γ_2)] &&\mapsto \tr[Γ_1]\times\tr[Γ_2]\\
        \\
        &\tr[(\struct{A})] &&\mapsto \tr[A]\\
        &\tr[(Δ\impl Γ)] &&\mapsto \tr[Γ]\ra\tr[Δ]\\
        &\tr[(Γ\impr Δ)] &&\mapsto \tr[Γ]\ra\tr[Δ]\\
        \\
        &\textit{Sequents}\\
        &\tr[(Γ\fCenterΔ)] &&\mapsto \tr[Γ]\fCenter\tr[Δ]\\
      \end{alignat*}
    \end{minipage}%
    \begin{minipage}{0.666\linewidth}
      \centering
      \vspace*{1\baselineskip}
      \begin{prooftree}
        \AXC{$x:Γ\fCenter M:A$}
        \AXC{$y:B\fCenter N:Δ$}
        \RightLabel{L$\impr$}
        \BIC{$f:\struct{A\impr B}\fCenter\lambda x.\sub{N}{f\;M}{y}:Γ\impr Δ$}
      \end{prooftree}
      \begin{prooftree}
        \AXC{$y:B\fCenter N:Δ$}
        \AXC{$x:Γ\fCenter M:A$}
        \RightLabel{L$\impl$}
        \BIC{$f:\struct{B\impl A}\fCenter\lambda x.\sub{N}{f\;M}{y}:Δ\impl Γ$}
      \end{prooftree}
      \begin{prooftree}
        \AXC{$y:Γ_2\fCenter M:Γ_1\impr Δ$}
        \RightLabel{Res$\impr\prod$}
        \UIC{$z:Γ_1\prod Γ_2\fCenter\case{z}{x}{y}{M\;x}:Δ$}
      \end{prooftree}
      \begin{prooftree}
        \AXC{$x:Γ_1\fCenter M:Δ\impl Γ_2$}
        \RightLabel{Res$\impl\prod$}
        \UIC{$z:Γ_1\prod Γ_2\fCenter\case{z}{x}{y}{M\;y}:Δ$}
      \end{prooftree}
      \begin{prooftree}
        \AXC{$z:Γ_1\prod Γ_2\fCenter M:Δ$}
        \RightLabel{Res$\prod\impr$}
        \UIC{$y:Γ_2\fCenter\lambda x.\sub{M}{(x,y)}{z}:Γ_1\impr Δ$}
      \end{prooftree}
      \begin{prooftree}
        \AXC{$z:Γ_1\prod Γ_2\fCenter M:Δ$}
        \RightLabel{Res$\prod\impl$}
        \UIC{$x:Γ_1\fCenter\lambda y.\sub{M}{(x,y)}{z}:Δ\impl Γ_2$}
      \end{prooftree}
      \vspace*{1\baselineskip}
    \end{minipage}
  \end{mdframed}
  \caption{Term labelling for display NL.}
  \label{fig:display-calculus-to-explicit-lamET}
\end{figure}


The term labelling for display NL is presented in
\autoref{fig:display-calculus-to-explicit-lamET}. The lambda terms are
typed by the translations of the  formulas from display NL. As is
usual when translating sequent calculus to natural deduction, our term
labelling employs substitution, which was defined in
\autoref{sec:simple-type-logical-grammar}.

\vspace*{1\baselineskip}

Now that we once again have a complete type-logical grammar, let us
take a quick look at an example, ``Mary likes Bill'':
\[
  \struct{\mary:\NP}\prod(\struct{\likes:\TV}\prod\struct{\bill:\NP})\;\fCenter\;?:\struct{\S}
\]
If we search for proofs, using backward-chaining proof search, we find
the following proof:
\begin{center}
  \begin{pfbox}
    \AXC{}\RightLabel{Ax}\UIC{$\struct{\NP}\fCenter\struct{\NP}$}
    \AXC{}\RightLabel{Ax}\UIC{$\struct{\S}\fCenter\struct{\S}$}
    \RightLabel{L$\impr$}
    \BIC{$\struct{\NP\impr\S}\fCenter\struct{\NP}\impr\struct{\S}$}
    \AXC{}\RightLabel{Ax}\UIC{$\struct{\NP}\fCenter\struct{\NP}$}
    \RightLabel{L$\impl$}
    \BIC{$\struct{(\NP\impr\S)\impl\NP}\fCenter(\struct{\NP}\impr\struct{\S})\impl\struct{\NP}$}
    \RightLabel{Res$\impl\prod$}
    \UIC{$\struct{(\NP\impr\S)\impl\NP}\prod\struct{\NP}\fCenter\struct{\NP}\impr\struct{\S}$}
    \RightLabel{Res$\impr\prod$}
    \UIC{$\struct{\NP}\prod(\struct{(\NP\impr\S)\impl\NP}\prod\struct{\NP})\fCenter\struct{\S}$}
  \end{pfbox}
\end{center}
Applying the translation from \autoref{fig:display-calculus-to-explicit-lamET}
gives us the following lambda term:
\[
  s:\e\times(\e\e\t\times\e)\fCenter
  (\case{s}{\mary}{(\likes,\bill)}{(\likes\;\bill)\;\mary}):\t
\]
This lambda term takes the sentence structure apart, and computes the
meaning. If this is desirable, it is possible to do some
post-processing with the structuralisation lemma:
\[
  \AX$A\hphantom{)}\fCenter B$
  \RightLabel{St}
  \UI$\text{St}(A)\fCenter B$
  \DisplayProof
  \quad
  \textbf{where}
  \quad
  \begin{aligned}
    &\text{St}(A\times B) &&\mapsto\text{St}(A),\text{St}(B)\\
    &\text{St}(\top)      &&\mapsto\emptyset                \\
    &\text{St}(A)         &&\mapsto A
  \end{aligned}
\]
This would result---after $\beta$-normalisation---in the following
lambda term:
\[
  \mary:\NP,\likes:\TV,\bill:\NP\fCenter((\likes\;\bill)\;\mary):\t
\]
The lemma itself is fairly easy to derive by induction on the
antecedent. Using it has the advantage that the lambda term takes the
lexical definitions---the values for \mary, \likes\ and \bill---from a
\emph{linear} structure, instead of from a nested tuple.

\subsection{Focusing and spurious ambiguity}
\label{sec:focusing-and-spurious-ambiguity}

In \autoref{sec:nl-as-a-display-calculus}, we briefly touched upon
spurious ambiguity. We investigated the spurious ambiguity inherent in
the axiom. In display NL as it stands, there is another kind of
spurious ambiguity that display NL shares with the sequent
calculus. In the case where there are two logical implications, not
nested under one another, there is an ambiguity in which one to
resolve first. We find this ambiguity, for instance, in the proofs
associated with ``Mary saw the fox.'':
\begin{pfblock}
  \AXC{}\RightLabel{Ax}\UIC{$\struct\N \fCenter\struct\N$}
  \AXC{}\RightLabel{Ax}\UIC{$\struct\NP\fCenter\struct\NP$}
  \RightLabel{L$\impl$}
  \BIC{$\struct{\NP\impl\N}\fCenter\struct\NP\impl\struct\N$}
  \RightLabel{Res$\impl\prod$}
  \UIC{$\struct{\NP\impl\N}\prod\struct\N\fCenter\struct\NP$}
  \AXC{}\RightLabel{Ax}\UIC{$\struct\NP\fCenter\struct\NP$}
  \AXC{}\RightLabel{Ax}\UIC{$\struct\S \fCenter\struct\S$}
  \RightLabel{L$\impr$}
  \BIC{$\struct{\NP\impr\S}\fCenter\struct\NP\impl\struct\S$}
  \RightLabel{L$\impl$}
  \BIC{$\struct{(\NP\impr\S)\impl\NP}
    \fCenter(\struct\NP\impr\struct\S)\impl(\struct{\NP\impl\N}\prod\struct\N)$}
  \RightLabel{Res$\impl\prod$}
  \UIC{$\struct{(\NP\impr\S)\impl\NP}\prod\struct{\NP\impl\N}\prod\struct\N
    \fCenter\struct\NP\impr\struct\S$}
  \RightLabel{Res$\impr\prod$}
  \UIC{$\struct\NP\prod\struct{(\NP\impr\S)\impl\NP}
    \prod\struct{\NP\impl\N}\prod\struct\N\fCenter\struct\S$}
\end{pfblock}%
\begin{pfblock}
  \AXC{}\RightLabel{Ax}\UIC{$\struct\N \fCenter\struct\N$}
  \AXC{}\RightLabel{Ax}\UIC{$\struct\NP\fCenter\struct\NP$}
  \AXC{}\RightLabel{Ax}\UIC{$\struct\NP\fCenter\struct\NP$}
  \AXC{}\RightLabel{Ax}\UIC{$\struct\S \fCenter\struct\S$}
  \RightLabel{L$\impr$}
  \BIC{$\struct{\NP\impr\S}\fCenter\struct\NP\impr\struct\S$}
  \RightLabel{L$\impl$}
  \BIC{$\struct{(\NP\impr\S)\impl\NP}
    \fCenter(\struct\NP\impr\struct\S)\impl\struct\NP$}
  \RightLabel{Res$\prod\impl$}
  \UIC{$\struct{(\NP\impr\S)\impl\NP}\prod\struct\NP
    \fCenter(\struct\NP\impr\struct\S)$}
  \RightLabel{Res$\impr\prod$}
  \UIC{$\struct\NP
    \fCenter\struct{(\NP\impr\S)\impl\NP}\impr(\struct\NP\impr\struct\S)$}
  \RightLabel{L$\impl$}
  \BIC{$\struct{\NP\impl\N}\fCenter(\struct{(\NP\impr\S)\impl\NP}
    \impr(\struct\NP\impr\struct\S))\impl\struct\N$}
  \RightLabel{Res$\impl\prod$}
  \UIC{$\struct{\NP\impl\N}\prod\struct\N
    \fCenter\struct{(\NP\impr\S)\impl\NP}\impr(\struct\NP\impr\struct\S)$}
  \RightLabel{Res$\impr\prod$}
  \UIC{$\struct{(\NP\impr\S)\impl\NP}\prod\struct{\NP\impl\N}\prod\struct\N
    \fCenter\struct\NP\impr\struct\S$}
  \RightLabel{Res$\impr\prod$}
  \UIC{$\struct\NP\prod\struct{(\NP\impr\S)\impl\NP}
    \prod\struct{\NP\impl\N}\prod\struct\N\fCenter\struct\S$}
\end{pfblock}
Although these two proofs are labelled with two distinct lambda terms,
the two are $\beta$-equivalent. This is problematic, as we only want
ambiguity in our syntax when there is ambiguity in the meaning.

In order to solve this, we can employ focusing, as developed by
\citet{andreoli1992} for linear logic. This extension was first
employed for proof search in Lambek calculi by
\citet{bastenhof2011} and \citet{moortgat2012} for classical
non-associative Lambek calculus (CNL) and the Lambek-Grishin calculus
(LG), respectively. Both of these calculi are extensions of NL, so
applying the techniques developed in their work to display NL is
trivial.

\begin{figure}
  \begin{mdframed}
    \centering
    \vspace*{1\baselineskip}
    \(\!
      \begin{aligned}
        &\text{Pol}(α)        &&\mapsto{-} \\
        &\text{Pol}(B\impl A) &&\mapsto{-}
      \end{aligned}
      \quad
      \begin{aligned}
        \\
        &\text{Pol}(A\impr B) &&\mapsto{-}
      \end{aligned}
    \)
    \\[1\baselineskip]
    \(\!
    \cancel{
      \AXC{}
      \RightLabel{Ax}
      \UIC{$\struct{α}\fCenter\struct{α}$}
      \DisplayProof
    }
    \)
    \\[1\baselineskip]
    \(\!
    \color{gray}
    \text{if}\ \text{Pol}(α) = {+}
    \left\lbrace
      \quad
      \begin{aligned}
        \\
        \AXC{}
        \RightLabel{Ax$^R$}
        \UIC{$\struct{α}\fCenter\focus{α}$}
        \DisplayProof
        \\[1\baselineskip]
      \end{aligned}
      \quad
      \normalcolor
      \middle\vert
      \normalcolor
      \quad
      \begin{aligned}
        \\
        \AXC{}
        \RightLabel{Ax$^L$}
        \UIC{$\focus{α}\fCenter\struct{α}$}
        \DisplayProof
        \\[1\baselineskip]
      \end{aligned}
      \quad
    \right\rbrace
    \normalcolor
    \text{if}\ \text{Pol}(α) = {-}
    \)
    \\[1\baselineskip]
    \(\!
    \text{if}\ \text{Pol}(A) = {+}
    \left\lbrace
      \quad
      \begin{aligned}
        \\
        \AXC{$Γ\fCenter\focus{A}$}
        \RightLabel{Foc$^R$}
        \UIC{$Γ\fCenter\struct{A}$}
        \DisplayProof
        \\[1\baselineskip]
        \AXC{$\struct{A}\fCenter Δ$}
        \RightLabel{Unf$^L$}
        \UIC{$\focus{A}\fCenterΔ$}
        \DisplayProof
        \\[1\baselineskip]
      \end{aligned}
      \quad
      \middle\vert
      \quad
      \begin{aligned}
        \\
        \AXC{$\focus{A}\fCenterΔ$}
        \RightLabel{Foc$^L$}
        \UIC{$\struct{A}\fCenter Δ$}
        \DisplayProof
        \\[1\baselineskip]
        \AXC{$Γ\fCenter\struct{A}$}
        \RightLabel{Unf$^R$}
        \UIC{$Γ\fCenter\focus{A}$}
        \DisplayProof
        \\[1\baselineskip]
      \end{aligned}
      \quad
    \right\rbrace
    \text{if}\ \text{Pol}(A) = {-}
    \)
    \\[1\baselineskip]
    \begin{pfbox}
      \AXC{$Γ\fCenter\focus{A}$}
      \AXC{$\focus{B}\fCenter Δ$}
      \RightLabel{L$\impr$}
      \BIC{$\focus{A\impr B}\fCenter Γ\impr Δ$}
    \end{pfbox}
    \begin{pfbox}
      \AXC{$Γ\fCenter\focus{A}$}
      \AXC{$\focus{B}\fCenter Δ$}
      \RightLabel{L$\impl$}
      \BIC{$\focus{B\impl A}\fCenter Δ\impl Γ$}
    \end{pfbox}
    \vspace*{1\baselineskip}
  \end{mdframed}
  \caption{Changes to the display calculus
    from~\autoref{fig:display-calculus}, implementing focusing.}
  \label{fig:focused-display-calculus}
\end{figure}


The extension is described in \autoref{fig:focused-display-calculus}.
It is a system with three kinds of sequents: the original structural
sequent $Γ\fCenterΔ$, and two focused sequents $\focus{A}\fCenterΔ$
and $Γ\fCenter\focus{B}$. There are four new rules which communicate
between these sequent kinds: left and right focusing and unfocusing.
In addition, the axiom is split into a left- and a right-focused
axiom. The crucial point is that all formulas are assigned a
polarity---based on the polarity of the main connective in case of a
complex formula, and arbitrarily assigned in case of atomic
formulas---and the focusing and unfocusing rules and axioms can only
be applied to principal formulas of a certain polarity.

Using the system from \autoref{fig:focused-display-calculus}, we see
that we can easily extend the first proof for ``Mary saw the fox.'' by
inserting the focusing and unfocusing rules in the right places.
The second proof, however, is rejected by our new system, as it
contains a polarity error.
\begin{pfblock}
  \AXC{}
  \RightLabel{Ax $^L$}\UIC{$\focus \N \fCenter\struct\N$}
  \RightLabel{Foc$^L$}\UIC{$\struct\N \fCenter\struct\N$}
  \RightLabel{Unf$^R$}\UIC{$\struct\N \fCenter\focus \N$}
  \AXC{}
  \RightLabel{Ax $^L$}\UIC{$\focus \NP\fCenter\struct\NP$}
  \RightLabel{Foc$^L$}\UIC{$\struct\NP\fCenter\struct\NP$}
  \RightLabel{Unf$^R$}\UIC{$\struct\NP\fCenter\focus \NP$}
  \AXC{}
  \RightLabel{Ax $^L$}\UIC{$\focus \NP\fCenter\struct\NP$}
  \RightLabel{Foc$^L$}\UIC{$\struct\NP\fCenter\struct\NP$}
  \RightLabel{Unf$^R$}\UIC{$\struct\NP\fCenter\focus \NP$}

  \AXC{}
  \RightLabel{Ax $^L$}\UIC{$\focus \S \fCenter\struct\S$}

  \RightLabel{L$\impr$}
  \BIC{$\focus{\NP\impr\S}\fCenter\struct\NP\impr\struct\S$}
  \RightLabel{L$\impl$}
  \BIC{$\focus{(\NP\impr\S)\impl\NP}
    \fCenter(\struct\NP\impr\struct\S)\impl\struct\NP$}
  \RightLabel{Foc$^L$}
  \UIC{$\struct{(\NP\impr\S)\impl\NP}
    \fCenter(\struct\NP\impr\struct\S)\impl\struct\NP$}
  \RightLabel{Res$\prod\impl$}
  \UIC{$\struct{(\NP\impr\S)\impl\NP}\prod\struct\NP
    \fCenter(\struct\NP\impr\struct\S)$}
  \RightLabel{Res$\impr\prod$}
  \UIC{$\struct\NP
    \fCenter\struct{(\NP\impr\S)\impl\NP}\impr(\struct\NP\impr\struct\S)$}
  \RightLabel{Unf$^L$ $\la$ wrong}
  \UIC{$\focus\NP
    \fCenter\struct{(\NP\impr\S)\impl\NP}\impr(\struct\NP\impr\struct\S)$}
  \RightLabel{L$\impl$}
  \BIC{$\focus{\NP\impl\N}\fCenter(\struct{(\NP\impr\S)\impl\NP}
    \impr(\struct\NP\impr\struct\S))\impl\struct\N$}
  \RightLabel{Foc$^L$}
  \UIC{$\struct{\NP\impl\N}\fCenter(\struct{(\NP\impr\S)\impl\NP}
    \impr(\struct\NP\impr\struct\S))\impl\struct\N$}
  \RightLabel{Res$\impl\prod$}
  \UIC{$\struct{\NP\impl\N}\prod\struct\N
    \fCenter\struct{(\NP\impr\S)\impl\NP}\impr(\struct\NP\impr\struct\S)$}
  \RightLabel{Res$\impr\prod$}
  \UIC{$\struct{(\NP\impr\S)\impl\NP}\prod\struct{\NP\impl\N}\prod\struct\N
    \fCenter\struct\NP\impr\struct\S$}
  \RightLabel{Res$\impr\prod$}
  \UIC{$\struct\NP\prod\struct{(\NP\impr\S)\impl\NP}
    \prod\struct{\NP\impl\N}\prod\struct\N\fCenter\struct\S$}
\end{pfblock}
Had we chosen differently, and assigned \NP\ a positive polarity, this
proof would be valid. However, in that case the \emph{first} proof
would have contained a polarity error. We leave it as an exercise for
the reader to find this error.

It should be mentioned at this point that \citet{bastenhof2011}
introduces focusing and polarisation in order to give a
continuation-passing style (CPS) translation for CNL, based on work by
\citet{girard1991}. By selecting the right polarities for atomic
formulas---positive for inputs such as \N\ and \NP, negative for
outputs such as \S---the logic can be tuned to allow ambiguous
derivations exactly where scope ambiguity may arise in natural
language. However, we employ different mechanisms to deal with scope
ambiguity, which we will discuss in \todo{NL$_{QR}$}. For this reason,
we assign all atomic formulas negative polarity, and use a direct
translation. This translation is identical to the term labelling given
in \autoref{fig:display-calculus-to-explicit-lamET}; the left and
right axioms translate no different than the old axiom, and the
focusing and unfocusing rules translate to the identity.

Looking at the proof above, one might wonder why we bother restricting
the polarity for axioms at all, since obviously the focusing and
unfocusing rules allow us to derive left- or right-focused sequents
for any atom, regardless of polarity. Once again, this is to avoid
spurious ambiguity: if we did not restrict the axioms, there would be
two ways to derive either axiom. The above proof has 4 axioms, so---if
it were correct---that would result in $2^4$ proofs.

Extending display NL with focusing raises one problem: does this
extension still obey the conditions for display calculus? We assume
that cut has the following form:
\vspace{-1\baselineskip}
\begin{pfblock}
  \AXC{$Γ\fCenter\focus{A}$}
  \AXC{$\focus{A}\fCenterΔ$}
  \RightLabel{Cut}
  \BIC{$Γ\fCenterΔ$}
\end{pfblock}
However, when we try to extend the proof for \textbf{C8}, we find
that---keeping in mind that we may want to add new connectives or
change the polarities of the atomic formulas---there are two places in
the proof where we cannot guarantee the polarity of certain formulas:
\begin{center}
  \vbox{
    \begin{pfbox}
      \AXC{$\vdots$}\noLine\UIC{$Π\fCenter\struct{A}\impr\struct{B}$}
      \RightLabel{R$\impr$}
      \UIC{$Π\fCenter\struct{A\impr B}$}
      \RightLabel{Unf$^R$}
      \UIC{$Π\fCenter\focus{A\impr B}$}
      \AXC{$\vdots$}\noLine\UIC{$Γ\fCenter\focus{A}$}
      \AXC{$\vdots$}\noLine\UIC{$\focus{B}\fCenter Δ$}
      \RightLabel{L$\impr$}
      \BIC{$\focus{A\impr B}\fCenter Γ\impr Δ$}
      \RightLabel{Cut}
      \BIC{$Π\fCenter Γ\impr Δ$}
    \end{pfbox}
    \\[1\baselineskip] $\Longrightarrow$ \\
    \begin{pfbox}
      \AXC{$\vdots$}\noLine\UIC{$Γ\fCenter\focus{A}$}
      \AXC{$\vdots$}\noLine\UIC{$Π\fCenter\struct{A}\impr\struct{B}$}
      \RightLabel{Res$\impr\prod$}
      \UIC{$\struct{A}\prodΠ\fCenter\struct{B}$}
      \RightLabel{Unf$^R$}
      \LeftLabel{wrong? $\ra$}
      \UIC{$\struct{A}\prodΠ\fCenter\focus{B}$}
      \AXC{$\vdots$}\noLine\UIC{$\focus{B}\fCenter Δ$}
      \RightLabel{Cut}
      \BIC{$\struct{A}\prod Π\fCenter Δ$}
      \RightLabel{Res$\prod\impl$}
      \UIC{$\struct{A}\fCenter Δ\impl Π$}
      \RightLabel{Unf$^L$ $\la$ wrong?}
      \UIC{$\focus{A}\fCenter Δ\impl Π$}
      \RightLabel{Cut}
      \BIC{$Γ\fCenter Δ\impl Π$}
      \RightLabel{Res$\impl\prod$}
      \UIC{$Γ\prod Π\fCenter Δ$}
      \RightLabel{Res$\prod\impr$}
      \UIC{$Π\fCenter Γ\impr Δ$}
    \end{pfbox}
  }
\end{center}
If this is the case, then we can no longer prove that focused NL is a
display calculus. For CNL, \citet{bastenhof2011} solves this problem
by proving that there is a normalisation procedure from display CNL to
focused CNL. Together with the trivial injection from focused CNL into
display CNL, this shows that the two are equivalent. Thus, focused CNL
is a display calculus by virtue of being equivalent to display CNL.
However, defining this procedure requires a proof of cut-elimination.

The work by \citeauthor{bastenhof2011} assures us that focused NL is
indeed a display calculus. However, since the proof of this fact
requires a proof of cut-elimination, using this approach would lose
the main advantage of display calculus. Worse so, if we make
extensions to display NL beyond CNL, we can no longer depend on the
assurance offered by \citeauthor{bastenhof2011}.

Because it does not fall within the scope of this thesis, we will
leave the problem of finding an elegant solution to this as future
work. For the remainder of this thesis, will depend on the display
calculus for the formal properties, and consider the focused calculus
to merely be an optimisation, believed to be complete w.r.t.\ the
display calculus.
This way, if an solution is found in the future, the entire body of
work---including focusing---remains valid. However, if the opposite is
proven, the display version of the calculus will still be of use.

%%% Local Variables:
%%% mode: latex
%%% TeX-master: t
%%% End:
