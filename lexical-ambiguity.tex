\section{Lexical Ambiguity}
In this section, we will introduce `\&' (pronounced `with') from
linear logic, and show how this can be used to encode lexical
ambiguity. The original framework for categorial grammar
\citep{lambek1958} already had machinery in place to deal with
ambiguity; it allowed for multiple lexical entries for each
word. However, in the spirit of wanting to deal with linguistic
phenomena in a logical manner, it seems to make more sense to
absorb lexical ambiguity into the logic itself. Not only is there a
perfectly well-understood connective that can deal with this
phenomenon, but it allows us to define ambiguity at any level in the
type.

\begin{figure}[h]
  \begin{mdframed}
    \centering
    \[\text{Type}\;A,B\coloneqq\ldots\vsep A\& B\vsep A\oplus B\]
    \begin{pfbox}
      \AXC{$\struct{A}\fCenter Δ$}
      \RightLabel{L\&$_1$}
      \UIC{$\struct{A\& B}\fCenter Δ$}
    \end{pfbox}
    \begin{pfbox}
      \AXC{$\struct{B}\fCenter Δ$}
      \RightLabel{L\&$_2$}
      \UIC{$\struct{A\& B}\fCenter Δ$}
    \end{pfbox}
    \begin{pfbox}
      \AXC{$Γ\fCenter\struct{A}$}
      \AXC{$Γ\fCenter\struct{B}$}
      \RightLabel{R\&}
      \BIC{$Γ\fCenter\struct{A\& B}$}
    \end{pfbox}
    \\[1\baselineskip]
    \begin{pfbox}
      \AXC{$\struct{A}\fCenter Δ$}
      \AXC{$\struct{B}\fCenter Δ$}
      \RightLabel{L$\oplus$}
      \BIC{$\struct{A\oplus B}\fCenter Δ$}
    \end{pfbox}
    \begin{pfbox}
      \AXC{$Γ\fCenter\struct{A}$}
      \RightLabel{R$\oplus_1$}
      \UIC{$Γ\fCenter\struct{A\oplus B}$}
    \end{pfbox}
    \begin{pfbox}
      \AXC{$Γ\fCenter\struct{B}$}
      \RightLabel{R$\oplus_2$}
      \UIC{$Γ\fCenter\struct{A\oplus B}$}
    \end{pfbox}
    \\[1\baselineskip]
    \hrulefill
    \[
      \tr[(A\& B)] \mapsto \tr[A]\times\tr[B]
    \]
    \begin{pfblock}
      \AXC{$x:\struct{A}\fCenter M:Δ$}
      \RightLabel{L\&$_1$}
      \UIC{$z:\struct{A\& B}\fCenter \case{z}{x}{\_}{M}:Δ$}
    \end{pfblock}
    \begin{pfblock}
      \AXC{$y:\struct{B}\fCenter M:Δ$}
      \RightLabel{L\&$_2$}
      \UIC{$z:\struct{A\& B}\fCenter \case{z}{\_}{y}{M}:Δ$}
    \end{pfblock}
    \begin{pfblock}
      \AXC{$x:Γ\fCenter\struct{M:A}$}
      \AXC{$x:Γ\fCenter\struct{N:B}$}
      \RightLabel{R\&}
      \BIC{$x:Γ\fCenter\struct{(M,N):A\& B}$}
    \end{pfblock}
    \vspace*{0.5\baselineskip}
  \end{mdframed}
  \caption{
    Extension of calculus in \autoref{fig:nl-display-calculus} which supports ambiguity.}%
  \label{fig:extension-lexical-ambiguity}
\end{figure}


We introduce `\&' in \autoref{fig:extension-lexical-ambiguity}, with
the associated inference rules, and term labelling in \lamET. You may
notice that the left rules seem to \textit{forget} either the left- or
the right-hand argument, and that in the right rule the context $Γ$ is
\textit{shared} between the two branches. While these look like
applications of weakening and contraction, it is perfectly
benign. What `\&' encodes is a \textit{choice}: The right rule says
that if you can construct $A$ from $Γ$, and you can construct $B$ from
$Γ$, and you have \textit{one} $Γ$, then you have a choice between
constructing either an $A$ or a $B$. The left rules commit to
\textit{one} of these options.

Again, we have to prove condition \textbf{C8}, in order to show that
this extension is compatible with display calculus. This time, the
proof is even easier. For L\&$_1$ and R\&:
\begin{center}
  \begin{pfbox}
    \AXC{$\vdots$}\noLine\UIC{$Γ\fCenter\struct{A}$}
    \AXC{$\vdots$}\noLine\UIC{$Γ\fCenter\struct{B}$}
    \RightLabel{R\&}
    \BIC{$Γ\fCenter\struct{A\& B}$}
    \AXC{$\vdots$}\noLine\UIC{$\struct{A}\fCenter Δ$}
    \RightLabel{L\&$_1$}
    \UIC{$\struct{A\& B}\fCenter Δ$}
    \RightLabel{Cut}
    \BIC{$Γ\fCenter Δ$}
  \end{pfbox}
  \\[1\baselineskip] $\Longrightarrow$ \\
  \begin{pfbox}
    \AXC{$\vdots$}\noLine\UIC{$Γ\fCenter\struct{A}$}
    \AXC{$\vdots$}\noLine\UIC{$\struct{A}\fCenter Δ$}
    \RightLabel{Cut}
    \BIC{$Γ\fCenter Δ$}
  \end{pfbox}
\end{center}
And likewise for L\&$_2$ and R\&.

When is this extension useful? Imagine a word like `want'. This can be
used in two different ways, with two different meanings:\\
\begin{center}
  \vspace{-1\baselineskip}
  \renewcommand{\arraystretch}{1}
  \begin{tabular}{c c}
    ``Mary wants John to leave.'' & ``Mary wants to leave.''\\
    $\WANT(\mary,\leave(\john))$  & $\WANT(\mary,\leave(\mary))$
  \end{tabular}
\end{center}
`Wants' has an implicit reflexive object: if no object is explicitly
given, it is assumed to be reflexive. Other words that show this
behaviour are words such as `to wash' and `to shave'.

Using our new connective, we can give a single definition for such
words, which takes this ambiguity into account:
\begin{alignat*}{3}
  &\wants\;&:\;&\tr[((\TV\impl\INF))\&(\IV\impl\INF)]\\
  &\wants\;&=\;&((\lambda p\;y\;x.\WANT(x,p\;y)),(\lambda p\;x.\WANT(x,p\;x)))
\end{alignat*}

How much of an advantage does this method have over allowing multiple
lexical entries for a word? It is a more elegant solution for words
like `want', which are ambiguous, but where all options encode a
similar concept---in both cases, the subject wants something to
happen. However, the similarity between the two definitions, and the
fact that implicit reflexive objects occur with numerous transitive
verbs, might be an indication that this is a quirk of grammar, instead
of lexical ambiguity.\todo{Reference? Proper linguistic terminology?}

Using `\&' is a lot less elegant for true lexical ambiguity. For
homonyms like `bow'---which have many different meanings, some of
them radically different---all of its meanings would have to be
encoded in a single definition.
However, this is easily remedied, as it is not hard to imagine a
procedure which encodes ambiguity in the style of
\citeauthor{lambek1958} using `\&'.

On the other hand, `\&' is more elegant in its formalisation.
\citeauthor{lambek1958}'s solution involves computing all possible
trees, given all possible definitions of each word, and attempting to
find a proof for all of them. The solution using `\&' elegantly
absorbs this traversal into the proof search.
