\section{Lexical Ambiguity}
\label{sec:lexical-ambiguity}
In this section, we will introduce `\&' (additive conjunction), and
show how this can be used to encode lexical ambiguity.

The original framework for categorial grammar \citep{lambek1958}
already had machinery in place to deal with ambiguity; it allowed for
multiple lexical entries for each word. However, in the spirit of
wanting to deal with linguistic phenomena in a logical manner, it
seems to make more sense to absorb lexical ambiguity into the logic
itself.
\citet[][p.\ 170]{lambek1961} already does this---he adds the additive
conjunction, which he writes `$\cap$'. In
\autoref{fig:extension-lexical-ambiguity}, we describe the same
connective, but define it within the framework of focused display
calculus. The notation `\&' comes from \citet{girard1987}.

\begin{figure}[h]
  \begin{mdframed}
    \centering
    \[\text{Type}\;A,B\coloneqq\ldots\vsep A\& B\vsep A\oplus B\]
    \begin{pfbox}
      \AXC{$\struct{A}\fCenter Δ$}
      \RightLabel{L\&$_1$}
      \UIC{$\struct{A\& B}\fCenter Δ$}
    \end{pfbox}
    \begin{pfbox}
      \AXC{$\struct{B}\fCenter Δ$}
      \RightLabel{L\&$_2$}
      \UIC{$\struct{A\& B}\fCenter Δ$}
    \end{pfbox}
    \begin{pfbox}
      \AXC{$Γ\fCenter\struct{A}$}
      \AXC{$Γ\fCenter\struct{B}$}
      \RightLabel{R\&}
      \BIC{$Γ\fCenter\struct{A\& B}$}
    \end{pfbox}
    \\[1\baselineskip]
    \begin{pfbox}
      \AXC{$\struct{A}\fCenter Δ$}
      \AXC{$\struct{B}\fCenter Δ$}
      \RightLabel{L$\oplus$}
      \BIC{$\struct{A\oplus B}\fCenter Δ$}
    \end{pfbox}
    \begin{pfbox}
      \AXC{$Γ\fCenter\struct{A}$}
      \RightLabel{R$\oplus_1$}
      \UIC{$Γ\fCenter\struct{A\oplus B}$}
    \end{pfbox}
    \begin{pfbox}
      \AXC{$Γ\fCenter\struct{B}$}
      \RightLabel{R$\oplus_2$}
      \UIC{$Γ\fCenter\struct{A\oplus B}$}
    \end{pfbox}
    \\[1\baselineskip]
    \hrulefill
    \[
      \tr[(A\& B)] \mapsto \tr[A]\times\tr[B]
    \]
    \begin{pfblock}
      \AXC{$x:\struct{A}\fCenter M:Δ$}
      \RightLabel{L\&$_1$}
      \UIC{$z:\struct{A\& B}\fCenter \case{z}{x}{\_}{M}:Δ$}
    \end{pfblock}
    \begin{pfblock}
      \AXC{$y:\struct{B}\fCenter M:Δ$}
      \RightLabel{L\&$_2$}
      \UIC{$z:\struct{A\& B}\fCenter \case{z}{\_}{y}{M}:Δ$}
    \end{pfblock}
    \begin{pfblock}
      \AXC{$x:Γ\fCenter\struct{M:A}$}
      \AXC{$x:Γ\fCenter\struct{N:B}$}
      \RightLabel{R\&}
      \BIC{$x:Γ\fCenter\struct{(M,N):A\& B}$}
    \end{pfblock}
    \vspace*{0.5\baselineskip}
  \end{mdframed}
  \caption{
    Extension of calculus in \autoref{fig:nl-display-calculus} which supports ambiguity.}%
  \label{fig:extension-lexical-ambiguity}
\end{figure}


Again, we have to prove condition \textbf{C8}, in order to show that
this extension is compatible with display calculus. This time, the
proof is even easier. For L\&$_1$ and R\&:
\begin{center}
  \begin{pfbox}
    \AXC{$\vdots$}\noLine\UIC{$Γ\fCenter\struct{A}$}
    \AXC{$\vdots$}\noLine\UIC{$Γ\fCenter\struct{B}$}
    \RightLabel{R\&}
    \BIC{$Γ\fCenter\struct{A\& B}$}
    \AXC{$\vdots$}\noLine\UIC{$\struct{A}\fCenter Δ$}
    \RightLabel{L\&$_1$}
    \UIC{$\struct{A\& B}\fCenter Δ$}
    \RightLabel{Cut}
    \BIC{$Γ\fCenter Δ$}
  \end{pfbox}
  \\[1\baselineskip] $\Longrightarrow$ \\
  \begin{pfbox}
    \AXC{$\vdots$}\noLine\UIC{$Γ\fCenter\struct{A}$}
    \AXC{$\vdots$}\noLine\UIC{$\struct{A}\fCenter Δ$}
    \RightLabel{Cut}
    \BIC{$Γ\fCenter Δ$}
  \end{pfbox}
\end{center}
And likewise for L\&$_2$ and R\&.

When is this extension useful? Imagine a word like `want'. This can be
used in two different ways, with two different meanings:\\
\begin{center}
  \vspace{-1\baselineskip}
  \renewcommand{\arraystretch}{1}
  \begin{tabular}{c c}
    ``Mary wants John to leave.'' & ``Mary wants to leave.''\\
    $\WANT(\MARY,\LEAVE(\JOHN))$  & $\WANT(\MARY,\LEAVE(\MARY))$
  \end{tabular}
\end{center}
`Wants' has an implicit reflexive object: if no object is explicitly
given, it is assumed to be reflexive. Other words that show this
behaviour are words such as `to wash' and `to shave'.
Using our new connective, we can give a single definition for such
words, which takes this ambiguity into account:
\begin{alignat*}{3}
  &\text{wants}\;&:\;&\tr[(((\IV\impl\INF\impl\NP))\&(\IV\impl\INF))]\\
  &\text{wants}\;&=\;&((\lambda y\;f\;x.\WANT(x,f\;y)),(\lambda f\;x.\WANT(x,f\;x)))
\end{alignat*}
For a detailed discussion of the expressiveness of Lambek calculi
enriched with `\&' (and its dual, `$\oplus$') we refer the reader to
\citet{kanazawa1992}.

Note that \autoref{fig:extension-lexical-ambiguity} does not present a
focusing regime. This is because, in our formulation, the focusing
regime for additives is significantly more complicated than the regime
for multiplicatives---perhaps an artefact of the fact that it was
formulated for a calculus with only multiplicatives. At any rate,
below we present a focusing regime for `\&', based on recent work by
\citet{morrill2015}:
\\
\makebox[\textwidth][c]{
  \(\!
    \text{if}~\text{Pol}(A)~=~{+}
    \left\lbrace
      \quad
      \begin{aligned}
        \\
        \AXC{$\struct{A}\fCenter Δ$}
        \RightLabel{L\&$_1$}
        \UIC{$\focus{A\& B}\fCenter Δ$}
        \DisplayProof
        \\[1\baselineskip]
        \AXC{$\struct{B}\fCenter Δ$}
        \RightLabel{L\&$_2$}
        \UIC{$\focus{A\& B}\fCenter Δ$}
        \DisplayProof
        \\[1\baselineskip]
      \end{aligned}
      \quad
      \middle\vert
      \quad
      \begin{aligned}
        \\
        \AXC{$\focus{A}\fCenter Δ$}
        \RightLabel{L\&$_1$}
        \UIC{$\focus{A\& B}\fCenter Δ$}
        \DisplayProof
        \\[1\baselineskip]
        \AXC{$\focus{B}\fCenter Δ$}
        \RightLabel{L\&$_2$}
        \UIC{$\focus{A\& B}\fCenter Δ$}
        \DisplayProof
        \\[1\baselineskip]
      \end{aligned}
      \quad
    \right\rbrace
    \text{if}~\text{Pol}(A)~=~{-}
  \)
}
\begin{center}
  \begin{pfbox}
    \AXC{$Γ\fCenter\struct{A}$}
    \AXC{$Γ\fCenter\struct{B}$}
    \RightLabel{R\&}
    \BIC{$Γ\fCenter\struct{A\& B}$}
  \end{pfbox}
  \begin{pfbox}
    \AXC{$\focus{A}\fCenter\struct{B}$}
    \AXC{$\focus{A}\fCenter\struct{C}$}
    \RightLabel{R\&}
    \BIC{$\focus{A}\fCenter\struct{B\& C}$}
  \end{pfbox}
\end{center}
The R\&-rules expose a restriction in our syntax: because we
encode focusing as left- and right-focused sequents, we cannot
abstract over it, and therefore need two copies of the R\& rule.
On the other hand, \citet{morrill2015} do not seem to have a syntactic
restriction that ensures that at most one formula is be focused at a
time, and the approach taken by \citet{laurent2004} critically depends
on allowing empty structures---something that we do not want to allow
in our grammar logic (see \autoref{sec:introduction}).

At the moment it is unclear how to extend \citepos{moortgat2012}
CPS-semantics to include additives.\footnote{%
  It is for this reason that we do not include additives in our
  Agda formalisation in appendix A, since we want to demonstrate
  CPS-semantics.
}


% How much of an advantage does this method have over allowing multiple
% lexical entriaes for a word? It is a more elegant solution for words
% such as `want', where there is an ambiguity, but where all options
% encode a similar concept---in the case of `want', in both
% interpretations, the subject wants something to be true.
% However, the similarity between the two definitions, and the
% fact that implicit reflexive objects occur with numerous transitive
% verbs, might be an indication that this is a quirk of grammar, instead
% of lexical ambiguity.\todo{Reference? Proper linguistic terminology?}
%
% Additionally, `\&' is slightly more elegant in its formalisation.
% Where \citepos{lambek1958} solution involves computing all possible
% sequents, given all possible definitions of each word, and attempting
% to find a proof for all of them. The solution using `\&' elegantly
% absorbs this traversal into the proof search. On the other hand, using
% `\&' is a lot less elegant for true lexical ambiguity. For homonyms
% like `bow'---which have many different meanings, some of them
% radically different---all of its meanings would have to be encoded in
% a single definition. However, this is easily remedied, as it is not
% hard to imagine a procedure which folds an ambiguous lexicon in the
% style of \citeauthor{lambek1958} down to an unambiguous one using
% `\&'.
